%% Generated by Sphinx.
\def\sphinxdocclass{jupyterBook}
\documentclass[letterpaper,10pt,german]{jupyterBook}
\ifdefined\pdfpxdimen
   \let\sphinxpxdimen\pdfpxdimen\else\newdimen\sphinxpxdimen
\fi \sphinxpxdimen=.75bp\relax
\ifdefined\pdfimageresolution
    \pdfimageresolution= \numexpr \dimexpr1in\relax/\sphinxpxdimen\relax
\fi
%% let collapsible pdf bookmarks panel have high depth per default
\PassOptionsToPackage{bookmarksdepth=5}{hyperref}
%% turn off hyperref patch of \index as sphinx.xdy xindy module takes care of
%% suitable \hyperpage mark-up, working around hyperref-xindy incompatibility
\PassOptionsToPackage{hyperindex=false}{hyperref}
%% memoir class requires extra handling
\makeatletter\@ifclassloaded{memoir}
{\ifdefined\memhyperindexfalse\memhyperindexfalse\fi}{}\makeatother

\PassOptionsToPackage{booktabs}{sphinx}
\PassOptionsToPackage{colorrows}{sphinx}

\PassOptionsToPackage{warn}{textcomp}

\catcode`^^^^00a0\active\protected\def^^^^00a0{\leavevmode\nobreak\ }
\usepackage{cmap}
\usepackage{fontspec}
\defaultfontfeatures[\rmfamily,\sffamily,\ttfamily]{}
\usepackage{amsmath,amssymb,amstext}
\usepackage{polyglossia}
\setmainlanguage[spelling=new]{german}



\setmainfont{FreeSerif}[
  Extension      = .otf,
  UprightFont    = *,
  ItalicFont     = *Italic,
  BoldFont       = *Bold,
  BoldItalicFont = *BoldItalic
]
\setsansfont{FreeSans}[
  Extension      = .otf,
  UprightFont    = *,
  ItalicFont     = *Oblique,
  BoldFont       = *Bold,
  BoldItalicFont = *BoldOblique,
]
\setmonofont{FreeMono}[
  Extension      = .otf,
  UprightFont    = *,
  ItalicFont     = *Oblique,
  BoldFont       = *Bold,
  BoldItalicFont = *BoldOblique,
]



\usepackage[Sonny]{fncychap}
\ChNameVar{\Large\normalfont\sffamily}
\ChTitleVar{\Large\normalfont\sffamily}
\usepackage[,numfigreset=1,mathnumfig]{sphinx}

\fvset{fontsize=\small}
\usepackage{geometry}


% Include hyperref last.
\usepackage{hyperref}
% Fix anchor placement for figures with captions.
\usepackage{hypcap}% it must be loaded after hyperref.
% Set up styles of URL: it should be placed after hyperref.
\urlstyle{same}

\addto\captionsgerman{\renewcommand{\contentsname}{Einführung}}

\usepackage{sphinxmessages}


% \usepackage{symbols}        

        % Start of preamble defined in sphinx-jupyterbook-latex %
         \usepackage[Latin,Greek]{ucharclasses}
        \usepackage{unicode-math}
        % fixing title of the toc
        \addto\captionsenglish{\renewcommand{\contentsname}{Contents}}
        \hypersetup{
            pdfencoding=auto,
            psdextra
        }
        % End of preamble defined in sphinx-jupyterbook-latex %
        

\title{Finite Elemente Methode 1}
\date{05.06.2025}
\release{}
\author{Prof.\@{} Dr.\sphinxhyphen{}Ing.\@{} Steffen Beese}
\newcommand{\sphinxlogo}{\vbox{}}
\renewcommand{\releasename}{}
\makeindex
\begin{document}

\pagestyle{empty}
\sphinxmaketitle
\pagestyle{plain}
\sphinxtableofcontents
\pagestyle{normal}
\phantomsection\label{\detokenize{intro::doc}}


\sphinxAtStartPar
Dies ist die erste Version des Skriptes zur Vorlesung: \sphinxstylestrong{Finite Elemente Methode I}. Das Skript wird während der Vorlesung angepasst und erweitert. Es lohnt sich also von Zeit zu Zeit wieder hier vorbei zu schauen.

\begin{sphinxadmonition}{note}{Inhalt nach der Modulbeschreibung}
\begin{itemize}
\item {} 
\sphinxAtStartPar
Grundsätzliche Berechnungsaufgaben; Anwendungsgebiete

\item {} 
\sphinxAtStartPar
Generelle Vorgehensweise (problemorientierte Differentialgleichung,
Näherungsansatz, Prinzip vom Minimum der potentiellen Energie…)

\item {} 
\sphinxAtStartPar
ausführliches Beispiel (Idealisierung, Diskretisierung, Formfunktion,
Näherungsansatz, Steifigkeitsmatrix und Gleichungssystem…)

\item {} 
\sphinxAtStartPar
Strategien zur Erhöhung der Genauigkeit (Elementanzahl,
Netzdichte…)

\item {} 
\sphinxAtStartPar
Koordinatensysteme, Koordinatentransformationen

\item {} 
\sphinxAtStartPar
Elementbibliothek (Stäbe, Balken, Platten, Schalen,
Volumenelemente…)

\item {} 
\sphinxAtStartPar
allgemeine Vorgehensweise (Preprocessing, Solution, Postprocessing)

\item {} 
\sphinxAtStartPar
direkte und indirekte Netzgenerierung

\item {} 
\sphinxAtStartPar
statische Analysen; CAD\sphinxhyphen{}FEM\sphinxhyphen{}Kopplung; Entwicklungstendenzen

\item {} 
\sphinxAtStartPar
ausführliche Beispiele mit dem FEM\sphinxhyphen{}System ANSYS

\end{itemize}
\end{sphinxadmonition}


\begin{itemize}
\item {} 
\sphinxAtStartPar
Einführung

\begin{itemize}
\item {} 
\sphinxAtStartPar
{\hyperref[\detokenize{chapters/chapter1/Einf_xfchrung_Konstruktion::doc}]{\sphinxcrossref{Numerische Simulation im Konstruktionsprozess}}}

\end{itemize}
\end{itemize}
\begin{itemize}
\item {} 
\sphinxAtStartPar
Theorie der Finiten Elemente Methode

\begin{itemize}
\item {} 
\sphinxAtStartPar
{\hyperref[\detokenize{chapters/chapter2/overview::doc}]{\sphinxcrossref{Die Finite Elemente Methode}}}

\end{itemize}
\end{itemize}
\begin{itemize}
\item {} 
\sphinxAtStartPar
Isoparametrische FEM

\begin{itemize}
\item {} 
\sphinxAtStartPar
{\hyperref[\detokenize{chapters/chapter3/isoparametrischeFEM::doc}]{\sphinxcrossref{Isoparametrische FEM}}}

\end{itemize}
\end{itemize}
\begin{itemize}
\item {} 
\sphinxAtStartPar
Diskretisierung

\begin{itemize}
\item {} 
\sphinxAtStartPar
{\hyperref[\detokenize{chapters/chapter4/Diskretisierung::doc}]{\sphinxcrossref{Diskretisierung}}}

\end{itemize}
\end{itemize}
\begin{itemize}
\item {} 
\sphinxAtStartPar
Quellenverzeichnis

\begin{itemize}
\item {} 
\sphinxAtStartPar
{\hyperref[\detokenize{quellen::doc}]{\sphinxcrossref{Quellenverzeichnis}}}

\end{itemize}
\end{itemize}

\sphinxstepscope


\part{Einführung}

\sphinxstepscope


\chapter{Numerische Simulation im Konstruktionsprozess}
\label{\detokenize{chapters/chapter1/Einf_xfchrung_Konstruktion:numerische-simulation-im-konstruktionsprozess}}\label{\detokenize{chapters/chapter1/Einf_xfchrung_Konstruktion::doc}}
\begin{figure}[htbp]
\centering
\capstart

\noindent\sphinxincludegraphics[height=300\sphinxpxdimen]{{Konstruktionsprozess_knothe}.png}
\caption{Einordnung numerischer Simulation in den Konstruktionsprozess nach {[}\hyperlink{cite.quellen:id4}{Knothe and Wessels, 1991}{]}}\label{\detokenize{chapters/chapter1/Einf_xfchrung_Konstruktion:konstruktionsprozess}}\end{figure}

\sphinxAtStartPar
Finite\sphinxhyphen{}Elemente\sphinxhyphen{}Programme spielen eine wichtige Rolle bei der Untersuchung realer Tragwerke, die bestimmten technischen Aufgabenstellungen unterliegen. Dieses leistungsfähige Tool ermöglicht es beispielsweise, die Integrität einer Pkw\sphinxhyphen{}Karosserie oder eines Rahmentragwerks zu analysieren, indem sie die Simulation eines Crash\sphinxhyphen{}Vorgangs durchführen oder die Traglast eines Bauteils ermitteln. Für die virtuelle Produktentwicklung und eine prototypenfreie (oder prototypenarme) Entwicklung ist die Finite Elemente Methode aus den Entwicklungsabteilungen nicht mehr wegzudenken.

\sphinxAtStartPar
Der Weg zu einer validen FE\sphinxhyphen{}Simulation ist mehrstufig und interdisziplinar. In der Abbildung \hyperref[\detokenize{chapters/chapter1/Einf_xfchrung_Konstruktion:konstruktionsprozess}]{Abb.\@ \ref{\detokenize{chapters/chapter1/Einf_xfchrung_Konstruktion:konstruktionsprozess}}} sind die eigentlichen Schritte, welche die Finite Elemente Methode betrifft durch einen dicken Rahmen gekennzeichnet. Die vorgelagerte Prozesskette sollte von dem/der berechnenden Ingenieur/in eng begleitet werden, denn die Ergebnisse eine Simulation können nur so gut sein wie die Eingangsdaten: \sphinxstylestrong{Garbage in, garbage out}.

\sphinxAtStartPar
Zu erst wird die reale Struktur und ihre Belastung definiert und anschließend in geeigneter Weise abstrahiert:
\begin{itemize}
\item {} 
\sphinxAtStartPar
genügt eine 2D Berechnung oder gar eine 1D Berechnung

\end{itemize}


\section{Der Systembegriff}
\label{\detokenize{chapters/chapter1/Einf_xfchrung_Konstruktion:der-systembegriff}}\begin{quote}

\sphinxAtStartPar
Ein System besteht aus einer Menge von Elementen (Teilsystemen), die Eigenschaften besitzen und durch Beziehungen miteinander verknüpft sind. Das System wird durch eine Systemgrenze von der Umgebung abgegrenzt und steht mit der Umgebung durch Ein\sphinxhyphen{} und Ausgangsgrößen in Beziehung. Die Funktion eines Systems kann durch den Unterschied der dem Zweck entsprechenden Ein\sphinxhyphen{} und Ausgangsgrößen beschrieben werden. Die Systemelemente können selbst wiederum Systeme sein, die aus Elementen und Beziehungen bestehen. {[}\hyperlink{cite.quellen:id15}{Ehrlenspiel and Meerkamm, 2013}{]}
\end{quote}

\sphinxAtStartPar
Die Definition eines Systems umfasst mehrere wesentliche Schritte:
\begin{enumerate}
\sphinxsetlistlabels{\arabic}{enumi}{enumii}{}{.}%
\item {} 
\sphinxAtStartPar
\sphinxstylestrong{Identifikation der Systemgrenzen}: Hier wird festgelegt, was zum System gehört und was nicht. Diese Abgrenzung ist entscheidend, um den Fokus der Analyse zu bestimmen und irrelevante Einflüsse auszuschließen.

\item {} 
\sphinxAtStartPar
\sphinxstylestrong{Bestimmung der Systemelemente}: In diesem Schritt werden die Komponenten identifiziert, die Teil des Systems sind. Diese Komponenten können physikalische Objekte, biologische Organismen, soziale Gruppen,  technische Bauteile, etc. sein.

\item {} 
\sphinxAtStartPar
\sphinxstylestrong{Beschreibung der Beziehungen zwischen den Systemelementen}: Hier wird untersucht, wie die einzelnen Komponenten miteinander interagieren.

\end{enumerate}

\sphinxAtStartPar
Ein gut definiertes System ist entscheidend für die Genauigkeit und Zuverlässigkeit der Simulationsergebnisse. Nur wenn alle relevanten Komponenten und deren Interaktionen korrekt erfasst und modelliert werden, können valide Aussagen über das Systemverhalten getroffen werden.

\sphinxAtStartPar
\sphinxstylestrong{Zustand, Zustandsgröße, Zustandsvariable}

\sphinxAtStartPar
In der numerischen Simulation wird der Zustand eines Systems durch eine Reihe von Zustandsgrößen oder Zustandsvariablen beschrieben. Diese Variablen repräsentieren die Eigenschaften des Systems zu einem bestimmten Zeitpunkt.
\begin{itemize}
\item {} 
\sphinxAtStartPar
\sphinxstylestrong{Zustand}: Der Zustand eines Systems ist eine vollständige Beschreibung des Systems zu einem bestimmten Zeitpunkt. Er umfasst alle relevanten Informationen, die notwendig sind, um das Verhalten des Systems zu verstehen und vorherzusagen. Der Zustand ist somit eine Momentaufnahme des Systems, die alle wesentlichen Eigenschaften und Charakteristika beinhaltet.

\item {} 
\sphinxAtStartPar
\sphinxstylestrong{Zustandsgröße / Zustandsvariable}: Eine Zustandsgröße ist eine messbare Eigenschaft des Systems, die den Zustand des Systems beschreibt. Beispiele für Zustandsgrößen sind Temperatur, Druck, Spannung und Verformung. Diese Größen sind oft physikalische Messwerte, die direkt beobachtet oder gemessen werden können. Es handelt sich bei Zustandsvariablen zudem meist um Felder, dass heißt sie haben eine räumliche und zeitliche Abhängigkeit.

\item {} 
\sphinxAtStartPar
\sphinxstylestrong{Systemverhalten}: Das Systemverhalten beschreibt, wie sich das System im Laufe der Zeit entwickelt. Es umfasst die Dynamik und die Reaktionen des Systems auf äußere Einflüsse und interne Prozesse. Das Systemverhalten ist somit die zeitliche Entwicklung des Zustands des Systems und kann durch die zeitliche Abfolge von Punkten im Zustandsraum beschrieben werden.

\end{itemize}


\section{Klassifikation von Systemen}
\label{\detokenize{chapters/chapter1/Einf_xfchrung_Konstruktion:klassifikation-von-systemen}}
\sphinxAtStartPar
System können nach verschiedenen Ordnungskriterien klassifiziert werden. In der nachfolgenden Tabelle sind die wichtigsten Kriterien und deren Beschreibung aufgeführt.


\begin{savenotes}\sphinxattablestart
\sphinxthistablewithglobalstyle
\centering
\begin{tabulary}{\linewidth}[t]{TT}
\sphinxtoprule
\sphinxstyletheadfamily 
\sphinxAtStartPar
\sphinxstylestrong{Klassifikation}
&\sphinxstyletheadfamily 
\sphinxAtStartPar
\sphinxstylestrong{Beschreibung}
\\
\sphinxmidrule
\sphinxtableatstartofbodyhook
\sphinxAtStartPar
\sphinxstylestrong{Statisch}
&
\sphinxAtStartPar
Systeme, deren Verhalten unabhängig von der Zeit ist; Zustände ändern sich nicht.
\\
\sphinxhline
\sphinxAtStartPar
\sphinxstylestrong{Quasistatisch}
&
\sphinxAtStartPar
Systeme, die sich langsam ändern, sodass sie zu jedem Zeitpunkt als statisch betrachtet werden können.
\\
\sphinxhline
\sphinxAtStartPar
\sphinxstylestrong{Dynamisch}
&
\sphinxAtStartPar
Systeme, deren Verhalten sich mit der Zeit ändert; Zustände entwickeln sich über die Zeit.
\\
\sphinxhline
\sphinxAtStartPar
\sphinxstylestrong{Zeitkontinuierlich}
&
\sphinxAtStartPar
Systeme, die zu jedem Zeitpunkt einen definierten Zustand haben; Änderungen erfolgen kontinuierlich.
\\
\sphinxhline
\sphinxAtStartPar
\sphinxstylestrong{Zeitdiskret}
&
\sphinxAtStartPar
Systeme, die nur zu bestimmten Zeitpunkten Zustandsänderungen aufweisen; Änderungen erfolgen in diskreten Schritten.
\\
\sphinxhline
\sphinxAtStartPar
\sphinxstylestrong{Ereignisorientiert}
&
\sphinxAtStartPar
Systeme, deren Verhalten durch bestimmte Ereignisse ausgelöst wird; Änderungen treten in Reaktion auf Ereignisse auf.
\\
\sphinxhline
\sphinxAtStartPar
\sphinxstylestrong{Deterministisch}
&
\sphinxAtStartPar
Systeme, deren Verhalten vollständig durch Anfangsbedingungen und Regeln bestimmt ist; keine Zufälligkeit.
\\
\sphinxhline
\sphinxAtStartPar
\sphinxstylestrong{Stochastisch}
&
\sphinxAtStartPar
Systeme, deren Verhalten durch Zufallsprozesse beeinflusst wird; es gibt eine gewisse Unsicherheit im Verhalten.
\\
\sphinxbottomrule
\end{tabulary}
\sphinxtableafterendhook\par
\sphinxattableend\end{savenotes}

\begin{figure}[htbp]
\centering
\capstart

\noindent\sphinxincludegraphics[height=300\sphinxpxdimen]{{Systemverhalten}.png}
\caption{Klassifikation von Systemen nach ihrem Verhalten}\label{\detokenize{chapters/chapter1/Einf_xfchrung_Konstruktion:systemverhalten}}\end{figure}


\section{Der Modellbegriff}
\label{\detokenize{chapters/chapter1/Einf_xfchrung_Konstruktion:der-modellbegriff}}
\sphinxAtStartPar
In der Entwicklung mechatronischer Systeme spielen die VDI\sphinxhyphen{}Richtlinien 2206 und 2211 eine zentrale Rolle. Diese Richtlinien bieten eine methodische Grundlage für die Modellbildung, die ein physikalisch\sphinxhyphen{}mathematisches Abbild eines technischen Bauelements, einer Baugruppe oder eines komplexen Systems darstellt.

\sphinxAtStartPar
\sphinxstylestrong{Modellbildung und ihre Bedeutung}

\sphinxAtStartPar
Modelle sind materielle oder immaterielle Gebilde, die geschaffen werden, um für einen bestimmten Zweck ein Original zu repräsentieren. Sie können als Abbildungen oder Nachbildungen von Originalen betrachtet werden. Die Modellbildung beinhaltet die Darstellung eines physikalisch\sphinxhyphen{}mathematischen Modells eines vorhandenen Systems oder eines zu entwickelnden Systems. Der Zweck der Modellbildung besteht darin, das Original durch das Modell zu ersetzen und es als Stellvertreter des Originals zu nutzen, um Rückschlüsse auf das Original zu ziehen.

\sphinxAtStartPar
Modelle können somit als zweckgerichtete, vereinfachte Abbildungen oder Nachbildungen von Originalen aufgefasst werden. Sie umfassen eine Vielzahl von Konstrukten, darunter Anschauungsmodelle, Prototypen, Konstruktionszeichnungen, Schaltpläne, mathematische Gleichungen, aber auch Gedankenmodelle bzw. mentale Modelle, Vorstellungen und Bilder.

\sphinxAtStartPar
\sphinxstylestrong{Anforderungen an Modelle}

\sphinxAtStartPar
Modelle müssen original\sphinxhyphen{} und realitätsnah sein, um charakteristische Eigenschaften und Verhalten des Originals genau zu beschreiben. Der Modellzweck hängt von den Lebensphasen des Produkts ab, und es ist wichtig, ein angemessenes Aufwand\sphinxhyphen{}/Nutzen\sphinxhyphen{}Verhältnis zu gewährleisten. Der Aufwand für Modellierung und Analyse ist eng mit dem Detaillierungsgrad (Granularität) des Modells verbunden. Eine sehr genaue Modellierung ist nicht immer notwendig; Unsicherheiten können den Nutzen eines detaillierten Modells in Frage stellen.

\sphinxAtStartPar
Das Modellverhalten muss im Gültigkeitsbereich dem realen Systemverhalten entsprechen, um die Modellgültigkeit zu gewährleisten. Das Verhalten resultiert aus den Eigenschaften der Modellelemente und deren Verknüpfungen. Bei mehreren geeigneten Modellierungsansätzen sollte die einfachste Methode bevorzugt werden, um die Modelleffizienz zu maximieren. Es gibt keine allgemeinen Regeln für die Herleitung eines einfachen, effizienten und gültigen Modells; vielmehr sind Erfahrung und Vorwissen entscheidend.


\section{Modellbildung}
\label{\detokenize{chapters/chapter1/Einf_xfchrung_Konstruktion:modellbildung}}
\begin{sphinxadmonition}{note}{Ziel der Modellbildung}

\sphinxAtStartPar
Modellbildung ist die Schaffung eines Modells, das dem Untersuchungszweck entspricht und demgemäß verändert und ausgewertet werden kann, um damit Rückschlüsse auf das Original ziehen zu können.
\end{sphinxadmonition}

\begin{figure}[htbp]
\centering
\capstart

\noindent\sphinxincludegraphics[height=300\sphinxpxdimen]{{Problemloesung}.png}
\caption{Problemlösung im Ingenieurwesen nach {[}\hyperlink{cite.quellen:id16}{Vajna \sphinxstyleemphasis{et al.}, 2009}{]}}\label{\detokenize{chapters/chapter1/Einf_xfchrung_Konstruktion:problemlosung}}\end{figure}

\sphinxAtStartPar
Die Modellabstraktion ist ein zentraler Bestandteil der Ingenieurwissenschaft, da sie die Realität einer Berechnung zugänglich macht. Durch die Abstraktion werden irrelevante Details vernachlässigt, während die relevanten Details erhalten bleiben. Das Ziel ist es, ein Modellergebnis zu erzielen, das eine hohe Relevanz für die Lösung der realen Problemstellung aufweist.

\sphinxAtStartPar
\sphinxstylestrong{Modellabstraktion als Grundlage für Analyse und Design}

\sphinxAtStartPar
Ein Modell dient als Grundlage für die Analyse und das Design von Systemen. Durch die Konzentration auf die spezifischsten und wichtigsten Merkmale und die Abstraktion von unwichtigen Eigenschaften und Details wird die Komplexität der Problemstellung reduziert. Diese Vereinfachung ist bereits ein Ziel der Analyse, da sie die Handhabbarkeit der Problemstellung ermöglicht. Ohne eine solche Vereinfachung wären viele Problemstellungen innerhalb der praktischen Beschränkungen von Zeit und Ressourcen äußerst komplex zu simulieren oder nicht effektiv zu analysieren.

\sphinxAtStartPar
\sphinxstylestrong{Vorteile einfacherer Modelle}

\sphinxAtStartPar
Einfachere Modelle sind von Natur aus leichter zu entwickeln, zu verstehen und zu modifizieren. Dies ist besonders wertvoll in iterativen Designprozessen, in denen häufige Aktualisierungen von Modellen erforderlich sind. Abstrakte Modelle können als gemeinsame Sprache für Ingenieure aus verschiedenen Disziplinen oder mit unterschiedlichem Fachwissen dienen. Sie reduzieren Missverständnisse und ermöglichen eine effektivere Zusammenarbeit in Ingenieurprojekten.

\sphinxAtStartPar
Darüber hinaus erfordern einfachere Modelle weniger Rechenleistung und Zeit für die Lösung. Dies ermöglicht mehr Iterationen, Sensitivitätsanalysen und die Untersuchung von Designalternativen innerhalb angemessener Zeitrahmen. Die Vereinfachung durch Modellabstraktion trägt somit wesentlich zur Effizienz und Effektivität des gesamten Design\sphinxhyphen{} und Analyseprozesses bei.

\begin{figure}[htbp]
\centering
\capstart

\noindent\sphinxincludegraphics[height=400\sphinxpxdimen]{{Modellbildung}.png}
\caption{Modellbildungsprozess nach {[}\hyperlink{cite.quellen:id16}{Vajna \sphinxstyleemphasis{et al.}, 2009}{]}}\label{\detokenize{chapters/chapter1/Einf_xfchrung_Konstruktion:id5}}\end{figure}

\sphinxAtStartPar
In Abbildung \hyperref[\detokenize{chapters/chapter1/Einf_xfchrung_Konstruktion:id5}]{Abb.\@ \ref{\detokenize{chapters/chapter1/Einf_xfchrung_Konstruktion:id5}}} ist der Modellbildungsprozess dargestellt. Die Modellplanung beginnt mit einer IST\sphinxhyphen{}Analyse, in der die Aufgabenstellung zu klären und zu präzisieren ist. Dabei treten typischerweise Fragen auf wie ({[}\hyperlink{cite.quellen:id16}{Vajna \sphinxstyleemphasis{et al.}, 2009}{]}):
\begin{itemize}
\item {} 
\sphinxAtStartPar
Was ist das zu untersuchende Original (System, Elemente, Systemgrenzen)?

\item {} 
\sphinxAtStartPar
Welche Fragestellungen sollen behandelt werden (Festlegung des Modellzwecks)?

\item {} 
\sphinxAtStartPar
Welche Sichtweisen auf das zu untersuchende Original (Systemaspekte, Bewertungskriterien) sind für den Modellzweck notwendig?

\item {} 
\sphinxAtStartPar
Welche Eigenschaften müssen für die gewünschte Bewertung (z. B. zur Eigenschaftsabsicherung) herangezogen werden?

\item {} 
\sphinxAtStartPar
Welche Effekte (Details) müssen daher berücksichtigt oder können vernachlässigt werden?

\item {} 
\sphinxAtStartPar
Welche Testsituationen sind zu untersuchen („Lastfälle“, Testszenarien, „use cases“)?

\item {} 
\sphinxAtStartPar
Welche Parameter bzw. Zustandsgrößen eines mathematischen Modells werden als vorgegeben (Parameter), welche als Zustandsvariablen betrachtet?

\item {} 
\sphinxAtStartPar
Welche Ergebnisse sind zur Klärung der Fragestellungen erforderlich und in welcher Form sollen die Ergebnisdaten aufbereitet und dokumentiert werden (Ergebnisdarstellung und Dokumentation)?

\item {} 
\sphinxAtStartPar
Welche Relevanz und Signifikanz haben die zu erwartenden Ergebnisse in Bezug auf die Fragestellungen? Wird die Fragestellung durch die Ergebnisse auch wirklich beantwortet?

\end{itemize}

\sphinxAtStartPar
\sphinxstylestrong{Verifikation und Validierung}

\sphinxAtStartPar
Jedes Modell stellt lediglich eine mehr oder weniger genaue Annäherung an das Original (z.B. ein reales System) dar. Daher ist es nach der Modellentwicklung notwendig zu überprüfen, ob das Modell mit seinen Idealisierungen das zu untersuchende Original ausreichend genau abbildet. Die Verifikation untersucht, ob sich das Modell grundsätzlich plausibel verhält, und bezieht sich dabei auf das Modellverhalten, unabhängig von Vergleichen mit einem konkreten Original. Die Modell\sphinxhyphen{}Verifikation betrifft somit die Überprüfung der Plausibilität des Modellverhaltens an sich, also für „fiktive“ Originale. Die Validierung hingegen liefert eine Aussage darüber, ob das erstellte Modell konkrete Originale hinreichend beschreibt und in welchem Bereich das Modell gültig ist (Grenzen des Modells).


\section{Ansätze zur Abstraktion mechanischer Modelle}
\label{\detokenize{chapters/chapter1/Einf_xfchrung_Konstruktion:ansatze-zur-abstraktion-mechanischer-modelle}}
\sphinxAtStartPar
Zur Abstraktion mechanischer Modelle können folgende Ansätze angewendet werden:
\begin{enumerate}
\sphinxsetlistlabels{\arabic}{enumi}{enumii}{}{.}%
\item {} 
\sphinxAtStartPar
Vereinfachung der Geometrie
\begin{itemize}
\item {} 
\sphinxAtStartPar
Abstraktion durch einfachere geometrische Grundkörper

\item {} 
\sphinxAtStartPar
Komplexer Maschinenrahmen als Skelettstruktur aus Balken oder ein dünnwandiges Druckgefäß als Schale modelliert werden

\end{itemize}

\item {} 
\sphinxAtStartPar
Dimensionsreduktion
\begin{itemize}
\item {} 
\sphinxAtStartPar
Übergang von einer 3D\sphinxhyphen{}Darstellung zu einer 2D\sphinxhyphen{} (ebener Spannungszustand, ebener Dehnungszustand, rotationssymmetrisch) oder sogar 1D\sphinxhyphen{}Darstellung (Balken, Stäbe)

\end{itemize}

\item {} 
\sphinxAtStartPar
Vereinfachung der Materialeigenschaften
\begin{itemize}
\item {} 
\sphinxAtStartPar
Wenn angemessen, linear\sphinxhyphen{}elastische Materialmodelle anstelle komplexerer nichtlinearer oder anisotroper Modelle

\end{itemize}

\item {} 
\sphinxAtStartPar
Abstraktion der Randbedingungen
\begin{itemize}
\item {} 
\sphinxAtStartPar
Darstellung komplexer Lagerungen oder Lasten durch idealisierte Randbedingungen

\item {} 
\sphinxAtStartPar
Z. B. feste, gelenkige oder rollende Lagerungen, oder durch die Verwendung von Punktlasten oder verteilten Lasten

\end{itemize}

\item {} 
\sphinxAtStartPar
Systemebenenabstraktion
\begin{itemize}
\item {} 
\sphinxAtStartPar
Subsysteme oder Komponenten werden als Black Boxes mit definierten Ein\sphinxhyphen{} und Ausgängen modelliert

\item {} 
\sphinxAtStartPar
Schwerpunkt auf ihrem Gesamtverhalten und nicht auf komplizierten internen Details

\item {} 
\sphinxAtStartPar
Ansatz besonders nützlich für die Analyse großer, komplexer Systeme, indem diese in kleinere, besser handhabbare Teile zerlegt werden

\end{itemize}

\item {} 
\sphinxAtStartPar
Vereinfachte physikalische Bilanzgleichung

\end{enumerate}

\sphinxstepscope


\section{Numerische Modelle in der Mechanik}
\label{\detokenize{chapters/chapter1/Einf_xfchrung_Modellklassifikation:numerische-modelle-in-der-mechanik}}\label{\detokenize{chapters/chapter1/Einf_xfchrung_Modellklassifikation::doc}}
\sphinxAtStartPar
Die numerische Mechanik bedient sich computergestützter Techniken zur Ermittlung von Näherungslösungen für Differentialgleichungen, die physikalische Problemstellungen beschreiben. Es wird zwischen zwei Hauptarten der Modellbildung unterschieden:
\begin{itemize}
\item {} 
\sphinxAtStartPar
Diskrete Modelle

\item {} 
\sphinxAtStartPar
Kontinuierliche Modelle

\end{itemize}


\subsection{Diskrete Modelle}
\label{\detokenize{chapters/chapter1/Einf_xfchrung_Modellklassifikation:diskrete-modelle}}
\sphinxAtStartPar
Diskrete Modelle der Mechanik basieren auf massebehafteten, starren Massepunkten oder Körpern, die durch Kraftpotentiale oder Federverbindungen miteinander in Wechselwirkung stehen. Solche Potentiale werden verwendet, um Interaktionen zwischen atomaren Partikeln (Molekulardynamik) oder Planeten (Gravitationsgesetze) zu modellieren.

\sphinxAtStartPar
Ein weiteres Anwendungsgebiet in der diskreten Modellierung der Mechanik sind Mechanismen: Systeme aus kinematisch verbundenen, starren oder verformbaren Körpern, verbunden durch Kontakte und diskret modellierte elastische Feder\sphinxhyphen{} und Dämpferelementen. Beispiele hierfür sind industrielle Fertigungsmaschinen oder der Mechanismus eines Kugelschreibers, von Spannelementen oder Fahrwerken.

\begin{figure}[htbp]
\centering
\capstart

\noindent\sphinxincludegraphics[height=300\sphinxpxdimen]{{MKS_Robot}.png}
\caption{Mehrkörperproblem eines zweiarmigen Roboters nach {[}\hyperlink{cite.quellen:id11}{Featherstone, 2014}{]}}\label{\detokenize{chapters/chapter1/Einf_xfchrung_Modellklassifikation:mks}}\end{figure}

\sphinxAtStartPar
Die mathematische Beschreibung solcher Modelle resultiert typischerweise in einem System gekoppelter Bewegungsgleichungen:
\label{equation:chapters/chapter1/Einführung_Modellklassifikation:f0161430-5f5f-4a80-a5c1-711ca2f47725}\begin{equation}
\boldsymbol{M} \ddot{\boldsymbol{x}} + \boldsymbol{D} \dot{\boldsymbol{x}} + \boldsymbol{K} \boldsymbol{x} = \boldsymbol{f}(t)
\end{equation}
\sphinxAtStartPar
Hierbei repräsentiert \(\boldsymbol{x}\) die Position des Massepunktes im Raum und \(\ddot{\boldsymbol{x}}\) dessen Beschleunigung. Die Trägheitsmatrix \(\boldsymbol{M}\) approximiert die Massenträgheit der Körper, während die Steifigkeitsmatrix \(\boldsymbol{K}\) die elastischen Bindungen darstellt. Die Matrix \(\boldsymbol{D}\) führt Dämpfungsterme in das System ein und \(\boldsymbol{f}(t)\) sind äußere zeitabhängige Kräfte.

\sphinxAtStartPar
Bei Mechanismen kommen oft Nebenbedingungen wie Kontaktbedingungen oder Regelungsalgorithmen hinzu, was zu Differential\sphinxhyphen{}Algebraischen\sphinxhyphen{}Gleichungssystemen (DAE) führt.

\sphinxAtStartPar
Die Herausforderung in der diskreten Mechanik liegt in der Modellierung: Das System muss abstrahiert, Gleichungen und Bedingungen formuliert und für den Rechner aufbereitet werden. Die numerische Aufgabe besteht dann hauptsächlich darin, die Bewegungsgleichungen über die Zeit zu integrieren. In der Festkörpermechanik sind zwei Methoden etabliert:
\begin{itemize}
\item {} 
\sphinxAtStartPar
Molekular\sphinxhyphen{}Dynamik Simulation (MDS) {[}\hyperlink{cite.quellen:id5}{Grotendorst \sphinxstyleemphasis{et al.}, 2009}{]}

\item {} 
\sphinxAtStartPar
Mehr\sphinxhyphen{}Körper\sphinxhyphen{}Simulation (MKS) {[}\hyperlink{cite.quellen:id6}{Shabana, 1997}{]}

\end{itemize}

\sphinxAtStartPar
Für strömungsmechanische Fragestellungen sind ebenfalls Partikelmethoden wie die Gitter\sphinxhyphen{}Boltzmann\sphinxhyphen{}Methode etabliert.


\subsubsection{Anwendungen für MKS}
\label{\detokenize{chapters/chapter1/Einf_xfchrung_Modellklassifikation:anwendungen-fur-mks}}
\sphinxAtStartPar
Die hier gezeigten Beispiele stammen aus der freien MKS\sphinxhyphen{}Software der TU\sphinxhyphen{}München \sphinxhref{https://www.mbsim-env.de/}{MBSim}



\sphinxAtStartPar
\sphinxstylestrong{Mehrkörpersimulation eines Planeten Getriebe}



\sphinxAtStartPar
\sphinxstylestrong{Mehrkörpersimulation eines rotierenden Tellers}


\subsection{Kontinuierliche Modelle}
\label{\detokenize{chapters/chapter1/Einf_xfchrung_Modellklassifikation:kontinuierliche-modelle}}
\sphinxAtStartPar
Die Modellbildung in diesem Bereich geht von einem gleichförmigen (homogenen) Materialverhalten aus. Feinheiten der komplexen atomaren Interaktionen, wie sie beispielsweise in Kristallstrukturen auftreten und auf sehr kleinen zeitlichen sowie räumlichen Skalen stattfinden, werden hierbei im Sinne einer statistischen Mittelung vereinfacht dargestellt. Die zugrundeliegende Vorstellung ist die eines Kontinuums, dessen Charakteristika mithilfe von partiellen Differentialgleichungen (Feldgleichungen) räumlich und zeitlich erfasst werden können. Dieses Feld gehört zur Disziplin der Kontinuumsmechanik.

\sphinxAtStartPar
In dieser Lehrveranstaltung konzentrieren wir uns auf die Behandlung dieser Feldgleichungen und ihre näherungsweise Lösung durch numerische Verfahren, insbesondere für Fragestellungen der linearen Elastizitätslehre. Es sollte jedoch erwähnt werden, dass vergleichbare Feldtheorien auch in anderen Bereichen, wie zum Beispiel bei elektromagnetischen Feldern, Anwendung finden und die Methoden entsprechend adaptiert werden können.

\sphinxAtStartPar
Unser Fokus liegt auf der Finite\sphinxhyphen{}Elemente\sphinxhyphen{}Methode (FEM), wobei auch die Finite\sphinxhyphen{}Differenzen\sphinxhyphen{}Methode am Rande behandelt wird. Die Finite\sphinxhyphen{}Elemente\sphinxhyphen{}Methode gilt aktuell als das effektivste Verfahren zur näherungsweisen Lösung partieller Differentialgleichungen in der Festkörpermechanik. Kommerzielle Softwarelösungen sind weit verbreitet und werden industriell genutzt. Ein wesentliches Lernziel dieses Kurses ist es daher, ein fundiertes Verständnis für die Finite\sphinxhyphen{}Elemente\sphinxhyphen{}Methode zu entwickeln, um sie sachgemäß in der Praxis anwenden zu können.

\sphinxAtStartPar
In der Fluidmechanik hat sich die FEM aufgrund der besonderen Beschaffenheit der Differentialgleichungen nicht durchgesetzt, was ebenfalls thematisiert wird. In diesem Gebiet sind traditionell die Finite\sphinxhyphen{}Differenzen\sphinxhyphen{}Methode und heute vermehrt die Finite\sphinxhyphen{}Volumen\sphinxhyphen{}Methode gebräuchlich.

\sphinxstepscope


\section{Numerische Lösungsverfahren}
\label{\detokenize{chapters/chapter1/Einf_xfchrung_Loesungsverfahren:numerische-losungsverfahren}}\label{\detokenize{chapters/chapter1/Einf_xfchrung_Loesungsverfahren::doc}}
\sphinxAtStartPar
Der kontinuierliche Modellansatz führt zu Feldgleichungen in Form von partielle Differentialgleichungen. Anwendungsgebiete sind hier die Strukturmechanik, der Stoff\sphinxhyphen{} und Wärmetransport oder auch die Elektrostatik  und vielen weitere. Diese partiellen Differentialgleichungen können in der Regel nicht geschlossen analytisch gelöst werden, weshalb sich numerische Berechnungsschemen hierfür etabliert haben. Die heute gebräuchlichsten werden in der Folge kurz beschrieben:
\begin{itemize}
\item {} 
\sphinxAtStartPar
Netzbasiertierte Methoden:
\begin{itemize}
\item {} 
\sphinxAtStartPar
Finite Elemente Methode \sphinxstylestrong{(FEM)}

\item {} 
\sphinxAtStartPar
Finite Volumen Methode \sphinxstylestrong{(FVM)}

\item {} 
\sphinxAtStartPar
Finite Differenzen Methode \sphinxstylestrong{(FDM)}

\item {} 
\sphinxAtStartPar
Randelementmethode \sphinxstylestrong{(BEM)}

\item {} 
\sphinxAtStartPar
Discontinous Galerkin Methods \sphinxstylestrong{(DG)}

\end{itemize}

\item {} 
\sphinxAtStartPar
Netzunabhängige Methoden:
\begin{itemize}
\item {} 
\sphinxAtStartPar
Optimal Transport Method \sphinxstylestrong{(OTM)}

\item {} 
\sphinxAtStartPar
Element\sphinxhyphen{}Free Galerkin \sphinxstylestrong{(EFG)}

\item {} 
\sphinxAtStartPar
Moving Least Squares \sphinxstylestrong{(MLS)}

\end{itemize}

\item {} 
\sphinxAtStartPar
Kombination:
\begin{itemize}
\item {} 
\sphinxAtStartPar
Material Point Method \sphinxstylestrong{(MPM)}

\end{itemize}

\item {} 
\sphinxAtStartPar
…

\end{itemize}

\sphinxAtStartPar
Alle diese Methoden haben eine Gemeinsamkeit: Sie führen auf ein Gleichungssystem der Form:
\begin{equation}\label{equation:chapters/chapter1/Einführung_Loesungsverfahren:LGS}
\begin{split}\boldsymbol{Ax}=\boldsymbol{b} \end{split}
\end{equation}
\sphinxAtStartPar
mit der Systemmatrix \(\boldsymbol{A}\). Im folgenden soll nur kurz auf die 4 am weitest verbreitetsten Methoden eingehen.


\subsection{Finite Differenzen Methode (FDM)}
\label{\detokenize{chapters/chapter1/Einf_xfchrung_Loesungsverfahren:finite-differenzen-methode-fdm}}
\sphinxAtStartPar
Die Finite Differenzen Methode (FDM) ist ein numerisches Verfahren zur Lösung partieller Differentialgleichungen (PDEs), das auf der Approximation der Ableitungen durch Differenzenquotienten basiert. Hier sind die grundlegenden Schritte und Konzepte der FDM:
\begin{enumerate}
\sphinxsetlistlabels{\arabic}{enumi}{enumii}{}{.}%
\item {} 
\sphinxAtStartPar
\sphinxstylestrong{Diskretisierung des Raums}: Der kontinuierliche Raum wird in ein Gitter unterteilt. Die Punkte auf diesem Gitter werden als Gitterpunkte bezeichnet.

\item {} 
\sphinxAtStartPar
\sphinxstylestrong{Approximation der Ableitungen}: Die Ableitungen in der PDE werden durch finite Differenzen approximiert. Zum Beispiel kann die erste Ableitung einer Funktion \(u\) an einem Punkt \(i\) durch den zentralen Differenzenquotienten approximiert werden:
\textbackslash{}begin\{equation\}
\textbackslash{}frac\{du\}\{dx\} \textbackslash{}approx \textbackslash{}frac\{u\_\{i+1\} \sphinxhyphen{} u\_\{i\sphinxhyphen{}1\}\}\{2\textbackslash{}Delta x\}
\textbackslash{}end\{equation\}
oder durch den vorwärts gerichteten Differenzenquotienten:
\textbackslash{}begin\{equation\}
\textbackslash{}frac\{du\}\{dx\} \textbackslash{}approx \textbackslash{}frac\{u\_\{i+1\} \sphinxhyphen{} u\_i\}\{\textbackslash{}Delta x\}
\textbackslash{}end\{equation\}

\item {} 
\sphinxAtStartPar
\sphinxstylestrong{Aufstellen eines Gleichungssystems}: Durch das Einsetzen der approximierten Ableitungen in die ursprüngliche PDE wird ein System von algebraischen Gleichungen aufgestellt, das die Werte der Funktion an den Gitterpunkten beschreibt.

\item {} 
\sphinxAtStartPar
\sphinxstylestrong{Lösen des Gleichungssystems}: Das resultierende Gleichungssystem kann mit verschiedenen numerischen Verfahren gelöst werden.

\end{enumerate}

\sphinxAtStartPar
Die Finite Differenzen Methode ist einfach zu implementieren und eignet sich gut für Probleme mit regelmäßigen Gitterstrukturen. Sie hat jedoch Einschränkungen bei der Behandlung komplexer Geometrien und kann in der Nähe von Randbedingungen oder Singularitäten weniger genau sein.


\subsection{Finite Volumen Methode (FVM)}
\label{\detokenize{chapters/chapter1/Einf_xfchrung_Loesungsverfahren:finite-volumen-methode-fvm}}
\sphinxAtStartPar
Die Finite Volumen Methode  ist ein numerisches Verfahren zur Lösung partieller Differentialgleichungen (PDEs), das insbesondere in der Strömungsmechanik und bei der Simulation von Transportphänomenen weit verbreitet ist. Hier sind die grundlegenden Schritte und Konzepte der FVM:
\begin{enumerate}
\sphinxsetlistlabels{\arabic}{enumi}{enumii}{}{.}%
\item {} 
\sphinxAtStartPar
\sphinxstylestrong{Diskretisierung des Raums}: Der kontinuierliche Raum wird in eine endliche Anzahl von Kontrollvolumen unterteilt. Jedes Kontrollvolumen ist ein kleines Volumen, das um einen Gitterpunkt zentriert ist. Diese Volumina können regelmäßig oder unregelmäßig sein, je nach der Geometrie des Problems.

\item {} 
\sphinxAtStartPar
\sphinxstylestrong{Integralform der PDE und Flussdiskretisierung}: Die PDE wird in ihrer Integralform formuliert und über jedes Kontrollvolumen integriert. Durch Anwendung des Gauss’schen Integralsatzes wird die Integralform in eine Form umgewandelt, die die Flüsse an den Grenzen der Kontrollvolumen berücksichtigt. Die Flüsse an den Grenzen werden approximiert, häufig durch Mittelwerte der Variablen an den benachbarten Kontrollvolumen oder durch spezielle Interpolationsmethoden.

\item {} 
\sphinxAtStartPar
\sphinxstylestrong{Aufstellen eines Gleichungssystems}: Durch die Diskretisierung der Flüsse und das Einsetzen in die Integralform wird ein System von algebraischen Gleichungen aufgestellt, das die Werte der gesuchten Variablen an den Gitterpunkten beschreibt.

\item {} 
\sphinxAtStartPar
\sphinxstylestrong{Lösen des Gleichungssystems}: Das resultierende Gleichungssystem kann mit verschiedenen numerischen Verfahren gelöst werden.

\end{enumerate}

\sphinxAtStartPar
Die Finite Volumen Methode ist besonders vorteilhaft, da sie die Erhaltungsgesetze direkt in die Berechnungen integriert und somit eine physikalisch konsistente Lösung gewährleistet. Sie eignet sich gut für Probleme mit komplexen Geometrien und ist robust gegenüber nichtlinearen Effekten. FVM wird häufig in der Strömungsmechanik, Wärmeübertragung und anderen Bereichen eingesetzt, in denen der Transport von Stoffen oder Energie eine Rolle spielt.


\subsection{Randelemente Methode (BEM)}
\label{\detokenize{chapters/chapter1/Einf_xfchrung_Loesungsverfahren:randelemente-methode-bem}}
\sphinxAtStartPar
Die Randelemente Methode (BEM) ist ein numerisches Verfahren zur Lösung partieller Differentialgleichungen (PDEs), das sich auf die Randbedingungen des Problems konzentriert. Hier sind die grundlegenden Schritte und Konzepte der BEM:
\begin{enumerate}
\sphinxsetlistlabels{\arabic}{enumi}{enumii}{}{.}%
\item {} 
\sphinxAtStartPar
\sphinxstylestrong{Reduktion des Problems auf den Rand}: Anstatt das gesamte Volumen des Problems zu diskretisieren, wird nur der Rand des betrachteten Gebiets in Betracht gezogen. Dies reduziert die Dimension des Problems, da nur die Randflächen (2D) oder Randlinien (1D) diskretisiert werden müssen.

\item {} 
\sphinxAtStartPar
\sphinxstylestrong{Formulierung der Integralgleichungen}: Die PDE wird in eine Integralform umgewandelt, die die Randbedingungen berücksichtigt. Diese Integralgleichungen beschreiben die Beziehung zwischen den Werten der gesuchten Funktion und ihren Ableitungen auf dem Rand des Gebiets.

\item {} 
\sphinxAtStartPar
\sphinxstylestrong{Diskretisierung des Randes}: Der Rand wird in eine endliche Anzahl von Elementen unterteilt, und die gesuchte Funktion wird durch Basisfunktionen approximiert, die auf diesen Elementen definiert sind. Dies führt zu einem System von algebraischen Gleichungen, das die Werte der gesuchten Funktion an den Randpunkten beschreibt.

\item {} 
\sphinxAtStartPar
\sphinxstylestrong{Lösen des Gleichungssystems}: Das resultierende Gleichungssystem kann mit verschiedenen numerischen Verfahren gelöst werden.

\end{enumerate}

\sphinxAtStartPar
Die Randelemente Methode ist besonders vorteilhaft, da sie die Dimension des Problems reduziert und somit die Anzahl der benötigten Berechnungen verringert. Sie ist besonders effektiv für Probleme mit unendlichen oder halbunendlichen Domänen, wie z.B. in der Elastizitätstheorie oder der Akustik. BEM ist jedoch oft auf Probleme mit linearen PDEs und gut definierten Randbedingungen beschränkt. Das heißt, Materielle Nichtlinearitäten können nicht berücksichtigt werden (keine Plastizität!).


\subsection{Finite Elemente Methode (FEM)}
\label{\detokenize{chapters/chapter1/Einf_xfchrung_Loesungsverfahren:finite-elemente-methode-fem}}
\sphinxAtStartPar
Die Finite Elemente Methode (FEM) ist ein weit verbreitetes numerisches Verfahren zur Lösung partieller Differentialgleichungen (PDEs), das insbesondere in der Ingenieurwissenschaft und der Physik Anwendung findet. Hier sind die grundlegenden Schritte und Konzepte der FEM:
\begin{enumerate}
\sphinxsetlistlabels{\arabic}{enumi}{enumii}{}{.}%
\item {} 
\sphinxAtStartPar
\sphinxstylestrong{Diskretisierung des Gebiets}: Der kontinuierliche Raum wird in eine endliche Anzahl von kleinen, nicht überlappenden Elementen unterteilt, die zusammen ein Netz (Mesh) bilden. Diese Elemente können verschiedene Formen haben, wie Dreiecke, Vierecke (in 2D) oder Tetraeder, Würfel (in 3D).

\item {} 
\sphinxAtStartPar
\sphinxstylestrong{Aufstellen der Elementgleichungen}: Für jedes Element wird eine lokale Formulierung der PDE aufgestellt, die die physikalischen Gesetze und Randbedingungen berücksichtigt. Dies geschieht häufig durch die Anwendung des Prinzips der virtuellen Verrückungen oder der Galerkin\sphinxhyphen{}Methode, um die Differentialgleichung in eine schwache Form zu überführen.

\item {} 
\sphinxAtStartPar
\sphinxstylestrong{Zusammenfügen der Elementgleichungen}: Die lokalen Gleichungen der einzelnen Elemente werden zu einem globalen Gleichungssystem zusammengefügt. Dies geschieht unter Berücksichtigung der Koppelung zwischen benachbarten Elementen und der gemeinsamen Knotenpunkte.

\item {} 
\sphinxAtStartPar
\sphinxstylestrong{Lösen des Gleichungssystems}: Das resultierende globale Gleichungssystem wird mit numerischen Verfahren gelöst.

\end{enumerate}

\sphinxAtStartPar
Die Finite Elemente Methode ist besonders vorteilhaft, da sie komplexe Geometrien und Materialverhalten effizient behandeln kann. Sie ermöglicht die Analyse von statischen und dynamischen Problemen in verschiedenen Bereichen, wie Strukturmechanik, Wärmeübertragung und Fluiddynamik. FEM ist flexibel und anpassungsfähig, erfordert jedoch eine sorgfältige Netzgenerierung und kann bei sehr feinen Netzen rechenintensiv sein.


\subsection{Zusammenfassung}
\label{\detokenize{chapters/chapter1/Einf_xfchrung_Loesungsverfahren:zusammenfassung}}
\begin{figure}[htbp]
\centering
\capstart

\noindent\sphinxincludegraphics[height=600\sphinxpxdimen]{{Diskretisierungsverfahren_knothe}.png}
\caption{Vergleich verschiedener Diskretisierungsverfahen nach {[}\hyperlink{cite.quellen:id4}{Knothe and Wessels, 1991}{]}}\label{\detokenize{chapters/chapter1/Einf_xfchrung_Loesungsverfahren:diskretisierungsvergleich}}\end{figure}

\sphinxAtStartPar
Jede der oben genannten Methoden hat ihre Daseinsberechtigung für gewählte Problemstellungen. Für die Strukturmechanik hat sich die Finite Elemente Methode durchgesetzt, da sie am flexibelsten und effizientesten eingesetzt werden kann. Dies wird in Abbildung \hyperref[\detokenize{chapters/chapter1/Einf_xfchrung_Loesungsverfahren:diskretisierungsvergleich}]{Abb.\@ \ref{\detokenize{chapters/chapter1/Einf_xfchrung_Loesungsverfahren:diskretisierungsvergleich}}} deutlich. Die resultierende Systemmatrix \(A\) ist für viele Problemstellungen symmetrische und weißt eine ausgesprochene Bandstruktur auf, was für die Lösung des Gleichungssystems erhebliche Vorteile bringt.


\begin{savenotes}\sphinxattablestart
\sphinxthistablewithglobalstyle
\centering
\begin{tabulary}{\linewidth}[t]{TTTTT}
\sphinxtoprule
\sphinxstyletheadfamily 
\sphinxAtStartPar
Methode
&\sphinxstyletheadfamily 
\sphinxAtStartPar
Dimension
&\sphinxstyletheadfamily 
\sphinxAtStartPar
Ansatz
&\sphinxstyletheadfamily 
\sphinxAtStartPar
Vorteile
&\sphinxstyletheadfamily 
\sphinxAtStartPar
Nachteile
\\
\sphinxmidrule
\sphinxtableatstartofbodyhook
\sphinxAtStartPar
\sphinxstylestrong{FEM}
&
\sphinxAtStartPar
Volumen
&
\sphinxAtStartPar
Diskretisierung des gesamten Gebiets in Elemente
&
\sphinxAtStartPar
Flexibel für komplexe Geometrien und Materialverhalten; gut für statische und dynamische Probleme
&
\sphinxAtStartPar
Erfordert sorgfältige Netzgenerierung; rechenintensiv bei feinen Netzen
\\
\sphinxhline
\sphinxAtStartPar
\sphinxstylestrong{BEM}
&
\sphinxAtStartPar
Rand
&
\sphinxAtStartPar
Reduktion des Problems auf den Rand des Gebiets
&
\sphinxAtStartPar
Reduziert die Dimension des Problems; effizient bei unendlichen oder halbunendlichen Domänen
&
\sphinxAtStartPar
Beschränkt auf Probleme mit gut definierten Randbedingungen; oft nur für lineare PDEs geeignet
\\
\sphinxhline
\sphinxAtStartPar
\sphinxstylestrong{FDM}
&
\sphinxAtStartPar
Volumen
&
\sphinxAtStartPar
Diskretisierung des Gebiets in Gitterpunkte
&
\sphinxAtStartPar
Einfach zu implementieren; gut für regelmäßige Geometrien
&
\sphinxAtStartPar
Schwierigkeiten bei komplexen Geometrien; weniger genau in der Nähe von Singularitäten
\\
\sphinxhline
\sphinxAtStartPar
\sphinxstylestrong{FVM}
&
\sphinxAtStartPar
Volumen
&
\sphinxAtStartPar
Diskretisierung des Gebiets in Kontrollvolumen
&
\sphinxAtStartPar
Berücksichtigt Erhaltungsgesetze direkt; robust bei nichtlinearen Effekten
&
\sphinxAtStartPar
Kann komplexer in der Implementierung sein; erfordert sorgfältige Flussapproximation
\\
\sphinxbottomrule
\end{tabulary}
\sphinxtableafterendhook\par
\sphinxattableend\end{savenotes}

\sphinxAtStartPar
Jede Methode hat ihre eigenen Stärken und Schwächen, und die Wahl der Methode hängt oft von der spezifischen Anwendung ab.

\sphinxstepscope





\begin{sphinxadmonition}{note}{Lernziele}
\begin{itemize}
\item {} 
\sphinxAtStartPar
Wie werden mechanische Problemstellungen formuliert?

\item {} 
\sphinxAtStartPar
Was sind die Unbekannten der Bilanzgleichungen?

\item {} 
\sphinxAtStartPar
Welche Zutaten braucht man um ein lösbares Problem zu erhalten?

\item {} 
\sphinxAtStartPar
Wie lauten die wesentlichen Bilanzgleichungen der Kontinuumsmechanik?

\item {} 
\sphinxAtStartPar
In welche Kategorien werden Randbedingungen untergliedert?

\item {} 
\sphinxAtStartPar
Wozu brauchen wir Materialmodelle?

\item {} 
\sphinxAtStartPar
Welches sind die grundlegenden Materialmodelle der Wärmeleitung, Fluidtransport und der Strukturmechanik?

\item {} 
\sphinxAtStartPar
Mit welchen Vereinfachungen kann die Dimensionalität der Problemstellung reduziert werden?

\item {} 
\sphinxAtStartPar
Was sind die Vorraussetzungen hierfür?

\end{itemize}
\end{sphinxadmonition}


\section{Mathematische Modellbildung in der Kontinuumstheorie}
\label{\detokenize{chapters/chapter1/elasticity:mathematische-modellbildung-in-der-kontinuumstheorie}}\label{\detokenize{chapters/chapter1/elasticity::doc}}
\sphinxAtStartPar
In diesem Kapitel werden die grundlegenden Gleichungen der Kontinuumsmechanik vorgestellt. Dies dient als Ausgangspunkt für die numerische Lösung mechanischer Problemstellungen. Während der Vorstellung der Ausgangsgleichungen wird zudem die verwendete Notation eingeführt.


\subsection{Notation}
\label{\detokenize{chapters/chapter1/elasticity:notation}}
\sphinxAtStartPar
Im Rahmen dieses Moduls wird die folgende symbolische Schreibweise verwendet:


\begin{savenotes}\sphinxattablestart
\sphinxthistablewithglobalstyle
\centering
\begin{tabulary}{\linewidth}[t]{TTT}
\sphinxtoprule
\sphinxstyletheadfamily 
\sphinxAtStartPar
Tensorstufe
&\sphinxstyletheadfamily 
\sphinxAtStartPar
Symbol
&\sphinxstyletheadfamily 
\sphinxAtStartPar
Tensorschreibweise
\\
\sphinxmidrule
\sphinxtableatstartofbodyhook
\sphinxAtStartPar
Tensor der Stufe 0 (Skalar)
&
\sphinxAtStartPar
\(a\), \(b\), \(\varphi\)
&
\sphinxAtStartPar
\(a\), \(b\), \(\varphi\)
\\
\sphinxhline
\sphinxAtStartPar
Tensor der Stufe 1 (Vektor)
&
\sphinxAtStartPar
\(\bm{a}\), \(\bm{b}\)
&
\sphinxAtStartPar
\(a_i\bm{e}_i\), \(b_i\bm{e}_i\)
\\
\sphinxhline
\sphinxAtStartPar
Tensor der stufe 2 (Dyade)
&
\sphinxAtStartPar
\(\bm{A}\), \(\bm{\sigma}\)
&
\sphinxAtStartPar
\(A_{ij}\bm{e}_i\bm{e}_j\), \(B_{ij}\bm{e}_i\bm{e}_j\)
\\
\sphinxhline
\sphinxAtStartPar
Tensor der stufe 4
&
\sphinxAtStartPar
\(\mathbb{C}\), \(\mathbb{A}\)
&
\sphinxAtStartPar
\(\mathbb{C}\bm{e}_i\bm{e}_j\bm{e}_k\bm{e}_l\), \(\mathbb{A}\bm{e}_i\bm{e}_j\bm{e}_k\bm{e}_l\)
\\
\sphinxbottomrule
\end{tabulary}
\sphinxtableafterendhook\par
\sphinxattableend\end{savenotes}

\sphinxAtStartPar
Soweit möglich erhalten Tensoren der Stufe 2 Großbuchstaben als Symbol, während Vektoren mit Kleinbuchstaben dargestellt werden. Handschriftlich wird ein Tensor durch einen Unterstrich repräsentiert \(\underline{A}=\bm{A}\). Für Vektoren ist alternativ auch der Vektorpfeil gebräuchlich \(\underline{a} = \vec{a} = \bm{a}\).


\subsection{Bilanzgleichungen der Kontinuumsmechanik}
\label{\detokenize{chapters/chapter1/elasticity:bilanzgleichungen-der-kontinuumsmechanik}}
\sphinxAtStartPar
In diesem Abschnitt werden wir die grundlegenden Bilanzgleichungen der Kontinuumsmechanik einführen. Diese Gleichungen sind das Fundament für die Beschreibung des physikalischen Verhaltens und bilden den Ausgangspunkt für die Finite\sphinxhyphen{}Elemente\sphinxhyphen{}Methode. Die Bilanzgleichungen repräsentieren Erhaltungsprinzipien, die für jedes Kontinuum gelten, unabhängig von den spezifischen Materialeigenschaften. Die Bilanzgleichungen dienen als Bestimmungsgleichung für die Feldgrößen:
\begin{itemize}
\item {} 
\sphinxAtStartPar
Druck \(p(\bm{x},t)\)

\item {} 
\sphinxAtStartPar
Verschiebung \(\bm{u}(\bm{x},t)\)

\item {} 
\sphinxAtStartPar
Temperatur \(\theta(\bm{x},t)\).

\end{itemize}

\sphinxAtStartPar
Als Feldgröße bezeichnet man physikalische Größen, denen zu jedem Zeitpunkt \(t\) und zu jedem materiellen Punkt \(\bm{x}\) ein Wert zugeordnet wird.

\sphinxAtStartPar
Die vier wesentlichen Bilanzgleichungen sind:
\begin{itemize}
\item {} 
\sphinxAtStartPar
Massenerhaltung

\item {} 
\sphinxAtStartPar
Impulserhaltung

\item {} 
\sphinxAtStartPar
Drehimpulserhaltung (hieraus resultiert die Symmetrie des Spannungstensors \(\bm{\sigma}\))

\item {} 
\sphinxAtStartPar
Energieerhaltung (1. Hauptsatz der Thermodynamik)

\end{itemize}


\subsubsection{Massenerhaltung}
\label{\detokenize{chapters/chapter1/elasticity:massenerhaltung}}
\begin{figure}[htbp]
\centering
\capstart

\noindent\sphinxincludegraphics[height=200\sphinxpxdimen]{{Massenbilanz}.jpg}
\caption{Ein\sphinxhyphen{} und Ausfluss eines infinitesimalen Volumenelementes \sphinxstylestrong{Austauschen}}\label{\detokenize{chapters/chapter1/elasticity:massenbilanz}}\end{figure}

\sphinxAtStartPar
Die Massenerhaltung besagt, dass die Masse eines geschlossenen Systems konstant bleibt, oder wie in der Abbildung, dass die zeitliche Änderung der Masse eines Systems gleich der Differenz von Zustrom und Abfluss ist. In Abwesenheit von Massenquellen oder \sphinxhyphen{}senken kann dies mathematisch ausgedrückt werden als:
\begin{equation}\label{equation:chapters/chapter1/elasticity:Massenbilanz}
\begin{split}\dot{\varrho} +  \div{\varrho {\bm{v}}} = 0\end{split}
\end{equation}
\sphinxAtStartPar
Hier ist \(\varrho\) die Dichte und \(\bm{v}\) die Geschwindigkeit des Materials. Die zeitliche Ableitung einer Größe \(\square\) wird durch einen Punkt über der Größe dargestellt \(\td{\square}\).

\begin{sphinxadmonition}{note}{Bemerkung:}
\sphinxAtStartPar
Die Massenbilanz ist für gewöhnliche Festkörper stets erfüllt und muss nicht separat gelöst werden. Anders sieht dies im Bereich der Biomechanik oder Geomechanik aus. Hier gibt es z.B. Wachstumsprozesse (Knochen) oder Sickerströmungen. Mit der Massenbilanz kann man die Konzentration \(c(\bm{x},t)\) eines Bestandteils oder den Druck \(p(\bm{x},t)\) einer Phase bestimmen.
\end{sphinxadmonition}

\sphinxAtStartPar
Formuliert man die Massenbilanz für Fluide, dann ist die Massendichte eine Funktion des Druckes \(p\) und man kann die Zeitableitung in \eqref{equation:chapters/chapter1/elasticity:Massenbilanz} über die Kettenregel auswerten:
\begin{equation}\label{equation:chapters/chapter1/elasticity:Massenbilanz2}
\begin{split} \underbrace{\Pd{\varrho}{p}}_{=:\frac{\varrho}{\kappa}} \td{p} + \div{\varrho {\bm{v}}} = 0 \; .\end{split}
\end{equation}
\sphinxAtStartPar
Die primäre Größe nach der die partielle Differentialgleichung gelöst wird ist damit der Druck \(p\). Als sekundäre Größe bezeichnet man Feldgrößen, welche von den primären Größen abgeleitet wurden. Im Fall der Massenbilanz des Fluides ist dies die Geschwindigkeit der Materialpartikel \(\bm{v}\). Die Gleichungen \eqref{equation:chapters/chapter1/elasticity:Massenbilanz} und \eqref{equation:chapters/chapter1/elasticity:Massenbilanz2} beschreiben wie sich die bilanzierte Größe (\(p\)) in der Zeit verändert, eine Information über den absoluten Wert der Größe erhält man erst mit der Einführung von Anfangs\sphinxhyphen{} und Randwerten:
\label{equation:chapters/chapter1/elasticity:b8b31b5c-f8df-431e-be1a-da9acb15a14d}\begin{align}
p(\bm{x},t=t_0) & = \tilde{p}_0 \qquad &\forall \bm{x} \in \mathcal{B}& \\
p(\bm{x},t) & = \tilde{p} \qquad &\forall \bm{x} \in \partial \mathcal{B}_p& \\
\varrho \bm{v}\T \bm{n} & = \tilde{q}_m \qquad &\forall \bm{x} \in \partial\mathcal{B}_m&
\end{align}
\sphinxAtStartPar
Hierbei wird der zweite Term auch als Dirichlet oder wesentliche Randbedingung und der dritte Term als Neumann oder natürliche Randbedingung bezeichnet.

\sphinxAtStartPar
Diese Art der mathematischen Problemstellung nennt man Anfangsrandwertproblem (IBVP \sphinxhyphen{} Initial Boundary Value Problem).

\begin{sphinxadmonition}{note}{IVBP1 \sphinxhyphen{} Massenerhaltung}

\sphinxAtStartPar
Bestimme das Druckfeld \(p(\bm{x},t): \mathcal{B} \times [t_0,t] \rightarrow \mathcal{R}^1\), sodass für alle materiellen Punkte \(\bm{x} \, \in \, \mathcal{B}\) zu jedem Zeitpunkt \(t \, \in \, [t_0,t]\) die Bilanzgleichung:
\begin{equation*}
\begin{split}
\frac{\varrho}{\kappa} \td{p} + \div{\varrho {\bm{v}}} = 0 
\end{split}
\end{equation*}
\sphinxAtStartPar
unter Einhaltung der Anfangs\sphinxhyphen{} und Randbedingungen:
\begin{equation*}
\begin{split}
p(\bm{x},t=t_0) & = \tilde{p}_0 \qquad &\forall \bm{x} \in \mathcal{B}& \\
p(\bm{x},t) & = \tilde{p} \qquad &\forall \bm{x} \in \partial \mathcal{B}_p& \\
\varrho \bm{v}\T \bm{n} & = \tilde{q}_m \qquad &\forall \bm{x} \in \partial\mathcal{B}_m&
\end{split}
\end{equation*}
\sphinxAtStartPar
erfüllt ist.
\end{sphinxadmonition}


\subsubsection{Impulserhaltung}
\label{\detokenize{chapters/chapter1/elasticity:impulserhaltung}}
\begin{figure}[htbp]
\centering
\capstart

\noindent\sphinxincludegraphics[height=200\sphinxpxdimen]{{Impulsbilanz}.jpg}
\caption{Körper \(\mathcal{B}\) unter Einwirkung der externen Oberflächenkraft \(\tilde{\bm{t}}\) und der Volumenlast \(\bm{b}\)}\label{\detokenize{chapters/chapter1/elasticity:impulsbilanz-fig-1}}\end{figure}

\sphinxAtStartPar
Die Impulserhaltung, oft formuliert als Newtons zweites Gesetz, besagt, dass die Änderung des Impulses \(\rho \bm{v}\) eines Körpers gleich der Summe der auf ihn wirkenden Kräfte ist. Für ein Kontinuum wird dies durch die Cauchy\sphinxhyphen{}Bewegungsgleichungen ausgedrückt:
\label{equation:chapters/chapter1/elasticity:5615693a-9010-4bfe-a27d-1993426ec866}\begin{equation}
 \varrho \td{\bm{v}} =\div{\bm{\sigma}} + \rho \bm{b}
\end{equation}
\sphinxAtStartPar
\(\boldsymbol{\sigma}\) steht hierbei für den Spannungstensor und \(\mathbf{b}\) für die Volumenkraft pro Masseneinheit.

\begin{sphinxadmonition}{note}{Bemerkung:}
\sphinxAtStartPar
Die Impulsbilanz ist eine vektorielle partielle Differentialgleichung. Mit ihrer Hilfe kann man das Verschiebungsfeld \(\bm{u}(\bm{x},t)\) und das Spannungsfeld \(\bm{\sigma}(\bm{x},t)\) eines Festkörpers bestimmen.
\end{sphinxadmonition}

\sphinxAtStartPar
Die primäre Feldgröße dieser Anfangsrandwertproblems ist die Verschiebung \(\bm{u}(\bm{x},t)\) der Materialpartikel. Die Sekundäre Größe ist die Spannung \(\bm{\sigma}(\bm{x},t)\), welche wiederum von den Dehnungen \(\bm{\epsilon}\) abhängig ist. Als Randwerte können entweder Verschiebungen (wesentliche Randbedingung) vorgegeben werden oder Oberflächenspannungen (natürliche Randbedingungen).

\begin{sphinxadmonition}{note}{IVBP2 \sphinxhyphen{} Impulsbilanz}

\sphinxAtStartPar
Bestimme das Verschiebungsfeld \(\bm{u}(\bm{x},t): \mathcal{B} \times [t_0,t] \rightarrow \mathcal{R}^3\), sodass für alle materiellen Punkte \(\bm{x} \, \in \, \mathcal{B}\) zu jedem Zeitpunkt \(t \, \in \, [t_0,t]\) die Bilanzgleichung:
\begin{equation*}
\begin{split}
\varrho \td{\bm{v}} =\div{\bm{\sigma}} + \rho \bm{b} 
\end{split}
\end{equation*}
\sphinxAtStartPar
unter Einhaltung der Anfangs\sphinxhyphen{} und Randbedingungen:
\begin{equation*}
\begin{split}
\bm{u}(\bm{x},t=t_0) & = \tilde{\bm{u}}_0 \qquad &\forall \bm{x} \in \mathcal{B}& \\
\bm{u}(\bm{x},t) & = \tilde{\bm{u}} \qquad &\forall \bm{x} \in \partial \mathcal{B}_u& \\
\bm{\sigma}\T \bm{n} & = \tilde{\bm{t}} \qquad &\forall \bm{x} \in \partial\mathcal{B}_{\sigma}&
\end{split}
\end{equation*}
\sphinxAtStartPar
erfüllt ist.
\end{sphinxadmonition}


\subsubsection{Energieerhaltung}
\label{\detokenize{chapters/chapter1/elasticity:energieerhaltung}}
\sphinxAtStartPar
Die Energieerhaltung stellt sicher, dass die gesamte Energie in einem abgeschlossenen System erhalten bleibt. Für ein Kontinuum kann die Energiebilanz folgendermaßen ausgedrückt werden:
\begin{equation}\label{equation:chapters/chapter1/elasticity:1HS_01}
\begin{split}\rho \MtdFull{e} = \bm{\sigma}\T \td{\bm{\epsilon}} - \div{\bm{q}} + \varrho r\end{split}
\end{equation}
\sphinxAtStartPar
Hier ist \(e\) die spezifische innere Energie, \(\bm{q}\) der Wärmeflussvektor und \(r\) die Wärmequelle pro Masseneinheit.

\begin{sphinxadmonition}{note}{Bemerkung:}
\sphinxAtStartPar
Die skalare Energiebilanzgleichung dient zur Berechnung des Temperaturfeldes \(\theta(\bm{x},t)\).
\end{sphinxadmonition}

\sphinxAtStartPar
Vernachlässigt man die Kopplung zwischen Verschiebung und Temperatur erhält man aus Gleichung \eqref{equation:chapters/chapter1/elasticity:1HS_01} die bekannte instationäre Wärmeleitungsgleichung:
\label{equation:chapters/chapter1/elasticity:18dc7df9-590c-425b-8135-e53e2449ffd9}\begin{equation}
\varrho c_e \Pd{\theta}{t} + \div{\bm{q}} - \varrho r = 0
\end{equation}
\sphinxAtStartPar
Hier ist die primäre Variable die Temperatur \(\theta(\bm{x},t)\) und die sekundäre Variable der Wärmeflussvektor \(\bm{q}(\bm{x},t)\). Gemeinsam mit den Rand\sphinxhyphen{} und Anfangsbedingungen erhält man das Anfangsrandwertproblem:

\begin{sphinxadmonition}{note}{IVBP3 \sphinxhyphen{} Energiebilanz}

\sphinxAtStartPar
Bestimme das Temperaturfeld \(\theta(\bm{x},t): \mathcal{B} \times [t_0,t] \rightarrow \mathcal{R}^1\), sodass für alle materiellen Punkte \(\bm{x} \, \in \, \mathcal{B}\) zu jedem Zeitpunkt \(t \, \in \, [t_0,t]\) die Bilanzgleichung:
\begin{equation*}
\begin{split}
\varrho c_e \Pd{\theta}{t} + \div{\bm{q}} - \varrho r = 0
\end{split}
\end{equation*}
\sphinxAtStartPar
unter Einhaltung der Anfangs\sphinxhyphen{} und Randbedingungen:
\begin{equation*}
\begin{split}
\theta(\bm{x},t=t_0) & = \tilde{\theta}_0 \qquad &\forall \bm{x} \in \mathcal{B}& \\
\theta(\bm{x},t) & = \tilde{\theta} \qquad &\forall \bm{x} \in \partial \mathcal{B}_{\theta}& \\
\bm{q}\T \bm{n} & = \tilde{\bm{q}} \qquad &\forall \bm{x} \in \partial\mathcal{B}_q&
\end{split}
\end{equation*}
\sphinxAtStartPar
erfüllt ist.
\end{sphinxadmonition}


\subsubsection{Zusammenfassung}
\label{\detokenize{chapters/chapter1/elasticity:zusammenfassung}}
\sphinxAtStartPar
In den vorangestellten Abschnitten wurde die Bilanzgleichungen der Kontinuumsmechanik verwendet um die für die Ingenieurpraxis wichtigen Anfangsrandwertprobleme der Fluid\sphinxhyphen{}, Struktur\sphinxhyphen{} und Thermomechanik aufzustellen. Für diese Systeme von gekoppelten partiellen Differentialgleichungen gibt es nur in wenigen Sonderfällen analytische Lösungen. Einige Davon sind uns aus den Vorlesungen zur Elastostatik und Stömungslehr bekannt. Aus diesem Grund haben sich numerische Berechnungsverfahren durchgesetzt. Um zu überprüfen ob die Anfangsrandwertprobleme in der obigen Form lösbar sind werden in der nachfolgenden Tabelle die Bestimmungsgleichungen und die Unbekannten gegenübergestellt.


\begin{savenotes}\sphinxattablestart
\sphinxthistablewithglobalstyle
\centering
\begin{tabulary}{\linewidth}[t]{TTTTT}
\sphinxtoprule
\sphinxstyletheadfamily 
\sphinxAtStartPar
Bestimmungsgleichung
&\sphinxstyletheadfamily 
\sphinxAtStartPar
Anzahl an Gleichungen
&\sphinxstyletheadfamily 
\sphinxAtStartPar
Unbekannte
&\sphinxstyletheadfamily 
\sphinxAtStartPar
Anzahl an Unbekannten
&\sphinxstyletheadfamily 
\sphinxAtStartPar
Anzahl an zusätzlichen Bedingungen
\\
\sphinxmidrule
\sphinxtableatstartofbodyhook
\sphinxAtStartPar
Massenbilanz
&
\sphinxAtStartPar
1
&
\sphinxAtStartPar
\(p\), \(\bm{v}\)
&
\sphinxAtStartPar
4
&
\sphinxAtStartPar
3
\\
\sphinxhline
\sphinxAtStartPar
Impulsbilanz
&
\sphinxAtStartPar
3
&
\sphinxAtStartPar
\(\bm{u}\), \(\bm{\sigma}\), \(\bm{\epsilon}\)
&
\sphinxAtStartPar
15
&
\sphinxAtStartPar
12
\\
\sphinxhline
\sphinxAtStartPar
Energiebilanz
&
\sphinxAtStartPar
1
&
\sphinxAtStartPar
\(\theta\), \(\bm{q}\)
&
\sphinxAtStartPar
4
&
\sphinxAtStartPar
3
\\
\sphinxbottomrule
\end{tabulary}
\sphinxtableafterendhook\par
\sphinxattableend\end{savenotes}

\sphinxAtStartPar
Wie leicht zu erkennen ist genügt die Anzahl an Gleichungen für keines der obigen Problemstellungen zur Lösung. Es werden darum zusätzliche Bedingungen formuliert um die Anfangsrandwertprobleme zu lösen:
\begin{itemize}
\item {} 
\sphinxAtStartPar
Kinematische Bedingungen

\item {} 
\sphinxAtStartPar
Materialmodelle

\end{itemize}


\subsection{Kinematik}
\label{\detokenize{chapters/chapter1/elasticity:kinematik}}
\sphinxAtStartPar
Der Kinemtischen Bedingungen haben besonders im Bereich der Strukturmechanik und der Lösung der Impulsbilanz eine hervorgehobene Bedeutung. So ist zum Beispiel die Forderung vom ebenbleiben des Querschnitts in der Bernoulli Balkentheorie eine starke kinematische Restriktion, welche auf die bekannte Differentialgleichung der Durchbiegung \(w(x)\) führt.
Im Rahmen der linearen Kontinuumsmechanik wird die Dehnung \(\bm{\epsilon}\) als symmetrischer Anteil des Verschiebungsgradienten \(\grad{\bm{u}}\) verstanden:
\begin{equation}\label{equation:chapters/chapter1/elasticity:linearStrain}
\begin{split}\bm{\epsilon} = \frac{1}{2} \left(\grad{\bm{u}}\T + \grad{\bm{u}} \right) \; .\end{split}
\end{equation}
\sphinxAtStartPar
Dieses Dehnungsmaß wird oft auch als Ingenieurdehnung bezeichnet und sollte nur in einem Dehnungsbereich < 10\% angewendet werden. Zudem ist hervorzuheben, dass Starrkörpertranslationen keine Ingenieurdehnung hervorrufen, Starrkörperrotationen hingegen sehrwohl. Aus diesem Grunde ist darauf zu achten, dass beim Auftreten größerer Rotationen stehts eine nichtlineare Theorie (2.Ordnung oder allgemein nichtlinear) verwendet wird.

\sphinxAtStartPar
Aufgrund der Symmetrie des Dehnungstensors erhalten wir \sphinxstylestrong{6} zusätzliche Bedingungen zur Lösung der Anfangsrandwertproblems der Impulsbilanz.


\subsection{Materialmodell}
\label{\detokenize{chapters/chapter1/elasticity:materialmodell}}
\sphinxAtStartPar
Was zur Schließung der mathematischen Problemstellung jetzt noch fehlt ist eine Relation zwischen der primären Feldgröße und ihrer zugeordneten Flussgröße (sekundäre Größe). Die oben beschriebenen Bilanzgleichungen sind physikalische Prinzipien die unabhängig vom Werkstoff Gültigkeit besitzen. Die spezifischen Werkstoffeigenschaften werden über Materialmodelle, sogenannte konstitutive Modelle, in das Anfangsrandwertproblem eingebracht. Die konstitutive Materialtheorie ist eine Wissenschaft für sich und soll in dieser Vorlesung nicht weiter behandelt werden. In der Strukturmechanik können Materialien entsprechend ihrer Eigenschaften eingeteilt werden in:

\begin{figure}[htbp]
\centering
\capstart

\noindent\sphinxincludegraphics[height=500\sphinxpxdimen]{{Materialklassen}.png}
\caption{Klassifizierung von Materialmodellen anhand ihrer Systemantwort. Abbildung nach {[}\hyperlink{cite.quellen:id7}{Haupt, 2013}{]}}\label{\detokenize{chapters/chapter1/elasticity:materialklassen}}\end{figure}


\subsubsection{Wärmeleitung}
\label{\detokenize{chapters/chapter1/elasticity:warmeleitung}}
\sphinxAtStartPar
Bei der Berechnung der Wärmeleitungsgleichung wird oftmals \sphinxstylestrong{Fourier}sche Wärmeleitung zugrunde gelegt. Diese basiert auf der Annahme, dass der Wärmestrom proportional zum negativen Temperaturgradienten ist:
\begin{equation}\label{equation:chapters/chapter1/elasticity:fourier}
\begin{split}\bm{q} = - \bm{k}\T  \grad{\theta} \, .\end{split}
\end{equation}
\sphinxAtStartPar
Hierbei ist \(\bm{k}\) die Konduktivitätstensor, welche für isotrope Materialien wie folgt aussieht:
\begin{equation}\label{equation:chapters/chapter1/elasticity:konduktivitaetsmatrixIsotrop}
\begin{split} \bm{k} = k \begin{bmatrix} 1 & 0 & 0 \\ 0 & 1 & 0 \\ 0 & 0 & 1\end{bmatrix}  \, ,\end{split}
\end{equation}
\sphinxAtStartPar
mit der Wärmeleitfähigkeit \(k\) in \(\frac{\text{W}}{\text{m}\cdot\text{K}}\). Der Wärmestromvektor hat damit die Einheit \(\frac{\text{W}}{\text{m}^2}\).

\begin{sphinxadmonition}{note}{Zusätzliche Gleichungen}

\sphinxAtStartPar
Aus Gleichung \eqref{equation:chapters/chapter1/elasticity:fourier} erhalten wir \sphinxstylestrong{3} zusätzliche Gleichungen. Wir haben jetzt genausoviele Gleichungen wie Unbekannte.
\end{sphinxadmonition}

\begin{sphinxadmonition}{note}{Materialsymmetrie (Auswahl)}
\begin{itemize}
\item {} 
\sphinxAtStartPar
isotrop \sphinxhyphen{} gleiche Eigenschaften in alle Raumrichtungen

\item {} 
\sphinxAtStartPar
transversal isotrop \sphinxhyphen{} Unterschiedliche Eigenschaften entlang einer Faser (Symmetrieachse) und der zu dieser Faser senkrecht stehenden Ebene

\item {} 
\sphinxAtStartPar
orthotrop \sphinxhyphen{} drei paarweise senkrecht zueinander stehende Symmetrieachsen

\end{itemize}
\end{sphinxadmonition}


\subsubsection{Durchströmung}
\label{\detokenize{chapters/chapter1/elasticity:durchstromung}}
\sphinxAtStartPar
Bei der Berechnung der Fluidgeschwindigkeit in Sickerströmungen wird oftmals das \sphinxstylestrong{Darcy}sche hydraulische Strömungsgesetz angewendet. Dies ist analog zur Wärmeleitung definiert als proportional zum negativer Druckgradienten:
\begin{equation}\label{equation:chapters/chapter1/elasticity:darcy}
\begin{split}\bm{v} = - \bm{k}_{\varepsilon}\T  \grad{p} \, .\end{split}
\end{equation}
\sphinxAtStartPar
Hierbei ist \(\bm{k}_{\varepsilon}\) die hydraulische Konduktivität. Für isotrope Materialien gilt:
\begin{equation*}
\begin{split} \bm{k}_{\varepsilon} = \frac{k_{\varepsilon}}{\mu} \begin{bmatrix} 1 & 0 & 0 \\ 0 & 1 & 0 \\ 0 & 0 & 1\end{bmatrix}  \, ,\end{split}
\end{equation*}
\sphinxAtStartPar
mit der hydraulischen Permeabilität \(k_{\varepsilon}\) in \(\text{m}^2\) und der dynamischen Viskosität \(\mu\) in \(\text{Pa}\cdot \text{s}\).

\begin{sphinxadmonition}{note}{Zusätzliche Gleichungen}

\sphinxAtStartPar
Aus Gleichung \eqref{equation:chapters/chapter1/elasticity:darcy} erhalten wir \sphinxstylestrong{3} zusätzliche Gleichungen. Wir haben jetzt genausoviele Gleichungen wie Unbekannte.
\end{sphinxadmonition}


\subsubsection{Lineare Elastizität}
\label{\detokenize{chapters/chapter1/elasticity:lineare-elastizitat}}
\sphinxAtStartPar
In \hyperref[\detokenize{chapters/chapter1/elasticity:materialklassen}]{Abb.\@ \ref{\detokenize{chapters/chapter1/elasticity:materialklassen}}} sind verschiedene Klassen von Festkörpermaterialien dargestellt. In dieser Vorlesung behandeln wir lediglich eine Unterklasse der ratenunabhängigen Materialien ohne Hystere. Die Lineare Elastizität ist gekennzeichnet durch einen linearen Zusammenhang zwischen Spannung \(\bm{\sigma}\) und Dehnungen \(\bm{\epsilon}\). In tensorieller Notation stellt sich das verallgemeinerte Hook’sche Gesetz wie folgt dar:
\begin{equation}\label{equation:chapters/chapter1/elasticity:generalHook}
\begin{split}\sigma_{ij} = \mathbb{C}_{ijkl} \epsilon_{kl} \;\end{split}
\end{equation}
\sphinxAtStartPar
wobei von der Einsteinschen Summenkonvention gebrauch gemacht wurde. In der Strukturmechanik ist es jedoch unüblich mit Tensoren 4. Stufe explizit zu rechnen. Es werden Symmetrieeigenschaften der Tensoren \(\sigma_{ij}=\sigma_{ji}\), \(\epsilon_{ij}=\epsilon_{ji}\) und \(\mathbb{C}_{ijkl}=\mathbb{C}_{klij}\) ausgenutzt um Tensorprodukte wie z.B. in Gleichung \eqref{equation:chapters/chapter1/elasticity:generalHook} in Matrix\sphinxhyphen{}Vektor\sphinxhyphen{}Operationen zu überführen. Dies ist vor allem für die numerische Implementierung von großem Vorteil.
\begin{equation}\label{equation:chapters/chapter1/elasticity:generalHook2}
\begin{split}\begin{bmatrix} 
\sigma_{11} \\
\sigma_{22} \\
\sigma_{33} \\
\sigma_{12} \\
\sigma_{23} \\
\sigma_{13} 
\end{bmatrix} = \begin{bmatrix}
C_{1111} & C_{1122} & C_{1133} & C_{1112} & C_{1123} & C_{1113} \\
C_{2211} & C_{2222} & C_{2233} & C_{2212} & C_{2223} & C_{2213} \\
C_{3311} & C_{3322} & C_{3333} & C_{3312} & C_{3323} & C_{3313} \\
C_{1211} & C_{1222} & C_{1233} & C_{1212} & C_{1223} & C_{1213} \\
C_{2311} & C_{2322} & C_{2333} & C_{2312} & C_{2323} & C_{2313} \\
C_{1311} & C_{1322} & C_{1333} & C_{1312} & C_{1323} & C_{1313} \\ 
\end{bmatrix} \begin{bmatrix} \epsilon_{11} \\ \epsilon_{22} \\ \epsilon_{33} \\ 2\epsilon_{12} \\ 2\epsilon_{23} \\ 2\epsilon_{13} \end{bmatrix}\end{split}
\end{equation}
\sphinxAtStartPar
Für isotrope lineare Elastizität mit dem Materialkonstanten:
\begin{itemize}
\item {} 
\sphinxAtStartPar
\(E\) \sphinxhyphen{} Elastizitätsmodul in \(\text{MPa}\)

\item {} 
\sphinxAtStartPar
\(\nu\) \sphinxhyphen{} Querkontraktionszahl in {[}\sphinxhyphen{}{]}

\end{itemize}

\sphinxAtStartPar
erhält man dann:
\begin{equation}\label{equation:chapters/chapter1/elasticity:generalHook3}
\begin{split}\begin{bmatrix} 
\sigma_{11} \\
\sigma_{22} \\
\sigma_{33} \\
\sigma_{12} \\
\sigma_{23} \\
\sigma_{13} 
\end{bmatrix} = \frac{E}{(1+\nu)(1-2\nu)}\begin{bmatrix}
1-\nu & \nu & \nu & 0 & 0 & 0 \\
\nu & 1-\nu & \nu & 0 & 0 & 0 \\
\nu & \nu & 1-\nu & 0 & 0 & 0 \\
0 & 0 & 0 & \frac{1-2\nu}{2} & 0 & 0 \\
0 & 0 & 0 & 0 & \frac{1-2\nu}{2} & 0 \\
0 & 0 & 0 & 0 & 0 & \frac{1-2\nu}{2} \\ 
\end{bmatrix} \begin{bmatrix} \epsilon_{11} \\ \epsilon_{22} \\ \epsilon_{33} \\ 2\epsilon_{12} \\ 2\epsilon_{23} \\ 2\epsilon_{13} \end{bmatrix}\end{split}
\end{equation}
\sphinxAtStartPar
Für die inverse Beziehung \(\bm{\epsilon} = \bm{C}^{-1} \bm{\sigma}\) mit der Nachgiebiegkeitsmatrix \(\bm{C}^{-1}\) erhält man:
\begin{equation}\label{equation:chapters/chapter1/elasticity:generalHook4}
\begin{split}\begin{bmatrix} 
\epsilon_{11} \\
\epsilon_{22} \\
\epsilon_{33} \\
2\epsilon_{12} \\
2\epsilon_{23} \\
2\epsilon_{13} 
\end{bmatrix} =\begin{bmatrix}
\frac{1}{E} & \frac{-\nu}{E} & \frac{-\nu}{E} & 0 & 0 & 0 \\
\frac{-\nu}{E} & \frac{1}{E} & \frac{-\nu}{E} & 0 & 0 & 0 \\
\frac{-\nu}{E} & \frac{-\nu}{E} & \frac{1}{E} & 0 & 0 & 0 \\
0 & 0 & 0 & \frac{1}{G} & 0 & 0 \\
0 & 0 & 0 & 0 & \frac{1}{G} & 0 \\
0 & 0 & 0 & 0 & 0 & \frac{1}{G} \\ 
\end{bmatrix} \begin{bmatrix} \sigma_{11} \\ \sigma_{22} \\ \sigma_{33} \\ \sigma_{12} \\ \sigma_{23} \\ \sigma_{13} \end{bmatrix}\end{split}
\end{equation}
\sphinxAtStartPar
wobei \(G=\frac{E}{2(1+\nu)}\) den Schubmodul in \(\text{MPa}\) darstellt.

\begin{sphinxadmonition}{note}{Zusätzliche Gleichungen}

\sphinxAtStartPar
Aus Gleichung \eqref{equation:chapters/chapter1/elasticity:generalHook3} und \eqref{equation:chapters/chapter1/elasticity:generalHook4} erhalten wir \sphinxstylestrong{6} zusätzliche Gleichungen. Wir haben jetzt genausoviele Gleichungen wie Unbekannte.
\end{sphinxadmonition}

\begin{sphinxadmonition}{note}{Einsteinsche Summenkonvention}

\sphinxAtStartPar
Die Einsteinsche Summenkonvention ist eine Notationsvereinbarung, die in der Tensorrechnung verwendet wird, um Ausdrücke zu vereinfachen. Wenn ein Index in einem mathematischen Ausdruck zweimal vorkommt, wird implizit über diesen Index summiert:
\begin{equation*}
\begin{split}
c_i = \sum_{j=1}^{n} a_{ij} b_j \qquad \rightarrow \qquad  c_i=a_{ij} b_j
\end{split}
\end{equation*}\end{sphinxadmonition}

\begin{sphinxadmonition}{note}{Kelvin Notation vs. Voigt Notation}

\sphinxAtStartPar
Die Kelvin Notation und die Voigt Notation sind zwei verschiedene Methoden um tensorielle Größen in Matrix\sphinxhyphen{} und Vektorschreibweise zu überführen. In den meisten kommerziellen FEM Programmen ist die Voigt Notation vorzufinden, wobei neuere FE\sphinxhyphen{}Codes durchaus die Vorteile der Kelvin Notation ausnutzen {[}\hyperlink{cite.quellen:id8}{Nagel \sphinxstyleemphasis{et al.}, 2016}{]}. Allgemein kann gesagt werden, dass die Voigt Notation näher an der Ingenieurmechanik ist, da die Scheranteile \(\epsilon_{ij} \quad \forall i\neq j\) der Dehnung doppelt eingehen und wir somit die Gleitungen \(\gamma_{ij}\quad \forall i\neq j\) erhalten. Die Kelvin Notation hat den Vorteil, dass mit ihr weiterhin alle Tensorprodukte einfach gebildet werden können ohne das über den Vorfaktor der Komponenten nachgedacht werden muss. Im aktuellen Kurs wird die \sphinxstylestrong{Voigt}\sphinxhyphen{}Notation zugrunde gelegt.


\begin{savenotes}\sphinxattablestart
\sphinxthistablewithglobalstyle
\centering
\begin{tabulary}{\linewidth}[t]{TT}
\sphinxtoprule

\sphinxAtStartPar

&\sphinxstyletheadfamily 
\sphinxAtStartPar
Voigt Notation
\\
\sphinxmidrule
\sphinxtableatstartofbodyhook
\sphinxAtStartPar
\(\sigma_{ij}\)
&
\sphinxAtStartPar
\(\bm{\sigma}= \begin{bmatrix} \sigma_{11} & \sigma_{22} & \sigma_{33} & \sigma_{12} & \sigma_{23} & \sigma_{13} \end{bmatrix}\T \)
\\
\sphinxhline
\sphinxAtStartPar
\(\epsilon_{ij}\)
&
\sphinxAtStartPar
\(\bm{\epsilon}= \begin{bmatrix} \epsilon_{11} & \epsilon_{22} & \epsilon_{33} & 2\epsilon_{12} & 2\epsilon_{23} & 2\epsilon_{13} \end{bmatrix}\T \)
\\
\sphinxhline
\sphinxAtStartPar
\(\sigma_{ij}=\mathbb{C}_{ijkl}\epsilon_{kl}\)
&
\sphinxAtStartPar
\(\bm{\sigma} = \bm{C} \bm{\epsilon} \)
\\
\sphinxhline
\sphinxAtStartPar
\(\mathcal{E}=\sigma_{ij}\epsilon_{ij}\)
&
\sphinxAtStartPar
\( \mathcal{E} = \bm{\sigma}\T\bm{\epsilon}\)
\\
\sphinxbottomrule
\end{tabulary}
\sphinxtableafterendhook\par
\sphinxattableend\end{savenotes}
\begin{equation*}
\begin{split}
\mathbb{C}_{ijkl} \quad \rightarrow \quad \bm{C}=\begin{bmatrix}
C_{1111} & C_{1122} & C_{1133} & C_{1112} & C_{1123} & C_{1113} \\
C_{2211} & C_{2222} & C_{2233} & C_{2212} & C_{2223} & C_{2213} \\
C_{3311} & C_{3322} & C_{3333} & C_{3312} & C_{3323} & C_{3313} \\
C_{1211} & C_{1222} & C_{1233} & C_{1212} & C_{1223} & C_{1213} \\
C_{2311} & C_{2322} & C_{2333} & C_{2312} & C_{2323} & C_{2313} \\
C_{1311} & C_{1322} & C_{1333} & C_{1312} & C_{1323} & C_{1313} \\ 
\end{bmatrix} 
\end{split}
\end{equation*}

\begin{savenotes}\sphinxattablestart
\sphinxthistablewithglobalstyle
\centering
\begin{tabulary}{\linewidth}[t]{TT}
\sphinxtoprule

\sphinxAtStartPar

&\sphinxstyletheadfamily 
\sphinxAtStartPar
Kelvin Notation
\\
\sphinxmidrule
\sphinxtableatstartofbodyhook
\sphinxAtStartPar
\(\sigma_{ij}\)
&
\sphinxAtStartPar
\(\bm{\sigma}= \begin{bmatrix} \sigma_{11} & \sigma_{22} & \sigma_{33} & \sqrt{2}\sigma_{12} & \sqrt{2}\sigma_{23} & \sqrt{2}\sigma_{13} \end{bmatrix}\T \)
\\
\sphinxhline
\sphinxAtStartPar
\(\epsilon_{ij}\)
&
\sphinxAtStartPar
\(\bm{\epsilon}= \begin{bmatrix} \epsilon_{11} & \epsilon_{22} & \epsilon_{33} & \sqrt{2}\epsilon_{12} & \sqrt{2}\epsilon_{23} & \sqrt{2}\epsilon_{13} \end{bmatrix}\T \)
\\
\sphinxhline
\sphinxAtStartPar
\(\sigma_{ij}=\mathbb{C}_{ijkl}\epsilon_{kl}\)
&
\sphinxAtStartPar
\(\bm{\sigma} = \bm{C} \bm{\epsilon}\)
\\
\sphinxhline
\sphinxAtStartPar
\(\mathcal{E}=\sigma_{ij}\epsilon_{ij}\)
&
\sphinxAtStartPar
\( \mathcal{E} = \bm{\sigma}\T\bm{\epsilon}\)
\\
\sphinxbottomrule
\end{tabulary}
\sphinxtableafterendhook\par
\sphinxattableend\end{savenotes}
\begin{equation*}
\begin{split}
\mathbb{C}_{ijkl} \quad \rightarrow \quad \bm{C}=\begin{bmatrix}
C_{1111} & C_{1122} & C_{1133} & \sqrt{2}C_{1112} & \sqrt{2}C_{1123} & \sqrt{2}C_{1113} \\
C_{2211} & C_{2222} & C_{2233} & \sqrt{2}C_{2212} & \sqrt{2}C_{2223} & \sqrt{2}C_{2213} \\
C_{3311} & C_{3322} & C_{3333} & \sqrt{2}C_{3312} & \sqrt{2}C_{3323} & \sqrt{2}C_{3313} \\
\sqrt{2}C_{1211} & \sqrt{2}C_{1222} & \sqrt{2}C_{1233} & 2C_{1212} & 2C_{1223} & 2C_{1213} \\
\sqrt{2}C_{2311} & \sqrt{2}C_{2322} & \sqrt{2}C_{2333} & 2C_{2312} & 2C_{2323} & 2C_{2313} \\
\sqrt{2}C_{1311} & \sqrt{2}C_{1322} & \sqrt{2}C_{1333} & 2C_{1312} & 2C_{1323} & 2C_{1313} \\ 
\end{bmatrix} 
\end{split}
\end{equation*}\end{sphinxadmonition}


\paragraph{Ebener Verzerrungszustand}
\label{\detokenize{chapters/chapter1/elasticity:ebener-verzerrungszustand}}
\begin{figure}[htbp]
\centering
\capstart

\noindent\sphinxincludegraphics[height=300\sphinxpxdimen]{{EVZ}.png}
\caption{Beispiel für das Auftreten eines ebenen Verzerrungszustandes {[}\hyperlink{cite.quellen:id10}{Chaves, 2013}{]}}\label{\detokenize{chapters/chapter1/elasticity:evz}}\end{figure}

\sphinxAtStartPar
Betrachten wir ein Strukturelement mit prismatischen Merkmalen, bei dem die Dimension in Richtung der prismatischen Achse deutlich größer ist als die anderen Dimensionen. Die aufgebrachten Lasten wirken senkrecht zur prismatischen Achse (\hyperref[\detokenize{chapters/chapter1/elasticity:evz}]{Abb.\@ \ref{\detokenize{chapters/chapter1/elasticity:evz}}}). Unter diesen Bedingungen sind die Dehnungskomponenten \(\e_{13},\, \e_{23}, \, \e_{33}\) gleich null. Dieser Zustand wird als ebener Verzerrungszustand bezeichnet. Beispiele hierfür sind Stützmauern, unter Druck stehende Zylinder, Dämme , Tunnel und Flachgründungen.

\sphinxAtStartPar
Es muss betont werden, dass die Variablen (Last, Querschnitt, Material) entlang der prismatischen Achse konstant sein müssen, um einen ebenen Verzerrungszustand zu betrachten. Andernfalls können erhebliche Fehler auftreten.

\begin{sphinxadmonition}{note}{Prismatische Merkmale}
\begin{itemize}
\item {} 
\sphinxAtStartPar
Konstanter Querschnitt entlang der Längsachse

\item {} 
\sphinxAtStartPar
Längliche Form

\item {} 
\sphinxAtStartPar
Geradlinige Achse

\end{itemize}

\sphinxAtStartPar
\sphinxstylestrong{Beispiele:} Balken, Säulen, Rohre
\end{sphinxadmonition}

\sphinxAtStartPar
Um die konstitutive Beziehung \(\sb(\eb)\) zu erhalten starten wir vom generalisierten Hook’schen Gesetz \eqref{equation:chapters/chapter1/elasticity:generalHook3} indem wir alle Spalten mit korrespondierenden 0\sphinxhyphen{}Dehnungen eliminieren:
\begin{equation}\label{equation:chapters/chapter1/elasticity:evz1}
\begin{split}\begin{bmatrix} 
\sigma_{11} \\
\sigma_{22} \\
\sigma_{33} \\
\sigma_{12} \\
\sigma_{23} \\
\sigma_{13} 
\end{bmatrix} = \frac{E}{(1+\nu)(1-2\nu)}\begin{bmatrix}
1-\nu & \nu & \cancel{\nu} & 0 & \cancel{0} & \cancel{0} \\
\nu & 1-\nu & \cancel{\nu} & 0 & \cancel{0} & \cancel{0} \\
\nu & \nu & \cancel{1-\nu} & 0 & \cancel{0} & \cancel{0} \\
0 & 0 & \cancel{0} & \frac{1-2\nu}{2} & \cancel{0} & \cancel{0} \\
0 & 0 & \cancel{0} & 0 & \cancel{\frac{1-2\nu}{2}} & \cancel{0} \\
0 & 0 & \cancel{0} & 0 & \cancel{0} & \cancel{\frac{1-2\nu}{2}} \\ 
\end{bmatrix} \begin{bmatrix} \epsilon_{11} \\ \epsilon_{22} \\ \cancel{\epsilon_{33}} \\ 2\epsilon_{12} \\ \cancel{2\epsilon_{23}} \\ \cancel{2\epsilon_{13}} \end{bmatrix}\end{split}
\end{equation}
\sphinxAtStartPar
Wir sehen sofort, dass die Komponenten \(\s_{23},\, \s_{13}\) verschwinden. Die Spannungskomponente entlang der prismatischen Achse \(\s_{33}\) hingegen berechnet sich zu:
\begin{equation*}
\begin{split}\s_{33} = \frac{E \nu}{(1+\nu)(1-2\nu)} \left(\e_{11} + \e_{22} \right)\end{split}
\end{equation*}
\sphinxAtStartPar
Arrangiert man die verbleibenden Einträge, so erhält man für \(\bm{C}\rs{EVZ}\):
\begin{equation}\label{equation:chapters/chapter1/elasticity:evz2}
\begin{split} \begin{bmatrix} 
\s_{11} \\
\s_{22} \\
\s_{12} 
\end{bmatrix}= \underbrace{\frac{E }{(1+\nu)(1-2\nu)} \begin{bmatrix}
1-\nu & \nu & 0 \\
\nu & 1-\nu & 0 \\
0 & 0 & \frac{1-2\nu}{2}
\end{bmatrix}}_{\bm{C}\rs{EVZ}}  \begin{bmatrix} 
\e_{11} \\
\e_{22} \\
\e_{12} 
\end{bmatrix}\end{split}
\end{equation}
\sphinxAtStartPar
Für die Nachgiebiegkeit \(\bm{C}\rs{EVZ}^{-1}\) erhält man die inverse Beziehung:
\begin{equation}\label{equation:chapters/chapter1/elasticity:evz3}
\begin{split} \begin{bmatrix} 
\e_{11} \\
\e_{22} \\
\e_{12} 
\end{bmatrix}= \underbrace{\frac{1+\nu }{E} \begin{bmatrix}
1-\nu & -\nu & 0 \\
-\nu & 1-\nu & 0 \\
0 & 0 & 2
\end{bmatrix}}_{\bm{C}\rs{EVZ}^{-1}}  \begin{bmatrix} 
\s_{11} \\
\s_{22} \\
\s_{12} 
\end{bmatrix}\end{split}
\end{equation}

\paragraph{Ebener Spannungszustand}
\label{\detokenize{chapters/chapter1/elasticity:ebener-spannungszustand}}
\begin{figure}[htbp]
\centering
\capstart

\noindent\sphinxincludegraphics[height=500\sphinxpxdimen]{{CarCrash}.jpg}
\caption{FEM Crash Simulation eines Pickup\sphinxhyphen{}Trucks gegen eine starre Wand \sphinxhref{https://enteknograte.com/wp-content/uploads/2022/06/truck-crash-Car-vehicle-Finite-Element-Simulation-Crash-Test-MSC-dytran-Crashworthiness-Ls-Dyna-Abaqus-PAM-CRASH.jpg}{Quelle}}\label{\detokenize{chapters/chapter1/elasticity:carcrash}}\end{figure}

\sphinxAtStartPar
Der ebene Spannungszustand tritt überall dort auf wo Flächenelemente vor allem in ihrer Ebene belastet werden und die Flächenabmessungen deutlich größer sind als die dazugehörige Dicke. Dies ist zum Beispiel bei Karosserieblechen wie in \hyperref[\detokenize{chapters/chapter1/elasticity:carcrash}]{Abb.\@ \ref{\detokenize{chapters/chapter1/elasticity:carcrash}}} der Fall.

\begin{figure}[htbp]
\centering
\capstart

\noindent\sphinxincludegraphics[height=300\sphinxpxdimen]{{Shellformulation}.png}
\caption{FEM Diskretisierung von Schalenelementen als degenerierte Volumenelemente (links) oder über eine Mittelflächendarstellung (rechts). {[}\hyperlink{cite.quellen:id9}{Wriggers, 2008}{]}}\label{\detokenize{chapters/chapter1/elasticity:shellformulation}}\end{figure}

\sphinxAtStartPar
In Abbildung \hyperref[\detokenize{chapters/chapter1/elasticity:shellformulation}]{Abb.\@ \ref{\detokenize{chapters/chapter1/elasticity:shellformulation}}} auf der rechten Seite ist eine Disketisierung für den ebenen Spannungszustand mit Mittelflächenelementen zu sehen.

\sphinxAtStartPar
Startpunkt für die konstitutive Beziehung ist die Gleichung \eqref{equation:chapters/chapter1/elasticity:generalHook4}. Zunächst streichen wir alle Spalten welche den 0\sphinxhyphen{}Spannungen zugeordnet werden können:
\begin{equation}\label{equation:chapters/chapter1/elasticity:generalHookESZ1}
\begin{split}\begin{bmatrix} 
\epsilon_{11} \\
\epsilon_{22} \\
\epsilon_{33} \\
2\epsilon_{12} \\
2\epsilon_{23} \\
2\epsilon_{13} 
\end{bmatrix} = \begin{bmatrix}
\frac{1}{E} & \frac{-\nu}{E} & \cancel{\frac{-\nu}{E}} & 0 & \cancel{0} & \cancel{0} \\
\frac{-\nu}{E} & \frac{1}{E} & \cancel{\frac{-\nu}{E}} & 0 & \cancel{0} & \cancel{0} \\
{\frac{-\nu}{E}} & {\frac{-\nu}{E}} & \cancel{\frac{1}{E}} & {0} & \cancel{0} & \cancel{0} \\
0 & 0 & \cancel{0} & \frac{1}{G} & 0 & 0 \\
{0} & {0} & \cancel{0} & {0} & \cancel{\frac{1}{G}} & \cancel{0} \\
{0} & {0} & \cancel{0} & {0} & \cancel{0} & \cancel{\frac{1}{G}} \\ 
\end{bmatrix} \begin{bmatrix} \sigma_{11} \\ \sigma_{22} \\ \cancel{\sigma_{33}} \\ \sigma_{12} \\ \cancel{\sigma_{23}} \\ \cancel{\sigma_{13}} \end{bmatrix}\end{split}
\end{equation}
\sphinxAtStartPar
Hier fällt auf, dass das resultierende System nicht mehr quadratisch ist, da die Dehnung in Dickenrichtung \(\epsilon_{33}\) ungleich null ist:
\begin{equation}\label{equation:chapters/chapter1/elasticity:generalHookESZ2}
\begin{split}\begin{bmatrix} 
\epsilon_{11} \\
\epsilon_{22} \\
\epsilon_{33} \\
2\epsilon_{12} \\
\end{bmatrix} = \begin{bmatrix}
\frac{1}{E} & \frac{-\nu}{E}  & 0  \\
\frac{-\nu}{E} & \frac{1}{E}  & 0  \\
{\frac{-\nu}{E}} & {\frac{-\nu}{E}} & {0} \\
0 & 0 & \frac{1}{G}  \\
\end{bmatrix} \begin{bmatrix} \sigma_{11} \\ \sigma_{22} \\ \sigma_{12} \end{bmatrix} \; .\end{split}
\end{equation}
\sphinxAtStartPar
Dieses System hat eine Gleichung zu viel. Zweckmäßig vernachlässigt man die Gleichung für die Dehnung in Dickenrichtung (diese Dehnung berechnet man in einer Nachlaufrechnung). Somit lautet die Nachgiebigkeitsmatrix \(\bm{C}^{-1}\rs{ESZ}\) und die Materielle Steifigkeitsmatrix \(\bm{C}\rs{ESZ}\) für den ebenen Spannungszustand:
\begin{equation}\label{equation:chapters/chapter1/elasticity:generalHookESZ3}
\begin{split}\bm{C}^{-1} = \frac{1}{E} \begin{bmatrix}
1 & -\nu & 0  \\
-\nu & 1  & 0  \\
0 & 0 & 2(1+\nu)  \\
\end{bmatrix}  \qquad 
\bm{C} = \frac{E}{(1-\nu^2)}\begin{bmatrix}
1 & \nu  & 0  \\
\nu & 1  & 0  \\
0 & 0 & \frac{1-\nu}{2}  \\
\end{bmatrix}\end{split}
\end{equation}

\paragraph{Rotationssymmetrie}
\label{\detokenize{chapters/chapter1/elasticity:rotationssymmetrie}}
\begin{figure}[htbp]
\centering
\capstart

\noindent\sphinxincludegraphics[height=300\sphinxpxdimen]{{Rotsym}.png}
\caption{Rotationssymmetrischer Körper im zylinderkoordinatensystem \(\{r,z,\theta\}\). Nach {[}\hyperlink{cite.quellen:id10}{Chaves, 2013}{]}}\label{\detokenize{chapters/chapter1/elasticity:rotsym}}\end{figure}

\sphinxAtStartPar
Die Dehnungen für rotationssymmetrische Problemstellungen werden berechnet zu:
\begin{equation}\label{equation:chapters/chapter1/elasticity:rotsymstrain}
\begin{split}\begin{bmatrix}
\e_{r} \\
\e_z \\
\e_{\theta} \\
\gamma_{rz}
\end{bmatrix} = \begin{bmatrix}
\Pd{u}{r} \\
\Pd{v}{z} \\
\frac{u}{r} \\
\Pd{u}{z} + \Pd{v}{r}
\end{bmatrix} \end{split}
\end{equation}
\sphinxAtStartPar
Das generalisierte Hook’sche Gesetz lautet:
\begin{equation}\label{equation:chapters/chapter1/elasticity:generalHookRotsym}
\begin{split}\begin{bmatrix} 
\sigma_{rr} \\
\sigma_{zz} \\
\sigma_{\theta} \\
\sigma_{rz} \\
\end{bmatrix} = \frac{E}{(1+\nu)(1-2\nu)}\begin{bmatrix}
1-\nu & \nu & \nu & 0  \\
\nu & 1-\nu & \nu & 0  \\
\nu & \nu & 1-\nu & 0  \\
0 & 0 & 0 & \frac{1-2\nu}{2} \\
\end{bmatrix} \begin{bmatrix} \epsilon_{r} \\ \epsilon_{\theta} \\ \epsilon_{z} \\ 2\epsilon_{rz}  \end{bmatrix}\end{split}
\end{equation}
\begin{sphinxadmonition}{note}{Berechnung Dehung \protect\(\e_{\theta}\protect\)}

\sphinxAtStartPar
Die Dehnung in Umfangsrichtung kann bestimmt werden als Längenänderung des Umfangs infolge einer radialen Verschiebung \(u\):

\sphinxAtStartPar
\(
\e_{\theta} = \frac{2 \pi (r+u) -2\pi r}{2\pi r} = \frac{u}{r}
\)
\end{sphinxadmonition}


\subsection{Zusammenfassung}
\label{\detokenize{chapters/chapter1/elasticity:id6}}
\sphinxAtStartPar
Um ein valides mechanisches Modell zu erlangen benötigt man:
\begin{enumerate}
\sphinxsetlistlabels{\arabic}{enumi}{enumii}{}{.}%
\item {} 
\sphinxAtStartPar
Die beschreibende partielle Differentialgleichung (physikalische Erhaltungssätze)

\item {} 
\sphinxAtStartPar
Die Kinematischen Beziehungen (Verbindung von primärer Größe zur abgeleiten Größe)

\item {} 
\sphinxAtStartPar
Materialmodell (Verbindung von sekundärer Größe zur kinematischen Größe)

\end{enumerate}

\sphinxAtStartPar
Eindeutig lösbar wird das Modell zudem erst, wenn die Randbedingungen entsprechend vorgegeben wurden.

\sphinxstepscope


\section{Grundgleichungen einfacher Strukturelemente}
\label{\detokenize{chapters/chapter1/strukturgleichungen:grundgleichungen-einfacher-strukturelemente}}\label{\detokenize{chapters/chapter1/strukturgleichungen::doc}}
\begin{figure}[htbp]
\centering
\capstart

\noindent\sphinxincludegraphics[width=600\sphinxpxdimen]{{Strukturelemente_Knothe}.png}
\caption{Übersicht über die Klassifizierung von Strukturvereinfachungen nach {[}\hyperlink{cite.quellen:id4}{Knothe and Wessels, 1991}{]}}\label{\detokenize{chapters/chapter1/strukturgleichungen:strukturkontinua}}\end{figure}

\sphinxAtStartPar
Die im vorherigen Abschnitt eingeführten Feldgleichungen lassen sich für viele Ingenieurwissenschaftlichen Problemstellungen deutlich vereinfachen. In \hyperref[\detokenize{chapters/chapter1/strukturgleichungen:strukturkontinua}]{Abb.\@ \ref{\detokenize{chapters/chapter1/strukturgleichungen:strukturkontinua}}} sind solche Vereinfachungen gezeigt.

\sphinxAtStartPar
Für die Impulsbilanz haben wir bereits in der Vorlesung zur „Technischen Mechanik 2“ zwei Vereinfachungen kennengelernt:
\begin{itemize}
\item {} 
\sphinxAtStartPar
den Dehnstab

\item {} 
\sphinxAtStartPar
den Balken

\end{itemize}

\sphinxAtStartPar
Nachfolgen werden lediglich die Differentialgleichungen dieser beiden Strukturelemente eingeführt. Für eine Herleitung der Gleichung sei auf die entsprechende Vorlesung verwiesen.


\subsection{Differentialgleichung des Stabes}
\label{\detokenize{chapters/chapter1/strukturgleichungen:differentialgleichung-des-stabes}}
\sphinxAtStartPar
Wird ein prismatisches Bauteil nur entlang seiner Achse \(x\) belastet, so resultieren hieraus nur Normalspannungen \(\sigma_x\). Solche Belastungsszenarien können über die Differentialgleichung des \sphinxstylestrong{Stabes} in guter Näherung bestimmt werden.

\begin{figure}[htbp]
\centering
\capstart

\noindent\sphinxincludegraphics[width=600\sphinxpxdimen]{{Stab_TMKompakt}.png}
\caption{Kräftegleichgewicht und Spannungen am geraden Stab nach {[}\hyperlink{cite.quellen:id12}{Wriggers \sphinxstyleemphasis{et al.}, 2006}{]}}\label{\detokenize{chapters/chapter1/strukturgleichungen:stab}}\end{figure}

\sphinxAtStartPar
Die Differentialgleichung des Stabes lautet:
\begin{equation}\label{equation:chapters/chapter1/strukturgleichungen:stabdgl}
\begin{split}\frac{\text{d } \left( E(x)A(x) \frac{\text{d } u}{\text{d } x} \right) }{\text{d } x} =  \frac{\text{d } \left(E(x)A(x) \alpha_T(x) \Delta \theta\right)}{\text{d } x} - n(x)\end{split}
\end{equation}
\sphinxAtStartPar
mit:
\begin{itemize}
\item {} 
\sphinxAtStartPar
\(E\) \sphinxhyphen{} Elastizitätsmodul in MPa

\item {} 
\sphinxAtStartPar
\(A\) \sphinxhyphen{} Querschnittsfläche in mm\(^2\)

\item {} 
\sphinxAtStartPar
\(u\) \sphinxhyphen{} Axiale Verschiebung in mm

\item {} 
\sphinxAtStartPar
\(\alpha\) \sphinxhyphen{} Wärmeausdehnungskoeffizient in mm/K

\item {} 
\sphinxAtStartPar
\(\Delta \theta\) \sphinxhyphen{} Temperaturänderung in K

\item {} 
\sphinxAtStartPar
\(n\) \sphinxhyphen{} axiale Streckenlast in N/mm

\end{itemize}

\sphinxAtStartPar
Für einen konstanten \(E\)\sphinxhyphen{}Modul und eine konstante Querschnittsfläche \(A\), sowie eine konstante Temperaturdifferenz \(\Delta \theta\) kann diese Differentialgleichung vereinfacht werden zu:
\begin{equation}\label{equation:chapters/chapter1/strukturgleichungen:stabdglsimple}
\begin{split} u^{\prime \prime} = \frac{\text{d}^2\,  u  }{\text{d } x^2} =  - \frac{n(x)}{EA}\end{split}
\end{equation}

\subsection{Differentialgleichung des Balkens}
\label{\detokenize{chapters/chapter1/strukturgleichungen:differentialgleichung-des-balkens}}
\sphinxAtStartPar
Balken sind Tragstrukturen, die eindimensional modelliert werden, da ihre Breite und Höhe im Verhältnis zu ihrer Länge sehr gering sind. Im Unterschied zu Stäben, bei denen die Belastung nur in Längsrichtung erfolgt, werden Balken zusätzlich senkrecht zur Längsachse belastet. Dies führt dazu, dass in Stäben ausschließlich Normalkräfte auftreten, während sich in Balken Momente und Querkräfte als Schnittgrößen ergeben.

\begin{figure}[htbp]
\centering
\capstart

\noindent\sphinxincludegraphics[width=600\sphinxpxdimen]{{Balken_TM2}.png}
\caption{Kräftegleichgewicht, Spannungen und Schnittgrößen am geraden Balken nach {[}\hyperlink{cite.quellen:id13}{Gross \sphinxstyleemphasis{et al.}, 2007}{]}}\label{\detokenize{chapters/chapter1/strukturgleichungen:balken}}\end{figure}

\sphinxAtStartPar
Die Differentialgleichung des Balken lautet:
\begin{equation}\label{equation:chapters/chapter1/strukturgleichungen:balkendgl}
\begin{split} \left(E I_y w^{\prime \prime}\right)^{\prime\prime} =  q(x)\end{split}
\end{equation}
\sphinxAtStartPar
Für eine konstante Biegesteifigkeit \(EI\) vereinfacht sich diese zu:
\begin{equation}\label{equation:chapters/chapter1/strukturgleichungen:balkendglsimple}
\begin{split}  w^{IV} =  \frac{q(x)}{E I_y}\end{split}
\end{equation}
\sphinxstepscope


\part{Theorie der Finiten Elemente Methode}

\sphinxstepscope


\chapter{Die Finite Elemente Methode}
\label{\detokenize{chapters/chapter2/overview:die-finite-elemente-methode}}\label{\detokenize{chapters/chapter2/overview::doc}}
\sphinxAtStartPar
In den nachfolgenden Abschnitten werden die Grundzüge der Finiten Elemente Methode eingeführt. Bei dieser Einführung handelt es sich um eine Ingenieurwissenschaftliche Darstellung. Auf mathematische Anforderungen und Restriktionen wird nur bei Bedarf eingegangen. Für eine mathematisch rigorose Herleitung sei an dieser Stelle auf das Buch von Braess {[}\hyperlink{cite.quellen:id14}{Braess, 2013}{]} verwiesen.

\sphinxstepscope




\section{Von der starken zur schwachen Form}
\label{\detokenize{chapters/chapter2/FEM:von-der-starken-zur-schwachen-form}}\label{\detokenize{chapters/chapter2/FEM::doc}}
\sphinxAtStartPar
Für die in Kapitel 1 eingeführten Randwertprobleme (Massenbilanz, Impulsbilanz und Energiebilanz) wird in diesem Kapitel die theoretischen Grundlagen der Finite Elemente Methode beschrieben. Für jedes der genannten Probleme ist der Ausgangspunkte die \sphinxstylestrong{starke} Form der beschreibenden partiellen Differentialgleichung. Am Beispiel der Impulsbilanz soll exemplarisch die zugehörige \sphinxstylestrong{schwache} Form hergeleitet werden. Die Begrifflichkeit stark und schwach bezieht sich hierbei auf die Stetigkeits\sphinxhyphen{} und Differenzierbarkeitsanforderungen der gesuchten Lösung \(\bm{u}\). Die starke Form hat somit \sphinxstylestrong{immer} höhere Anforderungen an die Lösung.

\begin{figure}[htbp]
\centering
\capstart

\noindent\sphinxincludegraphics[height=400\sphinxpxdimen]{{Impulsbilanz}.jpg}
\caption{Körper \(\mathcal{B}\) unter Einwirkung der externen Oberflächenkraft \(\tilde{\bm{t}}\) und der Volumenlast \(\bm{b}\)}\label{\detokenize{chapters/chapter2/FEM:impulsbilanz-fig}}\end{figure}

\sphinxAtStartPar
Ausgangspunkt für die exemplarische Herleitung ist die Impulsbilanz in der quasistatischen Form (ohne Trägheitsterme):
\begin{equation}\label{equation:chapters/chapter2/FEM:weakform_01}
\begin{split}0 = \div{\bm{\sigma}} + \rho \bm{b} \end{split}
\end{equation}
\sphinxAtStartPar
Diese starke form wird jetzt mit einer beliebigen vektorwertigen Testfunktion \(\delta \bm{u}\) multipliziert und über das betrachtete Gebiet \(\mathcal{B}\) (den betrachteten Körper) integriert.
\begin{equation}\label{equation:chapters/chapter2/FEM:weakform_02}
\begin{split}0 = \int_{\mathcal{B}} \delta  \bm{u}\T \left(\div{\bm{\sigma}} + \rho \bm{b} \right) \dV\end{split}
\end{equation}
\sphinxAtStartPar
Dieser Schritt bedeutet, dass wir im Weiteren versuchen werden, die gekoppelten partiellen Differentialgleichungen nicht punktuell exakt, sondern im gewichteten integralen Mittel zu erfüllen. Für elastomechanische Problemstellungen kann die Testfunktion \(\delta \bm{u}\) als virtuelle Verrückung aufgefasst werden und die Gleichung \eqref{equation:chapters/chapter2/FEM:weakform_02} als das bekannte Prinzip der virtuellen Verrückung aufgefasst werden. Das hier gezeigte Vorgehen ist aber unabhängig von der Elastostatik allgemeingültig.

\sphinxAtStartPar
Als nächstes integrieren wir die Gleichung \eqref{equation:chapters/chapter2/FEM:weakform_02} partiell. Dafür wenden wir zunächst die Produktregel auf den Divergenzterm an:
\begin{equation}\label{equation:chapters/chapter2/FEM:divergenzsatz}
\begin{split}\int_{\mathcal{B}} \delta \bm{u}\T \div{\bm{\sigma}} \dV = \int_{\mathcal{B}} \div{\delta \bm{u}\T \bm{\sigma}} \dV -\int_{\mathcal{B}} \underbrace{\grad{\delta\bm{u}\T}}_{\delta \bm{\eb}}\bm{\sigma} \dV \; . \end{split}
\end{equation}
\sphinxAtStartPar
Der Wechsel vom Divergenz\sphinxhyphen{}Operator zum Gradienten\sphinxhyphen{}Operator im zweiten Term auf der rechten Seite lässt sich schlüssig damit erklären, dass das Resultat ein Skalar sein soll. Während die Divergenz eines Vektors einen Skalar liefert, ergibt der Gradient eines Vektors einen Tensor zweiter Ordnung. Dieser Tensor wird dann in einer doppelten skalaren Multiplikation mit einem weiteren Tensor zweiter Ordnung verknüpft, was letztlich zu einem skalaren Ergebnis führt.

\sphinxAtStartPar
Nun lässt sich mit dem Gauß’schen Ingegralsatz der erste Term auf der rechten Seite von Gleichung \eqref{equation:chapters/chapter2/FEM:divergenzsatz} als Integral über den Rand formulieren:
\begin{equation}\label{equation:chapters/chapter2/FEM:gaussIntegralTheorem}
\begin{split}\int_{\mathcal{B}} \div{\delta \bm{u}\T  \bm{\sigma}} \dV = \int_{\partial\mathcal{B}} \delta \bm{u}\T \bm{\sigma}  \bm{n} \dA = \int_{\partial\mathcal{B}} \delta \bm{u}\T \bm{t} \dA \; . \end{split}
\end{equation}
\sphinxAtStartPar
Dieses Resultat wird jetzt wieder in Gleichung \eqref{equation:chapters/chapter2/FEM:weakform_02} eingesetzt und wir erhalten:
\begin{equation}\label{equation:chapters/chapter2/FEM:weakform_03}
\begin{split}\int_{\mathcal{B}} \delta\eb\T  \bm{\sigma} \dV = \int_{\mathcal{B}} \delta \bm{u}\T \rho \bm{b} \dV + \int_{\partial\mathcal{B}} \delta \bm{u}\T  \bm{t} \dA \; .\end{split}
\end{equation}
\begin{sphinxadmonition}{note}{Was haben wir damit erreicht?}
\begin{itemize}
\item {} 
\sphinxAtStartPar
der Oberflächenterm in \eqref{equation:chapters/chapter2/FEM:weakform_03} entspricht den Kraftrandbedingungen

\item {} 
\sphinxAtStartPar
der Term auf der linken Seite in \eqref{equation:chapters/chapter2/FEM:weakform_03} enthält nur noch Ableitungen 1. Ordnung von \(\bm{u}\), während in \eqref{equation:chapters/chapter2/FEM:weakform_01} Ableitungen 2. Ordnung gefordert wurden \(\rightarrow\) schwächere Anforderungen an \(\bm{u}\)!

\item {} 
\sphinxAtStartPar
Gleichung \eqref{equation:chapters/chapter2/FEM:weakform_03} ist exakt das Prinzip der virtuellen Verrückungen
\begin{itemize}
\item {} 
\sphinxAtStartPar
virtuelle Arbeit der Spannungen an den Verzerrungen ist gleich der virtuelle Arbeit der eingeprägten Volumenkräfte plus die virtuelle Arbeit der Oberflächenkräfte

\end{itemize}

\end{itemize}
\end{sphinxadmonition}

\sphinxAtStartPar
Setzen wir jetzt noch als Materialmodell die Beschreibung der linearen Elastizität ein, so erhalten wir final:
\begin{equation}\label{equation:chapters/chapter2/FEM:weakform_04}
\begin{split}\int_{\mathcal{B}} \delta\eb\T \bm{C} \eb \dV = \int_{\mathcal{B}} \delta \bm{u}\T \rho \bm{b} \dV + \int_{\partial\mathcal{B}} \delta \bm{u}\T \cdot \bm{t} \dA \; .\end{split}
\end{equation}
\sphinxstepscope




\section{Die Schwache Form am Beispiel des Stabes}
\label{\detokenize{chapters/chapter2/1dFEM:die-schwache-form-am-beispiel-des-stabes}}\label{\detokenize{chapters/chapter2/1dFEM::doc}}
\begin{figure}[htbp]
\centering
\capstart

\noindent\sphinxincludegraphics[height=400\sphinxpxdimen]{{Stab}.png}
\caption{Stab unter Eigengewicht und externer Kraft.}\label{\detokenize{chapters/chapter2/1dFEM:stab-fig}}\end{figure}

\sphinxAtStartPar
Auch hier beginnen wir mit der starken Form der Differentialgleichung:
\begin{equation}\label{equation:chapters/chapter2/1dFEM:stabdglsimple_2}
\begin{split} {EA} u^{\prime \prime} =  - {n(x)} A \; ,\end{split}
\end{equation}
\begin{sphinxadmonition}{note}{Analytische Lösung}

\sphinxAtStartPar
Die analytische Lösung erhalten wir durch zweifache Integration und einsetzen der Randbedingungen:
\label{equation:chapters/chapter2/1dFEM:00151201-f4db-4704-aade-97a5940eee76}\begin{equation}
u(x) = \left( \frac{F}{EA} + \frac{n \ell}{E}\right)\cdot x - \frac{1}{2} \frac{n}{E} x^2 \; . 
\end{equation}
\sphinxAtStartPar
Die Schnittkraft \(N\) berechnet sich wie folgt:
\label{equation:chapters/chapter2/1dFEM:b57581e6-42ec-49ae-a758-3a18ea1bdf52}\begin{equation}
N(x) = F + n A (\ell-x). 
\end{equation}\end{sphinxadmonition}

\sphinxAtStartPar
Zur Herleitung der Finiten Elemente Form multiplizieren  wir die starke Form mit der Testfunktion \(\delta u\) und integrieren die Gleichung über das Gebiet:
\begin{equation}\label{equation:chapters/chapter2/1dFEM:stabdglsimple_weak}
\begin{split} \int_{0}^{\ell} \delta u u^{\prime \prime} \dx = \int_{0}^{\ell} - \delta u \frac{n(x)}{EA} \dx\; .\end{split}
\end{equation}
\sphinxAtStartPar
Der Ausdruck auf der linken Seite wird umgeformt zu:
\begin{equation}\label{equation:chapters/chapter2/1dFEM:stabdglsimple_product}
\begin{split} \int_{0}^{\ell} \delta u u^{\prime \prime} \dx = \int_{0}^{\ell} \left(\delta u u^{\prime}\right)^{\prime} - \delta u^{\prime}  u^{\prime}  \dx\; .\end{split}
\end{equation}
\sphinxAtStartPar
Ersetzen wir nun \(\delta u^{\prime}=\delta \e\) und \(u^{\prime}=\e\) und setzen die obige Produktregel in Gleichung \eqref{equation:chapters/chapter2/1dFEM:stabdglsimple_weak}, so erhalten wir:
\begin{equation}\label{equation:chapters/chapter2/1dFEM:stabdglsimple_weak2}
\begin{split}\begin{align}
 \int_{0}^{\ell} \delta \e \cdot E\e \, A \dx & = \left[\delta u \cdot \sigma A \right]_0^{\ell} + \int_0^{\ell} \delta u \cdot n(x) \dx\;  \\
 & = \left[ \delta u(\ell) \cdot F + \delta u(0) \cdot 0\right]+ \int_0^{\ell} \delta u \cdot n(x) \dx\; .
 \end{align}\end{split}
\end{equation}

\subsection{Das Verfahren von Ritz}
\label{\detokenize{chapters/chapter2/1dFEM:das-verfahren-von-ritz}}
\sphinxAtStartPar
Als Einführung in die Finite\sphinxhyphen{}Element\sphinxhyphen{}Methoden beginnen wir mit dem Verfahren, welches von Walter Ritz (1878 \sphinxhyphen{} 1909) eingeführt wurde. Dies fußt auf dem Prinzip der virtuellen Arbeit, formuliert als \(\delta U − \delta W = 0\), wobei \(\delta U\) für die Arbeit der inneren Kräfte und \(\delta W\) für die der äußeren Kräfte steht. Für einen einfachen Stab ergibt sich die obige schwache Form \eqref{equation:chapters/chapter2/1dFEM:stabdglsimple_weak2}.

\sphinxAtStartPar
Der nächste Schritt ist die Aufstellung einer Näherungslösung \(u_h\) für das Verschiebungsfeld \(u\), zum Beispiel durch eine quadratische Funktion:
\begin{equation*}
\begin{split}u_h(x) = a_0 + a_1 x + a_2 x^2,\end{split}
\end{equation*}
\sphinxAtStartPar
wobei die Koeffizienten \(a_i\) durch Minimierung der virtuellen Arbeit ermittelt werden müssen. Diese Parameter müssen gleichzeitig kinematische Verträglichkeit garantieren und somit die geometrischen Randbedingungen einhalten. Aus \(u(0) = 0\) ergibt sich dann \(a_0 = 0\). Die benötigte Ableitung des Näherungsansatzes lautet daher:
\begin{equation*}
\begin{split}\e = u^{\prime} = a_1 + 2a_2 x.\end{split}
\end{equation*}
\sphinxAtStartPar
Die Variationen dieses Ansatzes werden durch Variieren der Koeffizienten gebildet:
\begin{equation*}
\begin{split}\delta u(x) = \delta a_1 x + \delta a_2 x^2\end{split}
\end{equation*}
\sphinxAtStartPar
und
\begin{equation*}
\begin{split}\delta \e= \delta a_1 + 2 \delta a_2 x.\end{split}
\end{equation*}
\sphinxAtStartPar
Eingesetzt in das Prinzip der virtuellen Arbeit führt es zu:
\begin{equation*}
\begin{split}EA \int_0^{\ell} (\delta a_1 + 2 \delta a_2 x) (a_1 +2 a_2 x) \dx = F (a_1 \ell +a_2 \ell^2) + \int_0^{\ell} (\delta a_1 x + \delta a_2 x^2) n A \dx \; .\end{split}
\end{equation*}
\sphinxAtStartPar
Nach Integration und Herausziehen der Variationen resultiert daraus:
\begin{equation*}
\begin{split}\delta a_1\left[EA \ell (a_1 + a_2 \ell) - \frac{1}{2} n A \ell^2 - F \ell\right] + \delta a_2 \left[EA\ell^2(a_1 + \frac{4}{3}a_2\ell) - \frac{1}{3}nA\ell^3 - F\ell^2\right] = 0.\end{split}
\end{equation*}
\sphinxAtStartPar
Da die Variationen beliebig sind, müssen die in eckigen Klammern stehenden Terme jeweils null sein. Daraus folgt ein lineares Gleichungssystem zur Bestimmung der Koeffizienten, das gelöst wird als:
\begin{equation*}
\begin{split}a_1 = \frac{F}{EA} + \frac{n \ell}{E} \qquad \text{und} \qquad  a_2 = -\frac{n}{2E}.\end{split}
\end{equation*}
\sphinxAtStartPar
Somit erhalten wir als Näherungslösung für diesen Fall:
\begin{equation*}
\begin{split}u_h(x) = \left(\frac{F}{EA} + \frac{n \ell}{E}\right)x - \left(\frac{n}{2E}\right)x^2.\end{split}
\end{equation*}
\sphinxAtStartPar
Man kann leicht erkennen, dass mit diesem quadratischen Ansatz die analytische Lösung der Differentialgleichung rekonstruiert werden konnte.

\begin{sphinxadmonition}{note}{Problem}

\sphinxAtStartPar
Ein Problem bei dem Ritz’schen Verfahren ist die Wahl der Ansatzfunktion. Diese muss die Randbedingungen erfüllen und die Stetigkeitsanforderungen der schwachen Form. Für komplexere Probleme ist die Wahl der Ansatzfunktion nicht trivial \sphinxhyphen{} und oft auch nicht möglich.
\end{sphinxadmonition}


\subsection{Finite Elemente Formulierung für den Stab}
\label{\detokenize{chapters/chapter2/1dFEM:finite-elemente-formulierung-fur-den-stab}}

\subsubsection{Diskretisierung der schwachen Form}
\label{\detokenize{chapters/chapter2/1dFEM:diskretisierung-der-schwachen-form}}
\begin{figure}[htbp]
\centering
\capstart

\noindent\sphinxincludegraphics[width=600\sphinxpxdimen]{{Stab_element}.png}
\caption{Stab unter Eigengewicht und externer Kraft mit Stabelementen diskretisiert.}\label{\detokenize{chapters/chapter2/1dFEM:stabelement}}\end{figure}

\sphinxAtStartPar
Im vorangegangenen Abschnitt haben wir uns von der Effektivität des Prinzips der virtuellen Arbeit in Kombination mit dem Ritz’schen Verfahren überzeugt. Es erweist sich jedoch oft als schwierig, bei komplex geformten Strukturen, die einer vielschichtigen Belastung ausgesetzt sind, einen adäquaten Näherungsansatz zu finden. Die Finite\sphinxhyphen{}Elemente\sphinxhyphen{}Methode (FEM) löst dieses Problem, indem sie simple Ansatzfunktionen auf überschaubar gestalteten Teilbereichen – den sogenannten finiten Elementen – anwendet und diese dann systematisch für das gesamte Gebiet integriert.

\sphinxAtStartPar
Als Beispiel hierfür dient die im Bild \hyperref[\detokenize{chapters/chapter2/1dFEM:stabelement}]{Abb.\@ \ref{\detokenize{chapters/chapter2/1dFEM:stabelement}}} dargestellte Diskretisierung eines Stabes mit der Gesamtlänge \(\ell\), der in zwei finite Elemente mit jeweils der Länge \(l_i = \frac{\ell}{2}\) unterteilt wird.

\begin{sphinxadmonition}{note}{Allgemeines zu Finiten Elementen}

\sphinxAtStartPar
Ein Finites Element besteht in dem vorliegenden Beispiel aus 2 Knoten. Sowohl die Knoten, als auch die Elemente werden im Allgemeinen nummeriert, damit man sie eindeutig ansprechen kann. Im Bild sind die Elementnummern in den rechteckigen Kästen neben dem Element eingetragen. Die Knotennummer sind in den Kreisen neben den Knoten dargestellt. Die Knoten sind die Träger der primären Feldvariablen (hier: Verschiebung \(u\)) und Ziel der Finiten Elemente Berechnung ist die Bestimmung der primären Variablen, auch Freiheitsgrade (English: Degree of freedom \sphinxstylestrong{Dof}) an den Knoten. Im Bild \hyperref[\detokenize{chapters/chapter2/1dFEM:stabelement}]{Abb.\@ \ref{\detokenize{chapters/chapter2/1dFEM:stabelement}}} auf der rechten Seite ist ein einzelnes Stabelement dargestellt. Das Stabelement hat 2 lokale Knoten und um die Verschiebung an diesen Knoten eindeutig zuzuordnen verwenden wir die Notation:

\sphinxAtStartPar
\(
u_1^{e} \text{ Verschiebung u am lokalen Knoten 1 von Element e}
\)

\sphinxAtStartPar
\(
u_2^{e} \text{ Verschiebung u am lokalen Knoten 2 von Element e}
\)
\end{sphinxadmonition}

\sphinxAtStartPar
Auf diesen Elementen werden dann einfache Ansatzfunktionen verwendet, welche eine lineare Approximation des Verschiebungsfeldes darstellen:
\begin{equation}\label{equation:chapters/chapter2/1dFEM:LinearDisplacementAnsatz}
\begin{split}\begin{align}
 N_1(\xi) & = (1-\xi) \qquad N_2(\xi)= \xi  \\
 u_h(\xi) & = N_1(\xi) u_1^{(e)}+ N_2(\xi) u_2^{(e)} \\
 u_h(\xi) & = \sum_{I=1}^2 N_I(\xi) u_I^{(e)}
 \end{align}\end{split}
\end{equation}
\sphinxAtStartPar
hierbei wurde das lokale Koordinatensystem \(\xi = \frac{x}{\ell_e} \; \in \; [0,1]\)  eingeführt, damit der gewählte Ansatz für alle Stabelmenete, unabhängig der Stablänge \(\ell_e\) gültig ist. In der Matrixschreibweise erhält man für den Ausdruck \eqref{equation:chapters/chapter2/1dFEM:LinearDisplacementAnsatz}:
\begin{equation}\label{equation:chapters/chapter2/1dFEM:LinearDisplacementAnsatz2}
\begin{split}\begin{align}
  u_h(\xi) & = \underbrace{\begin{bmatrix} (1-\xi) & \xi \end{bmatrix}}_{\bm{N}\T} \underbrace{\begin{bmatrix} u_1^{(e)} \\ u_2^{(e)}\end{bmatrix}}_{\bm{u}^{(e)}} = \bm{N}\T \bm{u}^{(e)}
 \end{align}\end{split}
\end{equation}
\begin{sphinxuseclass}{cell}
\begin{sphinxuseclass}{tag_hide-input}\begin{sphinxVerbatimOutput}

\begin{sphinxuseclass}{cell_output}
\begin{sphinxVerbatim}[commandchars=\\\{\}]
\PYGZlt{}matplotlib.legend.Legend at 0x7f259c515b70\PYGZgt{}
\end{sphinxVerbatim}

\noindent\sphinxincludegraphics{{c7ce0870469968c74561338cf1ac8522e40620c8f17b286370fc0a18b34d24e7}.png}

\end{sphinxuseclass}\end{sphinxVerbatimOutput}

\end{sphinxuseclass}
\end{sphinxuseclass}
\sphinxAtStartPar
Zur Berechnung des Prinzips der virtuellen Verrückungen bzw. der schwachen Form wird zusätzlich die Ableitung des Ansatzes für \(u_h\) benötigt. Diese repräsentiert die Dehnung des Stabes \(\e = \Dd{u}{x}\). Da der Ansatz über die normalisierte Koordinate \(\xi\) bestimmt wurde berechnet sich die Ableitung über die Kettenregel:
\begin{equation*}
\begin{split}\e_h = \Dd{u_h}{x} = \Dd{u_h}{\xi} \underbrace{\Dd{\xi}{x}}_{\frac{1}{\ell_e}} = \frac{1}{\ell_e} \begin{bmatrix} -1 & 1 \end{bmatrix} \begin{bmatrix} u_1^{(e)} \\ u_2^{(e)}\end{bmatrix} \end{split}
\end{equation*}
\sphinxAtStartPar
Für die Variation \(\delta u\) und \(\delta \e\) wählen wir den gleichen Ansatz:
\begin{equation*}
\begin{split}\delta u_h = \begin{bmatrix}  (1-\xi) & \xi \end{bmatrix} \begin{bmatrix} \delta u_1^{(e)} \\ \delta u_2^{(e)}\end{bmatrix} \qquad \delta \e_h = \frac{1}{\ell_e} \begin{bmatrix} -1 & 1 \end{bmatrix} \begin{bmatrix} \delta u_1^{(e)} \\ \delta u_2^{(e)}\end{bmatrix} \end{split}
\end{equation*}
\sphinxAtStartPar
Betrachten wir nun das Prinzip der virtuellen Verrückungen in Gleichung \eqref{equation:chapters/chapter2/1dFEM:stabdglsimple_weak2}:
\begin{equation*}
\begin{split}\begin{align}
 \textcolor{green}{\int_{0}^{\ell} \delta \e \cdot E\e \, A \dx} -\textcolor{red} {\int_0^{\ell} \delta u \cdot n A \dx} - \textcolor{blue}{\delta u(\ell) \cdot F} = 0\; 
 \end{align}\end{split}
\end{equation*}
\sphinxAtStartPar
und setzen hier unseren gewählten Ansatz für \(u\), \(\e\), \(\delta u\) und \(\delta \e\) ein, dann erhalten wir die folgende Gleichung:
\begin{equation}\label{equation:chapters/chapter2/1dFEM:stabFEM1}
\begin{split}\begin{align}
 \textcolor{green}{\int_{0}^{1} \frac{1}{\ell_e} \begin{bmatrix} -1 & 1 \end{bmatrix} \begin{bmatrix} \delta u_1^{(e)} \\ \delta u_2^{(e)}\end{bmatrix}  EA \frac{1}{\ell_e} \begin{bmatrix} -1 & 1 \end{bmatrix} \begin{bmatrix} u_1^{(e)} \\ u_2^{(e)}\end{bmatrix}   \underbrace{\ell_e \d \xi}_{\dx}} \\
 - \textcolor{red}{\int_0^{1} \begin{bmatrix} (1-\xi) & \xi \end{bmatrix} \begin{bmatrix} \delta u_1^{(e)} \\ \delta u_2^{(e)}\end{bmatrix}  n A \ell_e \d \xi} - \textcolor{blue}{\begin{bmatrix} \delta u_1^{(e)} & \delta u_2^{(e)}\end{bmatrix} \begin{bmatrix} S_1^{(e)} \\ S_2^{(e)} \end{bmatrix}}  = 0\; 
\end{align}\end{split}
\end{equation}
\sphinxAtStartPar
Die Variation der Knotenverschiebung \(\delta u_I\) und die Knotenverschiebung \(u_I\) sind nicht abhängig von \(\xi\) und können somit aus dem Integranden herrausgezogen werden:
\begin{equation}\label{equation:chapters/chapter2/1dFEM:stabFEM2}
\begin{split}\begin{align}
 \begin{bmatrix} \delta u_1^{(e)} & \delta u_2^{(e)}\end{bmatrix}\left\{ \textcolor{green}{   \int_{0}^{1} EA \frac{1}{\ell_e} \begin{bmatrix} 1 & -1 \\ -1 & 1\end{bmatrix}  \d \xi \begin{bmatrix} u_1^{(e)} \\ u_2^{(e)}\end{bmatrix}} \\
 - \textcolor{red}{n \ell_e A \int_0^{1} \begin{bmatrix} (1-\xi)  \\ \xi \end{bmatrix}   \d \xi} - \textcolor{blue}{ \begin{bmatrix} S_1^{(e)} \\ S_2^{(e)} \end{bmatrix}} \right\}  = 0\; 
\end{align}\end{split}
\end{equation}
\sphinxAtStartPar
Da die Variation beliebig ist, muss die Gleichung für jeden Faktor separat erfüllt sein. Somit erhält man für jedes Element die Gleichung:
\begin{equation}\label{equation:chapters/chapter2/1dFEM:stabFEM3}
\begin{split}\begin{align}
  \textcolor{green}{   \underbrace{\int_{0}^{1} EA \frac{1}{\ell_e} \begin{bmatrix} 1 & -1 \\ -1 & 1\end{bmatrix}  \d \xi}_{\bm{K}^{(e)}} \begin{bmatrix} u_1^{(e)} \\ u_2^{(e)}\end{bmatrix}} 
 - \textcolor{red}{ \underbrace{n \ell_e A \int_0^{1} \begin{bmatrix} (1-\xi) \\ \xi  \end{bmatrix}   \d \xi}_{\bm{F}_V^{(e)}}} - \textcolor{blue}{ \underbrace{\begin{bmatrix} S_1^{(e)} \\ S_2^{(e)} \end{bmatrix}}_{\bm{F}_K^{(e)}}}  & = 0\; \\
 \textcolor{green}{\bm{K}^{(e)} \bm{u}^{(e)}} &=  \textcolor{red}{\bm{F}_V^{(e)}} + \textcolor{blue}{\bm{F}_K^{(e)} } 
\end{align}\end{split}
\end{equation}
\sphinxAtStartPar
Dabei ist \(\bm{K}^{(e)}\) die \sphinxstylestrong{Elementsteifigkeitsmatrix}, \(\bm{F}_V^{(e)}\) ist der \sphinxstylestrong{Vektor der äquivalenten Knotenkräfte} aus den verteilten Lasten und \(\bm{F}_K^{(e)}\) sind die an den Knotenpunkten \sphinxstylestrong{eingeprägten Kräfte}. Für dieses einfache Element können die Integrale in Gleichung \eqref{equation:chapters/chapter2/1dFEM:stabFEM3} analytisch berechnet werden. Nach der Integration erhält man:
\begin{equation}\label{equation:chapters/chapter2/1dFEM:stabFEM4}
\begin{split}\begin{align}
 \bm{K}^{(e)} & = EA \frac{1}{\ell_e} \begin{bmatrix} 1 & -1 \\ -1 & 1\end{bmatrix} \\
 \bm{F}_V^{(e)} & = n \ell_e A  \begin{bmatrix} \frac{1}{2} \\ \frac{1}{2} \end{bmatrix} 
\end{align}\end{split}
\end{equation}
\sphinxAtStartPar
Damit wurde die Differentialgleichung 2. Ordnung in eine algebraische Gleichung umgeformt. Wir haben jetzt für die zwei Finiten Elemente zwei \sphinxstylestrong{lokale}, lineare Gleichungssysteme. Diese sind jedoch nicht unabhängig voneinander und müssen im nächsten Schritt in ein \sphinxstylestrong{globales} Gleichungssystem assembliert werden.


\subsubsection{Assemblierung der Finiten Elemente}
\label{\detokenize{chapters/chapter2/1dFEM:assemblierung-der-finiten-elemente}}
\sphinxAtStartPar
Das Ziel dieses Abschnitts ist es, die Entwicklung der Gleichungen für das Gesamtsystem aus den Elementsteifigkeitsmatrizen zu beschreiben. Wir werden die Assemblierungoperationen vorstellen, die hierfür verwendet werden. Diese Operationen sind ein fester Bestandteil der Finite\sphinxhyphen{}Elemente\sphinxhyphen{}Methode (FEM) und kommen selbst bei den komplexesten Problemen zum Einsatz. Daher ist es wesentlich, diese Verfahren zu beherrschen, um die FEM zu erlernen.

\sphinxAtStartPar
Die Elemente im unserem dargestellten Beispiel (\hyperref[\detokenize{chapters/chapter2/1dFEM:stabelement}]{Abb.\@ \ref{\detokenize{chapters/chapter2/1dFEM:stabelement}}}) werden mit den Nummern 1 und 2 bezeichnet, während die Knoten von 1 bis 3 nummeriert sind; weder die Knoten noch die Elemente müssen in einem FEM\sphinxhyphen{}Netz in einer bestimmten Reihenfolge nummeriert sein. Wir kommen hierzu nochmal später in der Vorlesung.

\sphinxAtStartPar
Beachten Sie, dass die Kraftgrößen der Elemente \(S_i^{(e)}\) mit den Indizes 1 und 2 versehen sind. Dies sind die lokalen Knotennummern. Die Knoten des Netzes sind die globalen Knotennummern. Die lokalen Knotennummern eines Stabelements sind immer in der positiven \(\xi\)\sphinxhyphen{}Richtung mit 1, 2 nummeriert. Die globalen Knotennummern hingegen sind willkürlich.

\sphinxAtStartPar
Wir versuchen nun, die lokalen Kräfte \(S_i^{(e)}\) für das Element 1 mit den globalen Verschiebungen in Verbindung zu bringen. Betrachten wir zunächst den Knoten 2, welchen sich Element 1 und 2 teilen. Hier gilt für die Verschiebung:
\label{equation:chapters/chapter2/1dFEM:3deffa07-4eeb-4983-93bb-bf8bba989f7c}\begin{align}
  u_2^{(1)} = u_1^{(2)} = u_2 \; .
\end{align}\begin{equation}\label{equation:chapters/chapter2/1dFEM:stabFEM5}
\begin{split}\begin{align}
\begin{bmatrix}
0 \\
S_2^{(1)}\\
S_1^{(1)}
\end{bmatrix} + \frac{1}{2}n\ell_1 A_1 \begin{bmatrix}
0 \\
1\\
1
\end{bmatrix} = \frac{E_1A_1}{\ell_1}\begin{bmatrix}
0 & 0 & 0\\
0 & 1 & -1\\
0 & -1 & 1
\end{bmatrix} \begin{bmatrix}
u_1 \\
u_2\\
u_3
\end{bmatrix}
\end{align} \; .\end{split}
\end{equation}
\sphinxAtStartPar
Dazu haben wir lediglich das Elementgleichungssystem \eqref{equation:chapters/chapter2/1dFEM:stabFEM3} genommen und die lokalen Freiheitsgrade \(u_i^{(e)}\) durch ihre zugeordneten globalen Freiheitsgrade \(u_i\) gemäß den globalen Knotennummern ersetzt. Gleichzeitig haben wir bereits eine zusätzliche Gleichung (1. Zeile) eingeführt. Deren Sinn wird gleich ersichtlich. Das gleiche Vorgehen wenden wir nun auf das Element 2 an:
\begin{equation}\label{equation:chapters/chapter2/1dFEM:stabFEM6}
\begin{split}\begin{align}
\begin{bmatrix}
S_2^{(2)} \\
S_1^{(2)} \\
0
\end{bmatrix} + \frac{1}{2}n\ell_2 A_2 \begin{bmatrix}
0 \\
1\\
1
\end{bmatrix} = \frac{E_2A_2}{\ell_2}\begin{bmatrix}
1 & -1 & 0\\
-1 & 1 & 0\\
0 & 0 & 0
\end{bmatrix} \begin{bmatrix}
u_1 \\
u_2\\
u_3
\end{bmatrix}
\end{align} \; .\end{split}
\end{equation}
\sphinxAtStartPar
Nun liegen uns zwei Gleichungssysteme vor, welche sich auf die gleichen Freiheitsgrade beziehen. Durch eine Addition erhalten wir schließlich:
\begin{equation}\label{equation:chapters/chapter2/1dFEM:stabFEM7}
\begin{split}\begin{align}
\begin{bmatrix}
S_2^{(2)} \\
S_1^{(2)} + S_2^{(1)} \\
S_1^{(1)}
\end{bmatrix} + \frac{1}{4}n\ell A \begin{bmatrix}
1 \\
1+1\\
1
\end{bmatrix} = \frac{EA}{\ell}\begin{bmatrix}
1 & -1 & 0\\
-1 & 2 & -1\\
0 & -1 & 1
\end{bmatrix} \begin{bmatrix}
u_1 \\
u_2\\
u_3
\end{bmatrix}
\end{align} \; .\end{split}
\end{equation}
\sphinxAtStartPar
Es bleibt noch übrig die Stabkräfte (innere Kräfte) \(S_i\) mit den externen Kräften in Beziehung zu setzen. Für jeden Knoten kann das Kräftegleichgewicht aufgestellt werden:
\begin{equation}\label{equation:chapters/chapter2/1dFEM:stabFEM8}
\begin{split}\underbrace{\begin{bmatrix}
S_2^{(2)} \\
S_1^{(2)} \\
0
\end{bmatrix}}_{\bm{F}_K^{(2)}} + \underbrace{\begin{bmatrix}
0 \\
S_2^{(1)} \\
S_1^{(1)}
\end{bmatrix}}_{\bm{F}_K^{(1)}} = \underbrace{\begin{bmatrix}
F \\
0 \\
0
\end{bmatrix}}_{\bm{F}\rs{ext}} + \underbrace{\begin{bmatrix}
0 \\
0 \\
R
\end{bmatrix}}_{\bm{R}} \;.\end{split}
\end{equation}
\sphinxAtStartPar
Somit können wir die inneren Stabkräfte durch die externen Lasten \(\bm{F}\rs{ext}\) und Reaktionskräfte \(\bm{R}\) ersetzen:
\begin{equation}\label{equation:chapters/chapter2/1dFEM:stabFEM9}
\begin{split}\begin{align}
\begin{bmatrix}
F \\
0 \\
R
\end{bmatrix} + \frac{1}{4}n\ell A \begin{bmatrix}
1 \\
1+1\\
1
\end{bmatrix} = \frac{2EA}{\ell}\begin{bmatrix}
1 & -1 & 0\\
-1 & 2 & -1\\
0 & -1 & 1
\end{bmatrix} \begin{bmatrix}
u_1 \\
u_2\\
u_3
\end{bmatrix}
\end{align} \; .\end{split}
\end{equation}
\begin{sphinxadmonition}{note}{Direkte Assemblierung}

\sphinxAtStartPar
Innerhalb einer FE\sphinxhyphen{}Software werden die entsprechenden Matrizen natürlich nicht explizit zunächst auf die Größe der globalen Matrix „aufgeblasen“ um sie später zu addieren. Über die Zuordnung von lokaler zur globalen Knotennummer kann dies effizient direkt erfolgen.
\end{sphinxadmonition}


\subsubsection{Lösung des Gleichungssystems}
\label{\detokenize{chapters/chapter2/1dFEM:losung-des-gleichungssystems}}
\sphinxAtStartPar
In der vorliegenden Form ist die Systemsteifigkeitsmatrix singulär und daher kann das lineare Gleichungssystem so nicht gelöst werden. Das bedeutet aus physikalischer Sicht, dass das System in der Lage ist, Starrkörperbewegungen durchzuführen. Um dieses Problem zu beheben, müssen die kinematischen Randbedingungen integriert werden. Dies geschieht durch Partitionierung des linearen Gleichungssystems. Dabei ordnen wir die Verschiebungsgrößen in bekannte \(\bar{\bm{u}}\) und unbekannte \(\bm{u}\) Größen:
\begin{equation}\label{equation:chapters/chapter2/1dFEM:LGSsolve1}
\begin{split}\begin{bmatrix}
\bm{K}_{uu} & \bm{K}_{u\bar{u}} \\
\bm{K}_{\bar{u}u} & \bm{K}_{\bar{u}\bar{u}}
\end{bmatrix}
\begin{bmatrix}
\bm{u} \\
\bar{\bm{u}}
\end{bmatrix} = \begin{bmatrix} \bm{F}\rs{ext} \\ \bm{R} \end{bmatrix} + \bm{F}_v\end{split}
\end{equation}
\sphinxAtStartPar
Hierbei repräsentiert \(\bm{F}\rs{ext}\) die einwirkenden Knotenkräfte und \(\bm{R}\) die Reaktionskräfte, die den vorgeschriebenen Verschiebungsgrößen zugeordnet sind. Nun lässt sich die erste Zeile nach den unbekannten Verschiebungen auflösen:
\begin{equation}\label{equation:chapters/chapter2/1dFEM:LGSsolve2}
\begin{split}\begin{bmatrix}
\bm{K}_{uu}
\end{bmatrix}
\begin{bmatrix}
\bm{u}
\end{bmatrix} = \begin{bmatrix} \bm{F}\rs{ext}\end{bmatrix} + \bm{F}_v - \begin{bmatrix}
\bm{K}_{u\bar{u}} 
\end{bmatrix} \begin{bmatrix}
\bar{\bm{u}}
\end{bmatrix}\end{split}
\end{equation}
\sphinxAtStartPar
Mit der Lösung dieses linearen Gleichungssystems erhält man die unbekannten Knotenverschiebungen. Anschließend können die unbekannten Reaktionskräfte einfach durch Matrizenmultiplikation bestimmt werden:
\begin{equation}\label{equation:chapters/chapter2/1dFEM:LGSsolve3}
\begin{split}\begin{bmatrix} \bm{R} \end{bmatrix} = 
\begin{bmatrix}
\bm{K}_{\bar{u}u}
\end{bmatrix}
\begin{bmatrix}
\bm{u}
\end{bmatrix} + \begin{bmatrix}
\bm{K}_{\bar{u}\bar{u}} 
\end{bmatrix} \begin{bmatrix}
\bar{\bm{u}}
\end{bmatrix}- \bm{F}_v\end{split}
\end{equation}
\sphinxAtStartPar
Im vorliegenden Beispiel erhalten wir für das Gleichungssystem \eqref{equation:chapters/chapter2/1dFEM:LGSsolve2}:
\label{equation:chapters/chapter2/1dFEM:f3e6769c-2de7-45f1-9e4c-35f34e9b017a}\begin{align}
\begin{bmatrix}
F \\
0 \\
\end{bmatrix} + \frac{1}{4}n\ell A \begin{bmatrix}
1 \\
1+1
\end{bmatrix} - \frac{2EA}{\ell}\begin{bmatrix}
 0\\
-1
\end{bmatrix} \begin{bmatrix}
0
\end{bmatrix} & = \frac{2EA}{\ell}\begin{bmatrix}
1 & -1\\
-1 & 2 \\
\end{bmatrix} \begin{bmatrix}
u_1 \\
u_2\\
\end{bmatrix} \\
\begin{bmatrix}
F +\frac{1}{4}n\ell A \\
\frac{1}{2}n\ell A \\
\end{bmatrix} & =
\frac{2EA}{\ell}\begin{bmatrix}
1 & -1\\
-1 & 2 \\
\end{bmatrix} \begin{bmatrix}
u_1 \\
u_2\\
\end{bmatrix} 
\end{align}
\sphinxAtStartPar
mit der Lösung:
\label{equation:chapters/chapter2/1dFEM:47e5f1ef-69cb-42b8-b6ca-12465490e1a0}\begin{align}
\begin{bmatrix}
u_1 \\
u_2\\
\end{bmatrix} = \begin{bmatrix}
 \frac{F\ell}{EA} + \frac{n \ell^2}{2E}\\
\frac{F\ell}{2EA} + \frac{3 n \ell^2}{8E}
\end{bmatrix} 
\end{align}
\sphinxAtStartPar
Die unbekannte Reaktionskraft am Knoten 3 berechnet sich dann zu:
\label{equation:chapters/chapter2/1dFEM:b5f7695e-258e-4330-9f04-28eda736cd6c}\begin{align}
R + F_v & = \frac{2EA}{\ell} \begin{bmatrix}
0 & -1 
\end{bmatrix} \begin{bmatrix}
u_1 \\ u_2
\end{bmatrix} + \begin{bmatrix}
1
\end{bmatrix} \begin{bmatrix}
0
\end{bmatrix} \\
&= -\frac{2EA}{\ell} \left(\frac{F\ell}{2EA} + \frac{3 n \ell^2}{8E} \right)- \frac{1}{4}n\ell A= -F - n \ell A 
\end{align}

\subsubsection{Berechnung der Schnittgrößen}
\label{\detokenize{chapters/chapter2/1dFEM:berechnung-der-schnittgroszen}}
\sphinxAtStartPar
Die Schnittgrößen \(N^{(i)}=\sigma^{(i)} A\) werden in einer Nachlaufrechnung (Postprocessing) mittels des Materialmodell \(\sigma^{(i)}=E\e^{(i)}\) bestimmt. Dabei sind die Schnittkräfte:
\begin{equation}\label{equation:chapters/chapter2/1dFEM:postproc}
\begin{split}\begin{align}
N^{(1)} & = 2\frac{EA}{\ell} \begin{bmatrix} -1 & - \end{bmatrix} \begin{bmatrix} u_3 \\ u_2 \end{bmatrix} = F + \frac{3}{4} n A \ell \\
N^{(2)} & = 2\frac{EA}{\ell} \begin{bmatrix} -1 & 1 \end{bmatrix} \begin{bmatrix} u_2 \\ u_1 \end{bmatrix}= F + \frac{1}{4} n A \ell
\end{align}\end{split}
\end{equation}

\subsubsection{Diskussion der Ergebnisse}
\label{\detokenize{chapters/chapter2/1dFEM:diskussion-der-ergebnisse}}
\begin{sphinxuseclass}{cell}
\begin{sphinxuseclass}{tag_remove-input}\begin{sphinxVerbatimOutput}

\begin{sphinxuseclass}{cell_output}
\begin{figure}[htbp]
\centering
\capstart

\noindent\sphinxincludegraphics{{c132fbcde5a0a9cd010188f575e66ae1f2b7a6eb1dbd7b12723ec7d3079eaa54}.png}
\caption{Vergleich der analytischen Lösung mit der FE\sphinxhyphen{}Lösung.}\label{\detokenize{chapters/chapter2/1dFEM:fe-compare}}\end{figure}

\end{sphinxuseclass}\end{sphinxVerbatimOutput}

\end{sphinxuseclass}
\end{sphinxuseclass}
\sphinxAtStartPar
Abbildung \hyperref[\detokenize{chapters/chapter2/1dFEM:fe-compare}]{Abb.\@ \ref{\detokenize{chapters/chapter2/1dFEM:fe-compare}}} links zeigt einen Vergleich zwischen dem mit der Finite\sphinxhyphen{}Elemente\sphinxhyphen{}Methode (FEM) ermittelten Verschiebungsfeld und der analytischen Lösung. Letztere zeigt aufgrund der Eigengewichtsbelastung einen quadratischen Verlauf, während die FEM diesen in Abschnitten durch lineare Näherungen darstellt. In diesem speziellen Fall ist erkennbar, dass die FEM die analytische Lösung an den Knotenpunkten exakt wiedergibt.

\sphinxAtStartPar
Im rechten Teil von Abbildung \hyperref[\detokenize{chapters/chapter2/1dFEM:fe-compare}]{Abb.\@ \ref{\detokenize{chapters/chapter2/1dFEM:fe-compare}}} wird die FE\sphinxhyphen{}Approximation für den Verlauf der Schnittkräfte mit der analytischen Lösung kontrastiert. Die analytische Lösung zeigt ein lineares Wachstum der Normalkraft auf, das jedoch durch die hier verwendeten linearen Ansatzfunktionen lediglich stückweise durch eine konstante Annäherung wiedergegeben wird. Dabei treten an den Grenzen der Elemente Diskontinuitäten in den von der FEM berechneten Schnittkraftverläufen auf. Auffällig ist auch in diesem Beispiel, dass die analytische Lösung im Zentrum jedes Elements exakt erreicht wird.

\sphinxAtStartPar
Die Genauigkeit dieser Näherungslösung kann augenscheinlich gesteigert werden, indem man die Anzahl der Elemente erhöht. Dies führt dazu, dass die Sprünge im Verlauf der Schnittkräfte bestehen bleiben, aber kleiner ausfallen. Um eine qualitativ signifikante Verbesserung zu erzielen, empfiehlt es sich, Elemente mit quadratischen Ansatzfunktionen für das Verschiebungsfeld zu nutzen. In diesem Fall wird bereits mit einem einzigen Element, das über drei Knotenpunkte verfügt, die exakte Lösung erlangt. In dieser Situation wäre dann die FEM äquivalent dem Ritz’schen Verfahren.


\subsubsection{Konvergenz der Ergebnisse}
\label{\detokenize{chapters/chapter2/1dFEM:konvergenz-der-ergebnisse}}
\begin{figure}[htbp]
\centering
\capstart

\noindent\sphinxincludegraphics[width=600\sphinxpxdimen]{{Stab_Normalkraft_refined}.png}
\caption{Ergebnisse für die Normalspannung in einem Stab unter Eigengewicht mit einer erhöhten Netzfeinheit.}\label{\detokenize{chapters/chapter2/1dFEM:stabelement-refined}}\end{figure}

\sphinxAtStartPar
Bei einer immer feiner werdenen Diskretisierung des Stabes, konvergiert die Lösung gegen die analytische Lösung. Dennoch liegen wie in Abbildung \hyperref[\detokenize{chapters/chapter2/1dFEM:stabelement-refined}]{Abb.\@ \ref{\detokenize{chapters/chapter2/1dFEM:stabelement-refined}}} zu sehen, immernoch Spannungssprünge an den Elementgrenzen vor. Betrachtet man die Fehler der Spannungsergebnisse, so zeigt sich, dass dieser Fehler mit steigender Anzahl an Freihheitsgraden abnimmt. Dies ist in Abbildung \hyperref[\detokenize{chapters/chapter2/1dFEM:stabelement-error}]{Abb.\@ \ref{\detokenize{chapters/chapter2/1dFEM:stabelement-error}}} zu sehen. Im doppelt logaritmischen Maßstab erhält man eine Gerade für die Abnahme des Fehlers. Auch bei der verwendung von Ansatzfunktionen höherer Ordnung reduziert sich der Fehler nur linear im doppeltlogarithmischen Maßstab. Die Steigung ist jedoch höher, sodass im Allgemeinen mit einer verbesserten Konvergenz zu rechnen ist.

\begin{figure}[htbp]
\centering
\capstart

\noindent\sphinxincludegraphics[width=600\sphinxpxdimen]{{Stab_error_convergence}.png}
\caption{Konvergenz der Spannungsergebniss in einem Stab unter Eigengewicht mit steigender Anzahl an Freihheitsgraden.}\label{\detokenize{chapters/chapter2/1dFEM:stabelement-error}}\end{figure}


\subsection{Interactives Notebook}
\label{\detokenize{chapters/chapter2/1dFEM:interactives-notebook}}
\sphinxAtStartPar
asierend auf der präsentierten Theorie wurde ein Jupyter Notebook erstellt. Dieses kann man ohne Systemvorraussetzungen im Browser ausführen:

\sphinxAtStartPar
StabFEM for Binder: \sphinxhref{https://mybinder.org/v2/gh/steffenbeese/FEM\_I\_Notebooks/main?urlpath=\%2Fdoc\%2Ftree\%2FNotebook\_StabFEM.ipynb}{\sphinxincludegraphics{{badge_logo}.png}}

\sphinxstepscope




\section{Die Balken\sphinxhyphen{}FEM}
\label{\detokenize{chapters/chapter2/BalkenFEM:die-balken-fem}}\label{\detokenize{chapters/chapter2/BalkenFEM::doc}}
\begin{figure}[htbp]
\centering
\capstart

\noindent\sphinxincludegraphics[height=400\sphinxpxdimen]{{Balken_TM21}.png}
\caption{Schnittgrößen und Kinematik am Balkenelement nach {[}\hyperlink{cite.quellen:id13}{Gross \sphinxstyleemphasis{et al.}, 2007}{]}}\label{\detokenize{chapters/chapter2/BalkenFEM:balken02}}\end{figure}

\sphinxAtStartPar
Als weiteren Vertreter der Strukturelemente innerhalb der Finiten Elemente Methode (FEM) betrachten wir im Folgenden die Balken\sphinxhyphen{}FEM. Die Impulsbilanz des Euler\sphinxhyphen{}Bernoulli\sphinxhyphen{}Balkens ist eine Differentialgleichung 4. Ordnung. Bei Voraussetzung einer konstanten Biegesteifigkeit \(EI_y\) lautet diese:
\begin{equation}\label{equation:chapters/chapter2/BalkenFEM:balkendglsimple_2}
\begin{split}  w^{IV} =  \frac{q(x)}{E I_y} \; .\end{split}
\end{equation}
\sphinxAtStartPar
Die obige \sphinxstylestrong{starke Form} der Balken\sphinxhyphen{}DGL wird im Folgenden in eine \sphinxstylestrong{schwache Form} umgewandelt. Formal wird die Differentialgleichung wieder mit einer beliebigen Testfunktion \(\delta w\) multipliziert und über den Balken integriert. Die schwache Form lautet dann:
\begin{equation}\label{equation:chapters/chapter2/BalkenFEM:weakBalken_01}
\begin{split}  EI_y \int_L \delta w'' w'' \dx - \int_L q(x) \delta w \dx - F_I \delta w_I - M_J \delta w'_J = 0 \; . \end{split}
\end{equation}
\sphinxAtStartPar
Hierbei stellen \(F_I\) und \(M_I\) die an den Knoten angreifenden Einzellasten und Momente dar. Es ist zu beachten, dass die zweite Ableitung der Durchbiegung die höchste Ableitungsordnung in dieser Variationsgleichung darstellt. Für die Finite\sphinxhyphen{}Elemente\sphinxhyphen{}Approximation müssen (\(C^1\))\sphinxhyphen{}stetige Ansatzfunktionen formuliert werden. Dies impliziert, dass an den Knotenpunkten die Verschiebungsansätze sowohl in der Durchbiegung (\(w\)) als auch in der Neigung (\(w'\)) stetig sein müssen. Anders als beim Stabelement sind die Freiheitsgrade an den Knotenpunkten nicht nur die Verschiebungen, sondern auch die Neigungen. Dies ist charakteristisch für die Formulierung von Balkenelementen, aber auch für die Formulierung von Plattenelementen und Schalenelementen.


\subsection{Ansatzfunktionen}
\label{\detokenize{chapters/chapter2/BalkenFEM:ansatzfunktionen}}
\begin{figure}[htbp]
\centering
\capstart

\noindent\sphinxincludegraphics[height=300\sphinxpxdimen]{{Balkenelement}.png}
\caption{Balkenelement mit lokalen Freiheitsgraden}\label{\detokenize{chapters/chapter2/BalkenFEM:balken03-1}}\end{figure}

\sphinxAtStartPar
Analog zum Stabelement führen wir die normierte Koordinate \(\xi=\frac{x}{\ell_e}\) ein. Die Ansatzfunktionen für die Durchbiegung \(w_h\) wird dann als Linearkombination der Formfunktionen \(N_I\) und der Freiheitsgrade \(w_I\) geschrieben:
\begin{equation}\label{equation:chapters/chapter2/BalkenFEM:Ansatz_Balken}
\begin{split}\begin{align}
  w_h & = N_1(\xi) w_1 + N_2(\xi) \psi_1 \ell_e +  N_3(\xi) w_2 + N_4(\xi) \psi_2 \ell_e = \sum_{I=1}^{4} N_I(\xi) w_I \\
  & = \begin{pmatrix} N_1(\xi) & N_2(\xi)\ell_e & N_3(\xi) & N_4(\xi)\ell_e \end{pmatrix} \begin{pmatrix} w_1 \\ \psi_1 \\ w_2 \\ \psi_2 \end{pmatrix}
\end{align}\end{split}
\end{equation}
\sphinxAtStartPar
Der Vektor der Freiheitsgrade \(\bm{w}=w_I\) enthält die Durchbiegung \(w\) und die Neigung \(\psi\) an den beiden Knoten des Balkenelements. Wie bei den Formfunktionen des Stabelementes, ist die Formfunktion \(N_I\) nur für den Freiheitsgrad \(w_I\) identisch mit 1 und für alle anderen Freiheitsgrade gleich 0. Formfunktionen die diese Eigenschaft haben, sind zum Beispiel die Hermite\sphinxhyphen{}Polynome:
\begin{equation}\label{equation:chapters/chapter2/BalkenFEM:Formfunktion_Balken}
\begin{split}\begin{align}
  \bm{N} = \begin{pmatrix} 
  1-3\xi^2+2\xi^3 \\ 
  (\xi-2\xi^2+\xi^3)\ell_e \\ 
  3\xi^2-2\xi^3 \\ 
  (-\xi^2+\xi^3)\ell_e \end{pmatrix}
\end{align}\end{split}
\end{equation}


\begin{sphinxuseclass}{cell}
\begin{sphinxuseclass}{tag_remove-input}\begin{sphinxVerbatimOutput}

\begin{sphinxuseclass}{cell_output}
\begin{figure}[htbp]
\centering
\capstart

\noindent\sphinxincludegraphics{{701440f17342ebe2f3daa08e467b90d1d9404576f1b03a1dd899ecbb5af74619}.png}
\caption{Formfunktionen für ein Balkenelelement mit Hermite\sphinxhyphen{}Interpolation.}\label{\detokenize{chapters/chapter2/BalkenFEM:shape-beam}}\end{figure}

\end{sphinxuseclass}\end{sphinxVerbatimOutput}

\end{sphinxuseclass}
\end{sphinxuseclass}
\begin{sphinxadmonition}{note}{Wichtige Eigenschaften des Ansatzes des Balkens}
\begin{itemize}
\item {} 
\sphinxAtStartPar
Die Durchbiegung \(w\) ist \(C^1\)\sphinxhyphen{}stetig

\item {} 
\sphinxAtStartPar
Die Durchbiegung \(w\) und die Neigung \(\psi=w'\) sind über die Elementgrenzen hinweg stetig

\item {} 
\sphinxAtStartPar
Durch die Wahl der Formfunktionen haben die Knotenfreiheitsgrade die Bedeutung von Durchbiegung und Neigung

\end{itemize}
\end{sphinxadmonition}

\sphinxAtStartPar
Auch für die Testfunktion \(\delta w\) wird der Ansatz \eqref{equation:chapters/chapter2/BalkenFEM:Ansatz_Balken} verwendet.


\subsection{Elementsteifigkeitsmatrix (Diskretisierung Term 1 in Gleichung \eqref{equation:chapters/chapter2/BalkenFEM:weakBalken_01} )}
\label{\detokenize{chapters/chapter2/BalkenFEM:elementsteifigkeitsmatrix-diskretisierung-term-1-in-gleichung-weakbalken-01}}
\sphinxAtStartPar
Eingesetzt in die Schwache Form liefert der erste Term die Elementsteifigkeitsmatrix. Hierfür müssen zunächst die Ableitungen des Ansatzes bzgl. \(x\) berechnet werden. Aus
\(\dx = \ell_e \dxi\) folgt:
\begin{equation}\label{equation:chapters/chapter2/BalkenFEM:Differentialquotient}
\begin{split}\begin{align}
  \Dd{w}{x} & = \frac{1}{\ell_e} \Dd{w}{\xi} = \frac{1}{\ell_e} \bm{N}_{,\xi} \bm{w} \\
  \Dd{^2 w}{x^2} & = \frac{1}{\ell_e^2} \Dd{^2 w}{\xi^2} = \frac{1}{\ell_e^2} \bm{N}_{,\xi\xi} \bm{w}
\end{align}\end{split}
\end{equation}
\sphinxAtStartPar
Eingesetzt in die Schwache Form ergibt sich für die Elementsteifigkeit:
\begin{equation}\label{equation:chapters/chapter2/BalkenFEM:Elementsteifigkeit_Balken_01}
\begin{split}\begin{align}
  & \delta \bm{w}^T \frac{EI}{\ell^3_e} \int_0^{1}  \bm{N}_{,\xi\xi}^T \bm{N}_{,\xi\xi} \dxi\, \bm{w} \\
  &= \delta \bm{w}^T \frac{EI}{\ell^3_e} \int_0^{1} \begin{pmatrix}
    12\xi -6 & (6\xi-4)\ell_e  & -12\xi+6 & (6\xi-2)\ell_e 
  \end{pmatrix} \begin{pmatrix}
    12\xi -6 \\
    (6\xi-4)\ell_e  \\
    -12\xi+6 \\
    (6\xi-2)\ell_e 
  \end{pmatrix} \dxi\, \bm{w} \\
  &= \delta \bm{w}^T \frac{EI}{\ell^3_e} \int_0^{1} \begin{pmatrix}\left(12 \xi - 6\right)^{2} & \ell_e \left(6 \xi - 4\right) \left(12 \xi - 6\right) & \left(6 - 12 \xi\right) \left(12 \xi - 6\right) & \ell_e \left(6 \xi - 2\right) \left(12 \xi - 6\right)\\\ell_e \left(6 \xi - 4\right) \left(12 \xi - 6\right) & \ell_e^{2} \left(6 \xi - 4\right)^{2} & \ell_e \left(6 - 12 \xi\right) \left(6 \xi - 4\right) & \ell_e^{2} \left(6 \xi - 4\right) \left(6 \xi - 2\right)\\\left(6 - 12 \xi\right) \left(12 \xi - 6\right) & \ell_e \left(6 - 12 \xi\right) \left(6 \xi - 4\right) & \left(6 - 12 \xi\right)^{2} & \ell_e \left(6 - 12 \xi\right) \left(6 \xi - 2\right)\\\ell_e \left(6 \xi - 2\right) \left(12 \xi - 6\right) & \ell_e^{2} \left(6 \xi - 4\right) \left(6 \xi - 2\right) & \ell_e \left(6 - 12 \xi\right) \left(6 \xi - 2\right) & \ell_e^{2} \left(6 \xi - 2\right)^{2}\end{pmatrix} \dxi\, \bm{w} \\
  &= \delta \bm{w} \underbrace{\frac{EI}{\ell^3} \begin{pmatrix}12 & 6 \ell_e & -12 & 6 \ell_e\\6 \ell_e & 4 \ell_e^{2} & - 6 \ell_e & 2 \ell_e^{2}\\-12 & - 6 \ell_e & 12 & - 6 \ell_e\\6 \ell_e & 2 \ell_e^{2} & - 6 \ell_e & 4 \ell_e^{2}\end{pmatrix}}_{\bm{K_e}} \,  \bm{w}
\end{align}\end{split}
\end{equation}

\subsection{Diskretisierung der Streckenlast}
\label{\detokenize{chapters/chapter2/BalkenFEM:diskretisierung-der-streckenlast}}
\sphinxAtStartPar
Die Streckenlast \(q(x)\) hat im Element \(e\) im Allgemeinen einen beliebigen Verlauf. Um die Streckenlast zu diskretisieren wird angenommen, dass die Streckenlast im Element linear veränderlich ist. Es wird also der folgende Ansatz gemacht:
\begin{equation}\label{equation:chapters/chapter2/BalkenFEM:AnsatzStreckenlast}
\begin{split}\begin{align}
  q_h(\xi) & = (1-\xi) q_1 + \xi q_2 \\
  & = \sum_I N_I(\xi)  q_I
\end{align}\end{split}
\end{equation}
\sphinxAtStartPar
Wird dieser Ansatz in die schwache Form eingesetzt, so ergibt sich:
\begin{equation}\label{equation:chapters/chapter2/BalkenFEM:weak_form_streckenlast_01}
\begin{split}\begin{align}
  & \int_L \delta \bm{w}  q(x) \dx \\
  & = \ell \int_{0}^1  \delta \bm{w}(\xi) q_h(\xi)  \dxi \\
  & \delta \bm{w} \ell \int_{0}^1 \begin{pmatrix} 2 \xi^{3} - 3 \xi^{2} + 1\\\ell \left(\xi^{3} - 2 \xi^{2} + \xi\right)\\- 2 \xi^{3} + 3 \xi^{2}\\\ell \left(\xi^{3} - \xi^{2}\right)\end{pmatrix} \begin{pmatrix} q_1 & q_2  \end{pmatrix} \dxi \\
  & = \delta \bm{w} \ell \int_{0}^1 \begin{pmatrix}
  \left(q_{1} \left(1 - \xi\right) + q_{2} \xi\right) \left(2 \xi^{3} - 3 \xi^{2} + 1\right)\\\ell \left(q_{1} \left(1 - \xi\right) + q_{2} \xi\right) \left(\xi^{3} - 2 \xi^{2} + \xi\right)\\\left(- 2 \xi^{3} + 3 \xi^{2}\right) \left(q_{1} \left(1 - \xi\right) + q_{2} \xi\right)\\\ell \left(\xi^{3} - \xi^{2}\right) \left(q_{1} \left(1 - \xi\right) + q_{2} \xi\right)
  \end{pmatrix} \\
  & = \delta \bm{w} \underbrace{ \ell \begin{pmatrix}
  \frac{7 q_{1}}{20} + \frac{3 q_{2}}{20}\\\frac{\ell q_{1}}{20} + \frac{\ell q_{2}}{30}\\\frac{3 q_{1}}{20} + \frac{7 q_{2}}{20}\\- \frac{\ell q_{1}}{30} - \frac{\ell q_{2}}{20}
  \end{pmatrix}}_{\bm{F}_Q}
\end{align}\end{split}
\end{equation}

\subsection{Beispiel der Balken FEM}
\label{\detokenize{chapters/chapter2/BalkenFEM:beispiel-der-balken-fem}}
\begin{figure}[htbp]
\centering
\capstart

\noindent\sphinxincludegraphics[height=81\sphinxpxdimen]{{Balken_Problem}.png}
\caption{Statisch bestimmt gelagerter Balken mit konstanter Streckenlast}\label{\detokenize{chapters/chapter2/BalkenFEM:balken03}}\end{figure}

\sphinxAtStartPar
Für das dargestellte Beispiel eines statisch bestimmten Balkens mit konstanter Streckenlast lässt sich die Lösung der Verschiebung durch vierfache Integration der Differentialgleichung \eqref{equation:chapters/chapter2/BalkenFEM:balkendglsimple_2} bestimmen. Die Lösung lautet:
\begin{equation}\label{equation:chapters/chapter2/BalkenFEM:Balken_example_01}
\begin{split}EI w(x) =\frac{\ell^{3} q_{0} x}{24} - \frac{\ell q_{0} x^{3}}{12} + \frac{q_{0} x^{4}}{24}\end{split}
\end{equation}
\sphinxAtStartPar
Über die Beziehungen:
\begin{equation}\label{equation:chapters/chapter2/BalkenFEM:Balken_example_02}
\begin{split}\begin{align}
EI w'' & =-M(x) =  \frac{q_{0} x \left(\ell - x\right)}{2} \\
EI w'''' &= -Q(x) = q_{0} \left(\frac{\ell}{2} - x\right)
\end{align}\end{split}
\end{equation}
\sphinxAtStartPar
erhält man die Schnittgrößen für das Biegemoment und die Querkraft.

\begin{sphinxuseclass}{cell}
\begin{sphinxuseclass}{tag_hide-input}\begin{sphinxVerbatimOutput}

\begin{sphinxuseclass}{cell_output}
\begin{figure}[htbp]
\centering
\capstart

\noindent\sphinxincludegraphics{{4305a0b746952cdedca7a8dacaa53d1c353ba141407a8f8f1da0b118142d8982}.png}
\caption{Verlauf der Durchbiegung und des}\label{\detokenize{chapters/chapter2/BalkenFEM:schnittgroszen-balken}}\end{figure}

\end{sphinxuseclass}\end{sphinxVerbatimOutput}

\end{sphinxuseclass}
\end{sphinxuseclass}
\sphinxAtStartPar
Wir lösen nun dieses Problem mit der obigen Balken\sphinxhyphen{}FEM und vergleichen die Ergebnisse für die Durchbiegung und die Schnittgrößen mit der analytischen Lösung.

\begin{figure}[htbp]
\centering
\capstart

\noindent\sphinxincludegraphics[height=600\sphinxpxdimen]{{Flächenträgheitsmomente}.png}
\caption{Flächenträgheitsmomente für I\sphinxhyphen{}Profile nach DIN 1025\sphinxhyphen{}1:2009\sphinxhyphen{}04. \sphinxhref{https://www.ezzat.org/de/Querschnittswerte/gewalzt/I/i.php}{ezzart.org}}\label{\detokenize{chapters/chapter2/BalkenFEM:ft}}\end{figure}


\subsection{Ergebnisse der FEM Berechnung}
\label{\detokenize{chapters/chapter2/BalkenFEM:ergebnisse-der-fem-berechnung}}
\begin{figure}[htbp]
\centering
\capstart

\noindent\sphinxincludegraphics[width=500\sphinxpxdimen]{{Balken_FEM_01}.png}
\caption{Gegenüberstellung der analytischen Verschiebungslösung mit der FEM Lösung.}\label{\detokenize{chapters/chapter2/BalkenFEM:balkenfem01}}\end{figure}

\sphinxAtStartPar
In Abbildung \hyperref[\detokenize{chapters/chapter2/BalkenFEM:balkenfem01}]{Abb.\@ \ref{\detokenize{chapters/chapter2/BalkenFEM:balkenfem01}}} ist die Verschiebungslösung der FEM mit der analytischen Lösung gegenüber gestellt. Wir sehen, dass die FEM bereits mit nur 2 Elementen die analytische Lösung exakt trifft. Dies ist kein Zufall, sondern ein Ergebnis der Wahl der Formfunktionen und der Form der Randbedingungen.

\sphinxAtStartPar
Dies sollte jedoch nicht darüber hinwegtäuschen, dass die FEM\sphinxhyphen{}Lösung für das Schnittmoment und die Querkraft im Vergleich zur analytischen Lösung nicht so gut ist.

\begin{figure}[htbp]
\centering
\capstart

\noindent\sphinxincludegraphics[width=500\sphinxpxdimen]{{Balken_FEM_02}.png}
\caption{Gegenüberstellung der analytischen Lösung des Momentenverlaufs mit der FEM Lösung.}\label{\detokenize{chapters/chapter2/BalkenFEM:balkenfem02}}\end{figure}

\sphinxAtStartPar
Das analytische Schnittmoment hat einen quadratischen Verlauf, während die FEM Lösung linear ist. Dies führt zu einer Abweichung der FEM Lösung von der analytischen Lösung.

\begin{figure}[htbp]
\centering
\capstart

\noindent\sphinxincludegraphics[width=500\sphinxpxdimen]{{Balken_FEM_03}.png}
\caption{Gegenüberstellung der analytischen Lösung für den Querkraftverlauf mit der FEM Lösung.}\label{\detokenize{chapters/chapter2/BalkenFEM:balkenfem03}}\end{figure}

\sphinxAtStartPar
Die Schnittkraft in der analytischen Lösung zeigt einen stückweise linearen Verlauf, während die FEM Lösung konstant ist. Dies führt zu einer Abweichung der FEM Lösung von der analytischen Lösung.


\subsection{Konvergenz}
\label{\detokenize{chapters/chapter2/BalkenFEM:konvergenz}}
\begin{figure}[htbp]
\centering
\capstart

\noindent\sphinxincludegraphics[width=600\sphinxpxdimen]{{Balken_FEM_04}.png}
\caption{Konvergenzverhalten der FEM Lösung für die das Schnittmoment und die Querkraft.}\label{\detokenize{chapters/chapter2/BalkenFEM:balkenfem04}}\end{figure}

\sphinxAtStartPar
Allgemein zeigt sich, dass die FEM Lösung mit zunehmender Anzahl an Elementen gegen die analytische Lösung konvergiert. Dabei ist festzuhalten, dass die Konvergenz des Schnittmomentes schneller ist als die Konvergenz der Querkraft. Dies zeigt sich auch im doppelt logarithmischen Diagramm \hyperref[\detokenize{chapters/chapter2/BalkenFEM:balkenfem05}]{Abb.\@ \ref{\detokenize{chapters/chapter2/BalkenFEM:balkenfem05}}}. Beide Kurven zeigen eine lineare Abhängigkeit, jedoch ist die Steigung der Konvergenzkurve für das Schnittmoment größer als die Steigung der Konvergenzkurve für die Querkraft.

\begin{figure}[htbp]
\centering
\capstart

\noindent\sphinxincludegraphics[width=600\sphinxpxdimen]{{Balken_FEM_05}.png}
\caption{Konvergenzverhalten der FEM Lösung für die das Schnittmoment und die Querkraft im doppelt logarithmischen Diagramm.}\label{\detokenize{chapters/chapter2/BalkenFEM:balkenfem05}}\end{figure}


\subsection{Interaktives Notebook}
\label{\detokenize{chapters/chapter2/BalkenFEM:interaktives-notebook}}
\sphinxAtStartPar
Basierend auf der präsentierten Theorie wurde ein Jupyter Notebook erstellt. Dieses kann man ohne Systemvorraussetzungen im Browser ausführen:

\sphinxAtStartPar
Balken FEM for Binder \sphinxhref{https://mybinder.org/v2/gh/steffenbeese/FEM\_I\_Notebooks/main?urlpath=\%2Fdoc\%2Ftree\%2FNotebook\_BalkenFEM.ipynb}{\sphinxincludegraphics{{badge_logo}.png}}

\sphinxstepscope


\part{Isoparametrische FEM}

\sphinxstepscope


\chapter{Isoparametrische FEM}
\label{\detokenize{chapters/chapter3/isoparametrischeFEM:isoparametrische-fem}}\label{\detokenize{chapters/chapter3/isoparametrischeFEM::doc}}
\begin{figure}[htbp]
\centering
\capstart

\noindent\sphinxincludegraphics[width=600\sphinxpxdimen]{{Nackenhorst_01}.png}
\caption{Geometrieapproximation am Beispiel der Diskretisierung eines Eisenbahnrades {[}\hyperlink{cite.quellen:id17}{Nackenhorst, 2022}{]}.}\label{\detokenize{chapters/chapter3/isoparametrischeFEM:isogeom}}\end{figure}

\sphinxAtStartPar
Die isoparametrische Finite Elemente Methode ist heute der geläufige Standard in FEA Software. Erstmals erwähnt wurde die isoparameterische Finite Elemente Methode von I.C. Taig (britischer Luftfahrtingenieur) im Jahre 1958. Die Methode bildet die Grundlage um komplizierte Geometrien mit gekrümmten Rändern zu modellieren. Als Beispiel sei hier die Diskretisierung in Abbildung \hyperref[\detokenize{chapters/chapter3/isoparametrischeFEM:isogeom}]{Abb.\@ \ref{\detokenize{chapters/chapter3/isoparametrischeFEM:isogeom}}} eines Eisenbahnrades gezeigt. In den nächsten Abschnitten werden einige Aspekte der isoparametrischen Finite Elemente Methode erläutert. Für einen umfassender Einblick sei auf die Standardwerke von {[}\hyperlink{cite.quellen:id3}{Zienkiewicz \sphinxstyleemphasis{et al.}, 2005}{]}, {[}\hyperlink{cite.quellen:id18}{Hughes, 2003}{]} oder  {[}\hyperlink{cite.quellen:id19}{Bathe, 2006}{]} verwiesen.


\section{Begriffsklärung}
\label{\detokenize{chapters/chapter3/isoparametrischeFEM:begriffsklarung}}
\sphinxAtStartPar
In der schwachen Form \eqref{equation:chapters/chapter2/FEM:weakform_03} werden die Felder \(\bm{u}_h\), \(\delta\bm{u}_h\) und die Koordinaten \(\bm{x}\), sowie deren räumliche Ableitungen berechnet. Das Aufstellen der Ansatzfunktionen für diese Felder ist für komplexe Geometrien nicht trivial. Daher werden die Ansatzfunktionen in einem sog. Elternelement definiert. Dieses wird dann auf die reale Geometrie abgebildet. Diese Abbildung wird als isoparametrische Abbildung bezeichnet. Die Abbildung wird durch die sog. Formfunktionen \(\bm{N}\) beschrieben. Diese Formfunktionen sind in der Regel Polynome und werden in der Regel in einem sog. Elternelement definiert. Dieses Elternelement ist ein einfaches geometrisches Objekt, wie z.B. ein Einheitsquadrat oder ein Einheitsdreieck. Die Formfunktionen sind in diesem Elternelement definiert und werden dann auf die reale Geometrie abgebildet. Dies ist in Abbildung \hyperref[\detokenize{chapters/chapter3/isoparametrischeFEM:isoparamtericmapping}]{Abb.\@ \ref{\detokenize{chapters/chapter3/isoparametrischeFEM:isoparamtericmapping}}} für den zweidimensionalen Fall eines Rechteckelementes dargestellt.

\begin{figure}[htbp]
\centering
\capstart

\noindent\sphinxincludegraphics[width=600\sphinxpxdimen]{{Taylor_isoQuad}.png}
\caption{Abbildung des Elternelementes auf die reale Geometrie ({[}\hyperlink{cite.quellen:id3}{Zienkiewicz \sphinxstyleemphasis{et al.}, 2005}{]}).}\label{\detokenize{chapters/chapter3/isoparametrischeFEM:isoparamtericmapping}}\end{figure}

\sphinxAtStartPar
Die Abbildung des Elternelementes auf die reale Geometrie entspricht einer Koordinaten\sphinxhyphen{}Transformation. Das Elternelement wird mit den Koordinaten \(\xi\), \(\eta\) und \(\zeta\) beschrieben, während die reale Geometrie mit den Koordinaten \(x\), \(y\) und \(z\) beschrieben wird. Die Transformation wird durch die Abbildung \(\bm{x}_h = \bm{x}_h(\xi, \eta, \zeta)\) beschrieben:
\begin{equation}\label{equation:chapters/chapter3/isoparametrischeFEM:geometric_mapping}
\begin{split}\begin{equation}
\bm{x}_h = \sum_{i=1}^{n} N_I(\bm{\xi}) \hat{\bm{x}}_I
\end{equation}\end{split}
\end{equation}
\sphinxAtStartPar
hierbei ist \(\hat{\bm{x}}_I\) der Vektor der Knotenkoordinaten und \(N_I(\bm{\xi})\) die Formfunktionen des Elternelementes. Die Formfunktionen sind in der Regel Lagrange\sphinxhyphen{}Polynome, die die Knoten des Elternelementes interpolieren. Die Formfunktionen sind somit stückweise linear, quadratisch oder kubisch, je nach Art des Elternelementes.

\sphinxAtStartPar
Wird für die Testfunktion \(\delta \bm{u}_h\) und die Verschiebung \(\delta\bm{u}_h\), sowie die diskrete Darstellung der Geometrie \(\bm{x}_h\) der gleiche Ansatz verwendet, so spricht man von einem isoparametrischen Element:
\begin{equation}\label{equation:chapters/chapter3/isoparametrischeFEM:isoparametric}
\begin{split}\begin{align}
  \delta \bm{u}_h &= \sum_{I=1}^{n} N_I(\bm{\xi}) \delta \hat{\bm{u}}_I \\
  \bm{u}_h &= \sum_{i=1}^{n} N_I(\bm{\xi}) \hat{\bm{u}}_I \\
  \bm{x}_h &= \sum_{i=1}^{n} N_I(\bm{\xi}) \hat{\bm{x}}_I
\end{align}\end{split}
\end{equation}
\begin{sphinxadmonition}{note}{Abgrenzung: weitere Parameterformulierungen}

\sphinxAtStartPar
Neben der isoparametrischen Formulierung gibt es auch andere Parameterformulierungen:
\begin{itemize}
\item {} 
\sphinxAtStartPar
subparametrische Formulierung: Geometrie wird mit niedrigerem Polynomgrad als die Verschiebung interpoliert

\item {} 
\sphinxAtStartPar
superparametrische Formulierung: Geometrie wird mit höherem Polynomgrad als die Verschiebung interpoliert
Beide Konzepte haben sich in der Praxis nicht durchgesetzt, wobei die superparametrische generell vermieden werden sollte (Keine Konvergenz sichergestellt)

\end{itemize}
\end{sphinxadmonition}


\section{Berechnung der räumlichen Ableitungen}
\label{\detokenize{chapters/chapter3/isoparametrischeFEM:berechnung-der-raumlichen-ableitungen}}
\sphinxAtStartPar
Möchte man ein diskretisiertes Feld z.B. \(\bm{u}_h\) ableiten, so muss man die über die Kettenregel die Formfunktionen ableiten:
\begin{equation}\label{equation:chapters/chapter3/isoparametrischeFEM:ableitungFormfunktion}
\begin{split}\begin{align}
\Pd{\bm{u}_h}{\bm{x}} = \sum_{I=1}^{n} \Pd{N_I}{\bm{\xi}} \Pd{\bm{\xi}}{\bm{x}}  \hat{\bm{u}}_I \;.
\end{align}\end{split}
\end{equation}
\sphinxAtStartPar
In dieser Gleichung ist \(\Pd{\bm{\xi}}{\bm{x}}\) jedoch unbekannt. Über die parametrische Darstellung der Geometrie \eqref{equation:chapters/chapter3/isoparametrischeFEM:geometric_mapping} kann man jedoch die Ableitung \(\Pd{\bm{\xi}}{\bm{x}}\) berechnen:
\begin{equation}\label{equation:chapters/chapter3/isoparametrischeFEM:ableitungFormfunktion_02}
\begin{split}\begin{align}
\bm{J} := \Pd{\bm{x}}{\bm{\xi}} & = \begin{pmatrix}
\Pd{x}{\xi} & \Pd{x}{\eta} & \Pd{x}{\zeta} \\
\Pd{y}{\xi} & \Pd{y}{\eta} & \Pd{y}{\zeta} \\
\Pd{z}{\xi} & \Pd{z}{\eta} & \Pd{z}{\zeta} 
\end{pmatrix}
=  \sum_{I=1}^{n} \Pd{N_I}{\bm{\xi}} \hat{\bm{x}}_I \\
\Rightarrow \Pd{\bm{\xi}}{\bm{x}} & = \bm{J}^{-1} \; .
\end{align}\end{split}
\end{equation}
\sphinxAtStartPar
Hier wurde die \sphinxstylestrong{Jacobi\sphinxhyphen{}Matrix} \(\bm{J}\) eingeführt. Zur Berechnung der räumlichen Ableitungen der Formfunktionen werden diese einfach mit der inversen Jacobi\sphinxhyphen{}Matrix multipliziert:
\begin{equation}\label{equation:chapters/chapter3/isoparametrischeFEM:ableitungFormfunktion_03}
\begin{split}\Pd{N_I}{\bm{x}} = \Pd{N_I}{\bm{\xi}} \Pd{\bm{\xi}}{\bm{x}} = \Pd{N_I}{\bm{\xi}} \bm{J}^{-1} \; .\end{split}
\end{equation}
\begin{sphinxadmonition}{note}{Bedeutung der Jacobi Matrix}

\sphinxAtStartPar
Die Determinante der Jacobi\sphinxhyphen{}Matrix \(\det(\bm{J})\) wird als \sphinxstylestrong{Jacobian} bezeichnet und stellt ein wichtiges Maß für die Qualität der Elementform dar. Eine Jacobi\sphinxhyphen{}Determinante von Null bedeutet, dass die Elementform singulär ist und die Ableitung der Formfunktionen nicht mehr eindeutig definiert ist. Es ist daher wichtig, dass die Jacobi\sphinxhyphen{}Determinante in der gesamten Element positiv ist. Eine Jacobi\sphinxhyphen{}Determinante in der Nähe von 1 ist ideal, während Werte weit von 1 entfernt auf verzerrte Elemente hinweisen. Negative Werte der Jacobi\sphinxhyphen{}Determinante sind ebenfalls problematisch und sollten vermieden werden. Hier hat sich das Element selbst durchdrungen. Solche Elemente fügen dem System Energie hinzu und verfälschen die Lösung maßgeblich.
\end{sphinxadmonition}


\section{Anforderungen an den Finite Elemente Ansatz}
\label{\detokenize{chapters/chapter3/isoparametrischeFEM:anforderungen-an-den-finite-elemente-ansatz}}
\sphinxAtStartPar
Um die Finite\sphinxhyphen{}Elemente\sphinxhyphen{}Methode (FEM) effektiv anzuwenden, müssen die Formfunktionen bestimmte Anforderungen erfüllen:

\sphinxAtStartPar
\sphinxstylestrong{Kontinuität}:

\sphinxAtStartPar
Die Formfunktionen müssen im gesamten Bereich stetig vom Grad \(C^{n-1}\) sein, wobei \(n\) die höchste Ableitung ist, die in der schwachen Formulierung \eqref{equation:chapters/chapter2/FEM:weakform_03} vorkommt. Beispielsweise erfordert die Elastizitätstheorie \(n=1\), was bedeutet, dass der Ansatz \(C^0\)\sphinxhyphen{}stetig sein muss. Insbesondere muss diese Stetigkeit über die Elementgrenzen hinweg gewährleistet sein.
Beim Balkenelement trat die Krümmung (2. Ableitung der Durchbiegung) in der schwachen Form auf. Deshalb muss der Ansatz für das Balkenelement \(C^1\)\sphinxhyphen{}stetig sein.

\sphinxAtStartPar
\sphinxstylestrong{Vollständigkeit}:

\sphinxAtStartPar
Die Formfunktionen müssen die exakte Lösung approximieren können, wenn die Anzahl der Elemente gegen unendlich geht. Dies stellt sicher, dass die numerische Lösung mit zunehmender Verfeinerung des Netzes immer genauer wird.

\sphinxAtStartPar
\sphinxstylestrong{Eindeutigkeit}:

\sphinxAtStartPar
Die Formfunktionen müssen eindeutig sein, d.h., es darf keine lineare Abhängigkeit zwischen den Formfunktionen geben. Diese Eigenschaft ist entscheidend, um sicherzustellen, dass die Lösung der Gleichungen eindeutig bestimmt ist.

\sphinxAtStartPar
Diese Anforderungen sind essenziell, um die Genauigkeit und Zuverlässigkeit der FEM\sphinxhyphen{}Lösungen zu gewährleisten. Sie stellen sicher, dass die numerischen Ergebnisse physikalisch sinnvoll und mathematisch korrekt sind und dass die Methode gegen die analytische Lösung konvergiert.

\sphinxAtStartPar
Für diese drei Anforderungen existiert auch eine Ingenieurmäßige Deutung, welche im Folgenden beschrieben wird.

\begin{figure}[htbp]
\centering
\capstart

\noindent\sphinxincludegraphics[width=600\sphinxpxdimen]{{NonConforming_Mesh_01}.png}
\caption{Verletzung der Stetigkeitsanforderung in der Finiten Elemente Methode.}\label{\detokenize{chapters/chapter3/isoparametrischeFEM:stetigkeit-01}}\end{figure}

\sphinxAtStartPar
Die Forderung der Stetigkeit über die Elementgrenzen hinweg ist wird über eine konforme Vernetzung sichergestellt. Ein typisches Beispiel zur Veranschaulichung dieser Anforderung ist die Elementierung mit 4\sphinxhyphen{}Knoten\sphinxhyphen{}Rechteckelementen.
Betrachten wir eine eingespannte Scheibe, die einer Zugbelastung ausgesetzt ist. Um die Schnittkräfte genauer zu erfassen, wird das Netz in Richtung der Einspannung verfeinert. Dies soll die Behinderung der Querkontraktion besser erfassen. Allerdings führt diese Netzverfeinerung in einigen Fällen zu unzulässigen Klaffungen im deformierten Netz.

\sphinxAtStartPar
Die Ursache für diese unzulässigen Klaffungen liegt in der Zuordnung der Knoten zu den Elementen. In diesem Beispiel werden die Knoten 7,9 und 12 nur zu zwei Elementen zugeordnet. Die Elementkante von Element 5, 6 und 7 wird nicht an die Deformation der genannten Knoten gekoppelt. Diese fehlerhafte Zuordnung führt dazu, dass die Schnittkräfte in der Nähe von Inkompatibilitäten nicht korrekt berechnet werden.

\begin{figure}[htbp]
\centering
\capstart

\noindent\sphinxincludegraphics[width=600\sphinxpxdimen]{{NonConforming_Mesh_02}.png}
\caption{Verschiebungsfeld bei einer nicht\sphinxhyphen{}konformen Netzstruktur.}\label{\detokenize{chapters/chapter3/isoparametrischeFEM:stetigkeit-02}}\end{figure}

\sphinxAtStartPar
Diese lokalen Störungen klingen jedoch mit ausreichend großem Abstand ab, wie es das Prinzip von St. Venant beschreibt. Dieses Prinzip besagt, dass die Auswirkungen von lokalen Unregelmäßigkeiten in der Belastung oder Geometrie mit zunehmendem Abstand von der Störungsquelle abnehmen. Daher sind die Auswirkungen der Inkompatibilitäten auf die Gesamtlösung begrenzt.

\sphinxAtStartPar
Ein weiterer Fehler, der bei der Vernetzung entstehen kann, ist die Verwendung von Elementen mit unterschiedlicher Elementordnung.

\begin{figure}[htbp]
\centering
\capstart

\noindent\sphinxincludegraphics[width=400\sphinxpxdimen]{{NonConforming_Mesh_03}.png}
\caption{Verletzung der Stetigkeitsanforderung in der Finiten Elemente Methode durch Verwendung von Elementen mit unterschiedlicher Ordnung.}\label{\detokenize{chapters/chapter3/isoparametrischeFEM:stetigkeit-03}}\end{figure}

\sphinxAtStartPar
Auch hier ist die resultierende Verschiebung nicht korrekt. Eigentlich sollte die Verschiebung linear sein, was jedoch nicht der Fall ist.

\begin{figure}[htbp]
\centering
\capstart

\noindent\sphinxincludegraphics[width=600\sphinxpxdimen]{{NonConforming_Mesh_04}.png}
\caption{Verletzung der Stetigkeitsanforderung in der Finiten Elemente Methode durch Verwendung von Elementen mit unterschiedlicher Ordnung \sphinxhyphen{} Resultierendes Verschiebungsfeld.}\label{\detokenize{chapters/chapter3/isoparametrischeFEM:stetigkeit-04}}\end{figure}

\sphinxAtStartPar
Die Stetigkeitsanforderung ist von großer praktischer Bedeutung, da sie sicherstellt, dass die numerische Lösung physikalisch sinnvoll und mathematisch korrekt ist. Eine konforme Vernetzung, bei der die Knoten korrekt zugeordnet sind, ist entscheidend, um die Genauigkeit der FEM\sphinxhyphen{}Lösungen zu gewährleisten.

\sphinxAtStartPar
Um eine lokale Verfeinerung zu erzielen kann man entweder spezielle Elemente verwenden (bei denen die entsprechenden Knoten angelegt wurden) oder man fügt eine entsprechende Zwischenschicht wie in Abbildung \hyperref[\detokenize{chapters/chapter3/isoparametrischeFEM:stetigkeit-05}]{Abb.\@ \ref{\detokenize{chapters/chapter3/isoparametrischeFEM:stetigkeit-05}}} ein.

\begin{figure}[htbp]
\centering
\capstart

\noindent\sphinxincludegraphics[width=600\sphinxpxdimen]{{Conforming_Mesh_01}.png}
\caption{Übergangselementschicht zur lokalen Verfeinerung.}\label{\detokenize{chapters/chapter3/isoparametrischeFEM:stetigkeit-05}}\end{figure}

\sphinxAtStartPar
Hierbei wird das Verschiebungsfeld durch die Zwischenschicht korrekt wiedergegeben.

\begin{figure}[htbp]
\centering
\capstart

\noindent\sphinxincludegraphics[width=600\sphinxpxdimen]{{Conforming_Mesh_02}.png}
\caption{Übergangselementschicht zur lokalen Verfeinerung \sphinxhyphen{} Verschiebungsfeld.}\label{\detokenize{chapters/chapter3/isoparametrischeFEM:stetigkeit-06}}\end{figure}

\sphinxAtStartPar
Ein zentraler Aspekt der Vollständigkeit in der Finite\sphinxhyphen{}Elemente\sphinxhyphen{}Methode ist die Fähigkeit, Starrkörperverschiebungen darzustellen. Die Starrkörperverschiebung \(\bm{u}_c = \bm{c}\) muss durch die Ansatzfunktionen \(\bm{N}_I(\bm{\xi})\) exakt wiedergegeben werden können. Der mathematische Ausdruck dafür ist:
\begin{equation}\label{equation:chapters/chapter3/isoparametrischeFEM:completeness}
\begin{split}\begin{align}
\bm{u}_c & = \sum_{I=1}^{n} N_I(\bm{\xi}) \hat{\bm{u}}_I  = \sum_{I=1}^{n} N_I(\bm{\xi}) {\bm{c}} = \bm{c} \sum_{I=1}^{n} N_I(\bm{\xi}) \\
\Rightarrow \qquad 1 &= \sum_{I=1}^{n} N_I(\bm{\xi}) 
\end{align}\end{split}
\end{equation}
\sphinxAtStartPar
Diese Eigenschaft wird in der Literatur als \sphinxstylestrong{Partitions of Unity} bezeichnet. Die praktische Relevanz dieser Eigenschaft liegt darin, dass in Tragstrukturen die Lösung in einen Anteil der Starrkörperverschiebung und einen Anteil, der Dehnungen hervorruft, aufgeteilt werden kann. Ein Beispiel hierfür ist ein Balken: Am freien Ende ist die Verschiebung am größten, der Balken jedoch erfährt hier am wenigsten Dehnung. Eine praktische Überprüfung besteht darin, dass Starrkörperverschiebungen keine Dehnungen/Spannungen hervorrufen dürfen.

\sphinxAtStartPar
Ein weiteres Beispiel zur Überprüfung der Vollständigkeit ist die Darstellbarkeit konstanter Dehnungszustände (Patch\sphinxhyphen{}Test). Ist ein Element in der Lage, einen konstanten Dehnungszustand darzustellen, so ist auch ein praktischer Nachweis der Konvergenz der Elementformulierung bei Verfeinerung der Vernetzung gegeben. Dies kann man sich einfach am Beispiel in Abbildung \hyperref[\detokenize{chapters/chapter3/isoparametrischeFEM:konvergenzpatchtest}]{Abb.\@ \ref{\detokenize{chapters/chapter3/isoparametrischeFEM:konvergenzpatchtest}}} verdeutlichen. Hier werden die Schnittkräfte in einen konstanten Anteil und einen linear veränderlichen unterteilt. Mit steigender Elementanzahl wird der konstante Term der Schnittkraft dominant und ist somit der maßgebliche Anteil für die Konvergenz.

\begin{figure}[htbp]
\centering
\capstart

\noindent\sphinxincludegraphics[width=600\sphinxpxdimen]{{Knothe_02}.png}
\caption{Mit steigender Elementanzahl wird der konstante Term der Schnittkraft dominant ({[}\hyperlink{cite.quellen:id4}{Knothe and Wessels, 1991}{]}).
``}\label{\detokenize{chapters/chapter3/isoparametrischeFEM:konvergenzpatchtest}}
\begin{sphinxlegend}\end{sphinxlegend}
\end{figure}

\sphinxstepscope


\section{Ansatzfunktionen}
\label{\detokenize{chapters/chapter3/Ansatzfunktionen:ansatzfunktionen}}\label{\detokenize{chapters/chapter3/Ansatzfunktionen::doc}}
\sphinxAtStartPar
Um verzerrungsfreie Starrkörperbewegungen sowie konstante Verzerrungszustände beschreiben zu können, und dabei ein räumlich isotropes Verformungsverhalten zu approximieren, müssen die Ansatzfunktionen vollständig sein.
Für ein ebenes Dreieckselement bedeutet dies z.B.
\begin{equation}\label{equation:chapters/chapter3/Ansatzfunktionen:ansatzfunktionenCompleteTriangle}
\begin{split}\begin{align}
u(\xi,\eta) &= \underbrace{\underbrace{ a_1 + a_2 \xi + a_3 \eta}_{\text{3 Knoten Dreieck}} + a_4 \xi^2 + a_5 \eta^2 + a_6 \xi \eta}_{\text{6 Knoten Dreieck}} \; .
\end{align}\end{split}
\end{equation}
\sphinxAtStartPar
Hier sind für ein 3\sphinxhyphen{}Knoten\sphinxhyphen{}Dreieck Element alle linearen Terme enthalten, für ein 6\sphinxhyphen{}Knoten\sphinxhyphen{}Dreieck Element sind auch die quadratischen Terme und bilinearen Terme enthalten.
Illustriert kann man sich am besten ein pascalsches Dreieck wie in Abbildung \hyperref[\detokenize{chapters/chapter3/Ansatzfunktionen:pascaltriangle}]{Abb.\@ \ref{\detokenize{chapters/chapter3/Ansatzfunktionen:pascaltriangle}}} vorstellen.

\begin{figure}[htbp]
\centering
\capstart

\noindent\sphinxincludegraphics[width=0.450\linewidth]{{Taylor_PascalTriangle}.png}
\caption{Pascal’sches Dreieck für Dreieckselemte ({[}\hyperlink{cite.quellen:id3}{Zienkiewicz \sphinxstyleemphasis{et al.}, 2005}{]})}\label{\detokenize{chapters/chapter3/Ansatzfunktionen:pascaltriangle}}\end{figure}

\sphinxAtStartPar
Analog kann dazu man ein vollständiges Viereckselement definieren:
\begin{equation}\label{equation:chapters/chapter3/Ansatzfunktionen:ansatzfunktionenCompleteQuad}
\begin{split}\begin{align}
u(\xi,\eta) &= \underbrace{\underbrace{ a_1 + a_2 \xi + a_3 \eta + a_4 \xi \eta}_{\text{4 Knoten Viereck}} + a_5 \xi^2 + a_6 \eta^2 + a_7 \xi^2 \eta + a_8 \xi \eta^2}_{\text{8 Knoten Viereck}} \; .
\end{align}\end{split}
\end{equation}
\sphinxAtStartPar
Auch hier liefert das Pascalsche Dreieck eine anschauliche Interpretation:

\begin{figure}[htbp]
\centering
\capstart

\noindent\sphinxincludegraphics[width=0.450\linewidth]{{Taylor_PascalQuad}.png}
\caption{Pascalsches Dreieck für Rechteckelemente ({[}\hyperlink{cite.quellen:id3}{Zienkiewicz \sphinxstyleemphasis{et al.}, 2005}{]}). Grau unterlegt ist ein kubisches Rechteckelement.}\label{\detokenize{chapters/chapter3/Ansatzfunktionen:pascalquad}}\end{figure}


\subsection{Formfunktionen für \protect\(C^0\protect\)\sphinxhyphen{}Elemente}
\label{\detokenize{chapters/chapter3/Ansatzfunktionen:formfunktionen-fur-c-0-elemente}}
\sphinxAtStartPar
Nachfolgend werden kurz die Ansatzfunktionen für 1D, 2D und 3D Elemente mit Lagrange\sphinxhyphen{}Polynomen vorgestellt:


\subsubsection{1D Formfunktionen}
\label{\detokenize{chapters/chapter3/Ansatzfunktionen:d-formfunktionen}}
\begin{figure}[htbp]
\centering
\capstart

\noindent\sphinxincludegraphics[width=500\sphinxpxdimen]{{Line-Elements}.png}
\caption{1D\sphinxhyphen{}Elemente mit linearer und quadratischer Ansatzfunktion}\label{\detokenize{chapters/chapter3/Ansatzfunktionen:linearelements}}\end{figure}

\sphinxAtStartPar
Für den linearen Fall haben 1D\sphinxhyphen{}Elemente 2 Knoten und damit 2 Formfunktionen:
\begin{equation}\label{equation:chapters/chapter3/Ansatzfunktionen:1D_Formfunktionen}
\begin{split}\begin{align}
N_1 &= \frac{1}{2} (1 - \xi) & N_2 &= \frac{1}{2} (1 + \xi)
\end{align}\end{split}
\end{equation}
\sphinxAtStartPar
Für den quadratischen Fall haben 1D\sphinxhyphen{}Elemente 3 Knoten und damit 3 Formfunktionen:
\begin{equation}\label{equation:chapters/chapter3/Ansatzfunktionen:1D_Formfunktionen_quad}
\begin{split}\begin{align}
N_1 &= \frac{1}{2} \xi (\xi - 1) & N_2 &= 1 - \xi^2 & N_3 &= \frac{1}{2} \xi (\xi + 1)
\end{align}\end{split}
\end{equation}


\begin{sphinxuseclass}{cell}
\begin{sphinxuseclass}{tag_hide-input}\begin{sphinxVerbatimOutput}

\begin{sphinxuseclass}{cell_output}
\begin{figure}[htbp]
\centering
\capstart

\noindent\sphinxincludegraphics{{addf2972060c3f0b184f047bafaae2ad2817d51480b703f58df7efe416ad9e3b}.png}
\caption{Formfunktionen für 1D\sphinxhyphen{}Elemente}\label{\detokenize{chapters/chapter3/Ansatzfunktionen:shapefunctions-1dd}}\end{figure}

\end{sphinxuseclass}\end{sphinxVerbatimOutput}

\end{sphinxuseclass}
\end{sphinxuseclass}

\subsubsection{2D Formfunktionen}
\label{\detokenize{chapters/chapter3/Ansatzfunktionen:id3}}
\begin{figure}[htbp]
\centering
\capstart

\noindent\sphinxincludegraphics[width=500\sphinxpxdimen]{{QuadElements}.png}
\caption{2D\sphinxhyphen{}Rechteck\sphinxhyphen{}Elemente mit linearer und quadratischer Ansatzfunktion}\label{\detokenize{chapters/chapter3/Ansatzfunktionen:quadelements}}\end{figure}

\sphinxAtStartPar
Im 2D\sphinxhyphen{}Fall werden die Formfunktionen für Rechteck\sphinxhyphen{}Elemente als Tensorprodukt der 1D\sphinxhyphen{}Formfunktionen definiert. Für ein 2D\sphinxhyphen{}Rechteck\sphinxhyphen{}Element mit 4 Knoten (lineares Element) sind die Formfunktionen:
\begin{equation}\label{equation:chapters/chapter3/Ansatzfunktionen:2D_Formfunktionen_Q4}
\begin{split}\begin{align}
N_1 &= \frac{1}{4} (1 - \xi)(1 - \eta) & N_2 &= \frac{1}{4} (1 + \xi)(1 - \eta) \\
N_3 &= \frac{1}{4} (1 + \xi)(1 + \eta) & N_4 &= \frac{1}{4} (1 - \xi)(1 + \eta)
\end{align}\end{split}
\end{equation}
\begin{figure}[htbp]
\centering
\capstart

\noindent\sphinxincludegraphics[width=500\sphinxpxdimen]{{shape_functions_Q4}.png}
\caption{Bilineare Formfunktionen für 2D\sphinxhyphen{}Rechteck\sphinxhyphen{}Elemente.}\label{\detokenize{chapters/chapter3/Ansatzfunktionen:quadelementsq4n}}\end{figure}

\sphinxAtStartPar
Um ein vollständiges Element der Ordnung 2 zu erhalten benötigt man 9 Knoten beim Rechteck\sphinxhyphen{}Element. Die entsprechenden Formfunktionen lauten:
\begin{equation}\label{equation:chapters/chapter3/Ansatzfunktionen:2D_Formfunktionen_Q9}
\begin{split}\begin{align*}
N_1 &= \frac{1}{4} \xi \eta (\xi - 1)(\eta - 1) & N_2 &= \frac{1}{2} (1 - \xi^2) (1 - \eta) \\
N_3 &= \frac{1}{4} \xi \eta (\xi + 1)(\eta - 1) & N_4 &= \frac{1}{2} (1 + \xi) (1 - \eta^2) \\
N_5 &= \frac{1}{4} \xi \eta (\xi + 1)(\eta + 1) & N_6 &= \frac{1}{2} (1 - \xi^2) (1 + \eta) \\
N_7 &= \frac{1}{4} \xi \eta (\xi - 1)(\eta + 1) & N_8 &= \frac{1}{2} (1 - \xi) (1 - \eta^2) \\
N_9 &= (1 - \xi^2)(1 - \eta^2)
\end{align}\end{split}
\end{equation}
\begin{figure}[htbp]
\centering
\capstart

\noindent\sphinxincludegraphics[width=500\sphinxpxdimen]{{shape_functions_Q9}.png}
\caption{Biquadratische Formfunktionen für 2D\sphinxhyphen{}Rechteck\sphinxhyphen{}Elemente (Q2).}\label{\detokenize{chapters/chapter3/Ansatzfunktionen:quadelementsq9n}}\end{figure}

\sphinxAtStartPar
In der Praxis der Finite\sphinxhyphen{}Elemente\sphinxhyphen{}Methode beschränkt man sich jedoch nicht ausschließlich auf Q9\sphinxhyphen{}Elemente, die die vollständige Polynomordnung 2 aufweisen, wie in Abbildung \hyperref[\detokenize{chapters/chapter3/Ansatzfunktionen:pascalquad}]{Abb.\@ \ref{\detokenize{chapters/chapter3/Ansatzfunktionen:pascalquad}}} dargestellt. Serendipity\sphinxhyphen{}Elemente sind im oben beschriebenen Sinne nicht vollständig, erfüllen jedoch die Anforderungen an die räumliche Isotropie, die Verzerrungsfreiheit bei Starrkörperbewegungen und die Abbildung konstanter Verzerrungszustände. Der Begriff „Serendipity“ geht vermutlich auf ein Märchen von H. Walpole, „Die drei Prinzen von Serendip“, zurück, in dem die Prinzen die Fähigkeit besitzen, durch Zufall unverhoffte und glückliche Entdeckungen zu machen. Das zugehörige Element ist in Abbildung \hyperref[\detokenize{chapters/chapter3/Ansatzfunktionen:quadelements}]{Abb.\@ \ref{\detokenize{chapters/chapter3/Ansatzfunktionen:quadelements}}} als \sphinxstylestrong{Q2S} dargestellt. Die Ansatzfunktionen lauten:
\begin{equation}\label{equation:chapters/chapter3/Ansatzfunktionen:2D_Formfunktionen_Q8}
\begin{split}\begin{align}
N_1(\xi ,\eta ) & =\frac{1}{4} (1-\eta ) (1-\xi ) (-\eta -\xi -1) & N_2(\xi ,\eta ) & =\frac{1}{4} (1-\eta ) (\xi +1) (-\eta +\xi -1) \\
N_3(\xi ,\eta ) & =\frac{1}{4} (\eta +1) (\xi +1) (\eta +\xi -1) & N_4(\xi ,\eta ) & =\frac{1}{4} (\eta +1) (1-\xi ) (\eta -\xi -1) \\
N_5(\xi ,\eta ) & =\frac{1}{2} (1-\eta ) \left(1-\xi ^2\right) & N_6(\xi ,\eta ) & =\frac{1}{2} \left(1-\eta ^2\right) (\xi +1)\\
N_7(\xi ,\eta ) & =\frac{1}{2} (\eta +1) \left(1-\xi ^2\right) &
N_8(\xi ,\eta ) & =\frac{1}{2} \left(1-\eta ^2\right) (1-\xi )
\end{align}\end{split}
\end{equation}
\begin{figure}[htbp]
\centering
\capstart

\noindent\sphinxincludegraphics[width=500\sphinxpxdimen]{{shape_functions_Q8}.png}
\caption{Biquadratische Serendipity\sphinxhyphen{}Formfunktionen für 2D\sphinxhyphen{}Rechteck\sphinxhyphen{}Elemente (Q2S).}\label{\detokenize{chapters/chapter3/Ansatzfunktionen:quadelementsq8n}}\end{figure}

\sphinxAtStartPar
Eine weiter Möglichkeit der zweidimensionalen Vernetzung ist die Verwendung von Dreiecken. In den gebräulichen FEM\sphinxhyphen{}Programmen werden dabei Dreiecke mit 3 oder 6 Knoten implementiert. Diese sind in \hyperref[\detokenize{chapters/chapter3/Ansatzfunktionen:triangleelements}]{Abb.\@ \ref{\detokenize{chapters/chapter3/Ansatzfunktionen:triangleelements}}} dargestellt.

\begin{figure}[htbp]
\centering
\capstart

\noindent\sphinxincludegraphics[width=500\sphinxpxdimen]{{TriangleElements}.png}
\caption{Dreieck\sphinxhyphen{}Elemente mit linearen (T1) und quadratischen (T2) Formfunktionen.}\label{\detokenize{chapters/chapter3/Ansatzfunktionen:triangleelements}}\end{figure}

\sphinxAtStartPar
Die linearen Dreieck\sphinxhyphen{}Elemente (T1) sind die einfachsten Elemente und werden in der Literatur häufig als „konstant“ bezeichnet. Dies bezieht sich auf die abgeleiteten Größen (z.B. Spannungen, Dehnungen, etc.). Diese werden als konstant dargestellt. Die Formfunktionen lauten:
\begin{equation}\label{equation:chapters/chapter3/Ansatzfunktionen:2D_Formfunktionen_T1}
\begin{split}\begin{align}
\lambda & = 1 - \xi - \eta & N_1(\xi ,\eta ) & = \lambda \\
N_2(\xi ,\eta ) & = \xi & N_3(\xi ,\eta ) & = \eta
\end{align}\end{split}
\end{equation}
\sphinxAtStartPar
Die Formfunktionen für Dreieck\sphinxhyphen{}Elemente mit 6 Knoten (T2) sind vollständig quadratisch. Insgesamt hat das T2\sphinxhyphen{}Element bessere Approximationseigenschaften als das T1\sphinxhyphen{}Element. Die Formfunktionen lauten:
\begin{equation}\label{equation:chapters/chapter3/Ansatzfunktionen:2D_Formfunktionen_T2}
\begin{split}\begin{align}
\lambda & = 1 - \xi - \eta & N_1(\xi ,\eta ) & = \lambda (2\lambda - 1) \\
N_2(\xi ,\eta ) & = \xi (2\xi - 1) & N_3(\xi ,\eta ) & = \eta (2\eta - 1) \\
N_4(\xi ,\eta ) & = 4\lambda \xi & N_5(\xi ,\eta ) & = 4\xi \eta \\
N_6(\xi ,\eta ) & = 4\lambda \eta
\end{align}\end{split}
\end{equation}

\subsubsection{3D Formfunktionen}
\label{\detokenize{chapters/chapter3/Ansatzfunktionen:id4}}
\begin{figure}[htbp]
\centering
\capstart

\noindent\sphinxincludegraphics[width=500\sphinxpxdimen]{{HexElements}.png}
\caption{3D Hexaeder Elemente mit linearer (8 Knoten) und quadratischer (27 Knoten) Approximation. Zusätzlich ist das 20\sphinxhyphen{}Knoten Hexaeder Serendipity\sphinxhyphen{}Element gezeigt.}\label{\detokenize{chapters/chapter3/Ansatzfunktionen:hexelements}}\end{figure}

\sphinxAtStartPar
Im 3D\sphinxhyphen{}Fall werden die Formfunktionen für Hexaeder\sphinxhyphen{}Elemente als Tensorprodukt der 1D\sphinxhyphen{}Formfunktionen definiert. Für ein 3D\sphinxhyphen{}Hexaeder\sphinxhyphen{}Element mit 8 Knoten (lineares Element) sind die Formfunktionen:
\begin{equation}\label{equation:chapters/chapter3/Ansatzfunktionen:3D_ShapeFunctions_H8}
\begin{split}\begin{align}
N_1 &= \frac{1}{8} (1 - \xi)(1 - \eta)(1 - \zeta) & N_2 &= \frac{1}{8} (1 + \xi)(1 - \eta)(1 - \zeta) \\
N_3 &= \frac{1}{8} (1 + \xi)(1 + \eta)(1 - \zeta) & N_4 &= \frac{1}{8} (1 - \xi)(1 + \eta)(1 - \zeta) \\
N_5 &= \frac{1}{8} (1 - \xi)(1 - \eta)(1 + \zeta) & N_6 &= \frac{1}{8} (1 + \xi)(1 - \eta)(1 + \zeta) \\
N_7 &= \frac{1}{8} (1 + \xi)(1 + \eta)(1 + \zeta) & N_8 &= \frac{1}{8} (1 - \xi)(1 + \eta)(1 + \zeta)
\end{align}\end{split}
\end{equation}
\sphinxAtStartPar
Um ein vollständiges Element der Ordnung 2 zu erhalten, benötigt man 27 Knoten beim Hexaeder\sphinxhyphen{}Element. Die Formfunktionen kann man zum Beispiel {[}\hyperlink{cite.quellen:id9}{Wriggers, 2008}{]} entnehmen. Mehr Verwendung findet das 20\sphinxhyphen{}Knoten\sphinxhyphen{}Element. Hierbei handelt es sich um ein Serendipity\sphinxhyphen{}Element, das nicht vollständig von der Ordnung 2 ist, aber die Bedingungen der Raumisotropie, keine Verformung bei starren Körperbewegungen und die Möglichkeit, konstante Spannungszustände zu modellieren, erfüllt.

\sphinxAtStartPar
Für tetraedrische Elemente implementieren gängige FEM\sphinxhyphen{}Programme Tetraeder mit entweder 4 oder 10 Knoten. Diese sind in Abbildung \hyperref[\detokenize{chapters/chapter3/Ansatzfunktionen:tetraelements}]{Abb.\@ \ref{\detokenize{chapters/chapter3/Ansatzfunktionen:tetraelements}}} dargestellt.

\begin{figure}[htbp]
\centering
\capstart

\noindent\sphinxincludegraphics[width=500\sphinxpxdimen]{{TetElements}.png}
\caption{Tetraedrische Elemente mit linearen (Tet1) und quadratischen (Tet2) Formfunktionen.}\label{\detokenize{chapters/chapter3/Ansatzfunktionen:tetraelements}}\end{figure}

\sphinxAtStartPar
Die linearen Tetraederelemente (T4) sind die einfachsten und werden häufig als „konstante“ Elemente in der Literatur bezeichnet, wobei sich auf die abgeleiteten Größen (z.B. Spannungen, Dehnungen, etc.) bezieht. Die Formfunktionen lauten:
\begin{equation}\label{equation:chapters/chapter3/Ansatzfunktionen:3D_ShapeFunctions_T4}
\begin{split}\begin{align}
\lambda &= 1 - \xi - \eta - \zeta & N_1(\xi ,\eta ,\zeta ) &= \lambda \\
N_2(\xi ,\eta ,\zeta ) &= \xi & N_3(\xi ,\eta ,\zeta ) &= \eta & N_4(\xi ,\eta ,\zeta ) &= \zeta
\end{align}\end{split}
\end{equation}
\sphinxAtStartPar
Die Formfunktionen für tetraedrische Elemente mit 10 Knoten (T10) sind vollständig quadratisch und bieten bessere Approximationseigenschaften als das T4\sphinxhyphen{}Element. Die Formfunktionen lauten:
\begin{equation}\label{equation:chapters/chapter3/Ansatzfunktionen:3D_ShapeFunctions_T10}
\begin{split}\begin{align}
\lambda &= 1 - \xi - \eta - \zeta & N_1(\xi ,\eta ,\zeta ) &= \lambda (2\lambda - 1) \\
N_2(\xi ,\eta ,\zeta ) &= \xi (2\xi - 1) & N_3(\xi ,\eta ,\zeta ) &= \eta (2\eta - 1) \\
N_4(\xi ,\eta ,\zeta ) &= \zeta (2\zeta - 1) & N_5(\xi ,\eta ,\zeta ) &= 4\lambda \xi \\
N_6(\xi ,\eta ,\zeta ) &= 4\xi \eta & N_7(\xi ,\eta ,\zeta ) &= 4\lambda \eta \\
N_8(\xi ,\eta ,\zeta ) &= 4\lambda \zeta & N_9(\xi ,\eta ,\zeta ) &= 4\xi \zeta \\
N_{10}(\xi ,\eta ,\zeta ) &= 4\eta \zeta
\end{align}\end{split}
\end{equation}
\sphinxstepscope


\section{Integration}
\label{\detokenize{chapters/chapter3/Integration:integration}}\label{\detokenize{chapters/chapter3/Integration::doc}}
\sphinxAtStartPar
Betrachten wir die schwache Form der Impulsbilanz:
\begin{equation*}
\begin{split}\int_{\mathcal{B}} \delta\eb\T  \bm{\sigma} \dV = \int_{\mathcal{B}} \delta \bm{u}\T \rho \bm{b} \dV + \int_{\partial\mathcal{B}} \delta \bm{u}\T  \bm{t} \dA \; .\end{split}
\end{equation*}
\sphinxAtStartPar
Das die Ableitung der Testfunktion \(\delta \bm{u}\) und der Verschiebung \(\bm{u}\) berechnet werden muss, wurde bereits im vergangenen Abschnitt erläutert. Zusätzlich muss der Term \(\delta\eb\T  \bm{\sigma}\) über das gesamte Volumen des Körpers \(\mathcal{B}\) integriert werden. Dabei wird in der Finite\sphinxhyphen{}Elemente\sphinxhyphen{}Methode die Integration über das Volumen des Körpers \(\mathcal{B}\) in eine Summe über die Volumen der Elemente \(\mathcal{B}^e\) zerlegt:
\begin{equation*}
\begin{split}\int_{\mathcal{B}} \delta\eb\T  \bm{\sigma} \dV = \bigcup_{e=1}^{n_{\text{el}}} \int_{\mathcal{B}^e} \delta\eb\T  \bm{\sigma} \dV_e \; .\end{split}
\end{equation*}
\sphinxAtStartPar
Nun kann die Integration auf Elementebene durchgeführt werden. Diese sind aber in der Regel krummlinig berandet und die Integration über das Volumen eines Elements \(\mathcal{B}^e\) ist nicht trivial. Daher wird die Integration über das Volumen des Elements \(\mathcal{B}^e\) in eine Integration über das Volumen des Referenzelements \(\mathcal{B}_{\square}\) umgewandelt:
\begin{equation*}
\begin{split}\int_{\mathcal{B}^e} \delta\eb\T  \bm{\sigma} \dV_e = \int_{\mathcal{B}_{\square}} \delta\eb\T  \bm{\sigma} \left| \bm{J} \right| \dV_{\square} \; .\end{split}
\end{equation*}
\sphinxAtStartPar
Dabei ist \(\left| \bm{J} \right|\) der Determinante der Jacobi\sphinxhyphen{}Matrix \(\bm{J}\), die die Transformation von \(\mathcal{B}_{\square}\) nach \(\mathcal{B}^e\) beschreibt. Damit können wir die Funktion \(\delta\eb\T  \bm{\sigma}\) jetzt über eine sehr einfache Geometrie integrieren. Sowohl der Integrand \(\delta\eb\T  \bm{\sigma}\) als auch die Determinante der Jacobi\sphinxhyphen{}Matrix \(\left| \bm{J} \right|\) sind in der Regel nicht konstant, ja sogar gebrochen rationale Funktionen. Die Integrale können daher nicht analytisch gelöst werden. Die bisherige exakte Transformation des Intergrals muss nun durch eine numerische Integration ersetzt approximiert werden. Die numerischen Integrationsformeln werden im Folgenden auch als \sphinxstyleemphasis{Quadraturformeln} bezeichnet. Allgemein gilt für eine Quadraturformel:
\begin{equation*}
\begin{split}\int_{\mathcal{B}_{\square}} f(\bm{\xi}) \dV_{\square} \approx \sum_{i=1}^{n_{\text{qp}}} w_i f(\bm{\xi}_i) \; .\end{split}
\end{equation*}
\sphinxAtStartPar
Die Integration der Funktion \(f(\bm{\xi})\) über das Gebiet \(\mathcal{B}_{\square}\) wird also ersetzt durch eine Summe über die Funktionswerte \(f(\bm{\xi}_i)\) an den Integrationspunkten \(\bm{\xi}_i\) multipliziert mit den Gewichten \(w_i\). Die Anzahl der Integrationspunkte \(n_{\text{qp}}\) hängt von der gewünschten Genauigkeit der Integration ab. Die Integrationspunkte und Gewichte sind für die im vorangegangenen Abschnitt beschriebenen Elemente in Tabellen zusammengestellt. Die Genauigkeit der Integration hängt maßgeblich von der Form der Funktion \(f(\bm{\xi})\) ab. Allgemein gilt, dass die Genauigkeit der Integration mit der Anzahl der Integrationspunkte steigt. Die in den FE\sphinxhyphen{}Programmen meißt verwendeten Integrationspunkte und Gewichte entsprechen den Gauß\sphinxhyphen{}Quadraturregeln oder sind zumindest an diese angelehnt. Bei der Gauß\sphinxhyphen{}Quadraturregel werden die Integrationspunkte so gewählt, dass die Integration einer Polynomfunktion \(f(\bm{\xi})\) bis zu einem bestimmten Grad genau ist. Die Gewichte \(w_i\) sind so gewählt, dass die Summe der Gewichte gleich der Fläche des Integrationsgebiets ist. Mit einer Integrationsordnung \(m\) kann mit der Gauss\sphinxhyphen{}Quadratur ein Polynom \((2m − 1)\)\sphinxhyphen{}ten Grades exakt integriert werden.
Betrachten wir erneut die schwache Form und schreiben den Integranden \(\delta\eb\T  \bm{\sigma}\) als \(\delta\eb\T  \bm{\sigma} = \delta\eb\T  \mathbb{C} \eb\), dann sehen wir, dass rechts und links der Materialtangente \(\mathbb{C}\) die die räumliche Ableitung von \(\bm{u}_h\) und \(\delta\bm{u}_h\) steht. Wenn wir nun quadratische Formfunktionen verwenden, dann ist die Ableitung der Formfunktionen zumindest in einer Koordinatenrichtung weiterhin quadratisch. Durch die Multiplikation im Integranden müssen wir also ein Polynom 4. Grades integrieren. Nach der obigen Definition ist hierfür eine Quadraturfomel 3. Grades erforderlich.

\begin{sphinxadmonition}{note}{Notwendige Ordnung der Integration}

\sphinxAtStartPar
Für die Integration der Steifigkeitsmatrix der FEM mit Elementen mit Formfunktionen der Ordnung \(n\) ist eine Quadraturformel der Ordnung:
\begin{equation*}
 m = \frac{n+1}{2} 
 \end{equation*}
\sphinxAtStartPar
erforderlich.
\end{sphinxadmonition}

\sphinxAtStartPar
Nachfolgende Tabellen zeigen die 1D\sphinxhyphen{} und 2D\sphinxhyphen{}Quadraturformeln.

\begin{figure}[htbp]
\centering
\capstart

\noindent\sphinxincludegraphics[width=500\sphinxpxdimen]{{Wriggers_GQ_01}.png}
\caption{1D\sphinxhyphen{}Gauß\sphinxhyphen{}Quadraturformeln für verschiedene Ordnungen \(m\) nach {[}\hyperlink{cite.quellen:id9}{Wriggers, 2008}{]}.}\label{\detokenize{chapters/chapter3/Integration:linequadrature}}\end{figure}

\begin{figure}[htbp]
\centering
\capstart

\noindent\sphinxincludegraphics[width=500\sphinxpxdimen]{{Wriggers_GQ_02}.png}
\caption{2D\sphinxhyphen{}Gauß\sphinxhyphen{}Quadraturformeln für Rechteckelemente verschiedener Ordnung \(m\) nach {[}\hyperlink{cite.quellen:id9}{Wriggers, 2008}{]}.}\label{\detokenize{chapters/chapter3/Integration:squarequadrature}}\end{figure}

\begin{figure}[htbp]
\centering
\capstart

\noindent\sphinxincludegraphics[width=500\sphinxpxdimen]{{Wriggers_GQ_03}.png}
\caption{2D\sphinxhyphen{}Gauß\sphinxhyphen{}Quadraturformeln für Dreickselementen verschiedener Ordnung \(m\) nach {[}\hyperlink{cite.quellen:id9}{Wriggers, 2008}{]}.}\label{\detokenize{chapters/chapter3/Integration:triquadrature}}\end{figure}


\subsection{Beispiel: 1D\sphinxhyphen{}Integration}
\label{\detokenize{chapters/chapter3/Integration:beispiel-1d-integration}}
\sphinxAtStartPar
Wir möchte das Polynom \(f(x) = \frac{1}{2} x^4 + 6x + 4\) über das Intervall \([-1,1]\) integrieren. Die exakte Lösung ist:
\begin{equation*}
\int_{-1}^{1} (\frac{1}{2} x^4 + 6x + 4) \, \mathrm{d}x = \left[\frac{1}{10} x^5+2x^3+4x \right]_{-1}^{1} = 12.2
\end{equation*}
\sphinxAtStartPar
Da wir im Interval \([-1,1]\) integrieren, können wir direkt die Gauß\sphinxhyphen{}Quadraturformeln verwenden. Wir müssen nicht die Jacobideterminante berechnen.
Verwenden wir jetzt eine Integrationsordnung mit einem Integrationspunkt, so ergibt sich:
\begin{align*}
\xi_p &= 0  & w_p &= 2 \\
\int_{-1}^{1} f(x) \, \mathrm{d}x &= \sum_{p=1}^{1} w_p f(\xi_p) \\
\int_{-1}^{1} f(x) \, \mathrm{d}x &= 2 \cdot 4  = 8
\end{align*}
\sphinxAtStartPar
Mit dieser Integrationsordnung ergibt sich also ein Fehler von 34 \%.

\sphinxAtStartPar
Als nächstes nehmen wir eine Integrationsordnung mit zwei Integrationspunkten:
\begin{align*}
\xi_p &= \pm \frac{1}{\sqrt{3}} & w_p &= 1 \\
\int_{-1}^{1} f(x) \, \mathrm{d}x &= \sum_{p=1}^{2} w_p f(\xi_p)  \\
\int_{-1}^{1} f(x) \, \mathrm{d}x &= 1 \cdot 6.06  + 1 \cdot 6.06 = 12.111
\end{align*}
\sphinxAtStartPar
Hier ergibt sich nur noch ein Fehler von 1 \%.

\sphinxAtStartPar
Erst eine Integrationsordnung mit drei Integrationspunkten liefert das exakte Ergebnis:
\begin{align*}
\xi_p &= 0, \pm \sqrt{\frac{3}{5}} & w_p &= \frac{8}{9}, \frac{5}{9} \\
\int_{-1}^{1} f(x) \, \mathrm{d}x &= \sum_{p=1}^{3} w_p f(\xi_p) \\
\int_{-1}^{1} f(x) \, \mathrm{d}x &= \frac{5}{9} \cdot 7.78  + \frac{8}{9} \cdot 4 + \frac{5}{9} \cdot 7.78  = 12.2
\end{align*}
\sphinxstepscope


\section{Postprocessing}
\label{\detokenize{chapters/chapter3/Postprocessing:postprocessing}}\label{\detokenize{chapters/chapter3/Postprocessing::doc}}
\sphinxAtStartPar
Das Postprocessing stellt einen unverzichtbaren Schritt in der Finite\sphinxhyphen{}Elemente\sphinxhyphen{}Analyse (FEA) dar. Es ist definiert als der Prozess der Transformation und Aufbereitung der oft hochdetaillierten und komplexen Rohausgaben von FEM\sphinxhyphen{}Berechnungen in ein für den Anwender leicht verständliches Format. Im Wesentlichen handelt es sich um die Modifikation oder Verbesserung von Ergebnisdaten, nachdem eine Lösung aus einem Rechenmodell erhalten wurde, mit dem Ziel, aussagekräftige Visualisierungen, Diagramme und andere Ausgaben zu erstellen.

\begin{figure}[htbp]
\centering
\capstart

\noindent\sphinxincludegraphics[width=500\sphinxpxdimen]{{postproc_ansys}.png}
\caption{Postprocessing in ANSYS Mechanical nach \sphinxhref{https://www.padtinc.com/simulation/simulation-product/ansys-mechanical/}{link}.}\label{\detokenize{chapters/chapter3/Postprocessing:postprocansys}}\end{figure}


\subsection{Postprocessing primärer Größen}
\label{\detokenize{chapters/chapter3/Postprocessing:postprocessing-primarer-groszen}}
\sphinxAtStartPar
In der Finite\sphinxhyphen{}Elemente\sphinxhyphen{}Analyse werden die sogenannten primären Größen direkt als Ergebnis der Lösung des globalen Gleichungssystems berechnet. Typische Beispiele hierfür sind Verschiebungen in der Strukturmechanik oder Temperaturen in der Wärmeübertragung. Diese Größen werden explizit an den Knoten des Finite\sphinxhyphen{}Elemente\sphinxhyphen{}Netzes ermittelt. Die resultierenden Knotenverschiebungen gelten als präzise, da jedem Knoten innerhalb des Netzes ein einziger, eindeutiger Verschiebungswert zugewiesen wird. Dies unterscheidet sich grundlegend von den sekundären Größen, die später erläutert werden.\\
Die Werte der primären Feldvariablen, die an den diskreten Knoten berechnet werden, dienen dazu, die Werte an allen anderen Punkten innerhalb eines Elements  zu approximieren. Dies geschieht durch Interpolation der Knotenwerte mittels der Formfunktionen des Elementes.
Dies gewährleistet die Inter\sphinxhyphen{}Element\sphinxhyphen{}Kontinuität der primären Feldvariablen (z.B. Verschiebung oder Temperatur) an den Knotenpunkten. Die Formfunktionen stellen somit die mathematische Brücke dar, die es ermöglicht, die diskrete Lösung an den Knoten in ein kontinuierliches Feld innerhalb jedes Elements zu überführen. Diese Interpolationsfähigkeit ist von grundlegender Bedeutung für die FEM, da sie es erlaubt, mit einer endlichen Anzahl von Freiheitsgraden ein kontinuierliches physikalisches Phänomen anzunähern.


\begin{savenotes}\sphinxattablestart
\sphinxthistablewithglobalstyle
\centering
\begin{tabulary}{\linewidth}[t]{TTT}
\sphinxtoprule
\sphinxstyletheadfamily 
\sphinxAtStartPar
Kriterium
&\sphinxstyletheadfamily 
\sphinxAtStartPar
Primäre Größen (z.B. Verschiebung, Temperatur)
&\sphinxstyletheadfamily 
\sphinxAtStartPar
Sekundäre Größen (z.B. Spannung, Dehnung, Moment)
\\
\sphinxmidrule
\sphinxtableatstartofbodyhook
\sphinxAtStartPar
\sphinxstylestrong{Typ}
&
\sphinxAtStartPar
Feldvariablen, direkte Lösung
&
\sphinxAtStartPar
Abgeleitete Größen
\\
\sphinxhline
\sphinxAtStartPar
\sphinxstylestrong{Berechnungspunkt}
&
\sphinxAtStartPar
Knoten
&
\sphinxAtStartPar
Gaußpunkte (dann zu Knoten extrapoliert)
\\
\sphinxhline
\sphinxAtStartPar
\sphinxstylestrong{Kontinuität}
&
\sphinxAtStartPar
C0\sphinxhyphen{}kontinuierlich (an Knoten)
&
\sphinxAtStartPar
Diskontinuierlich an Elementgrenzen
\\
\sphinxhline
\sphinxAtStartPar
\sphinxstylestrong{Genauigkeit (relativ)}
&
\sphinxAtStartPar
Präzise (eindeutiger Wert pro Knoten)
&
\sphinxAtStartPar
Weniger präzise (aber genauer an Gaußpunkten)
\\
\sphinxhline
\sphinxAtStartPar
\sphinxstylestrong{Rolle im Postprocessing}
&
\sphinxAtStartPar
Direkte Ausgabe, Visualisierung
&
\sphinxAtStartPar
Erfordert Glättung für physikalische Interpretation
\\
\sphinxbottomrule
\end{tabulary}
\sphinxtableafterendhook\par
\sphinxattableend\end{savenotes}


\subsection{Postprocessing sekundärer Größen}
\label{\detokenize{chapters/chapter3/Postprocessing:postprocessing-sekundarer-groszen}}
\begin{figure}[htbp]
\centering
\capstart

\noindent\sphinxincludegraphics[width=300\sphinxpxdimen]{{FEM_Postprocessing}.png}
\caption{Unterschied zwischen primären und sekundären Größen im Postprocessing.}\label{\detokenize{chapters/chapter3/Postprocessing:postprocfem}}\end{figure}

\sphinxAtStartPar
Im Gegensatz zu primären Größen werden sekundäre Größen, wie Spannungen, Dehnungen und Momente, nicht direkt an den Knoten berechnet. Stattdessen werden diese abgeleiteten Größen an den sogenannten Gauß\sphinxhyphen{}Integrationspunkten innerhalb jedes Finite\sphinxhyphen{}Elements ermittelt.\\
Der Grund für die Berechnung an Gaußpunkten liegt in der mathematischen Formulierung der FEM, insbesondere im Rahmen der Galerkin\sphinxhyphen{}Methode. Die Berechnung von Spannungen und Dehnungen erfolgt aus den Ableitungen der Verschiebungsfelder. Die Gaußpunkte sind die optimalen Punkte für die numerische Integration der Steifigkeitsmatrizen und Lastvektoren und bieten die genauesten Stellen zur Bestimmung dieser abgeleiteten Größen. Der Element\sphinxhyphen{}Solver „kennt“ die Dehnung innerhalb des Elements basierend auf den Knotenverschiebungen und kann daraus die Spannung berechnen. Es ist jedoch wichtig zu verstehen, dass das Element kein „vollständiges Wissen“ über das Geschehen in seinem Inneren besitzt; es prognostiziert lediglich einige korrekte Antworten an den Gaußpunkten und schätzt bzw. extrapoliert diese wenigen genauen Antworten auf den Rest seiner Fläche.\\
Um die an den Gaußpunkten berechneten Werte der sekundären Größen an den Knotenpositionen zu erhalten, ist eine Extrapolation dieser Werte von den Gaußpunkten zu den Knoten erforderlich. Diese Extrapolation ist notwendig, da die Knoten oft die primären Punkte für die Ergebnisdarstellung und \sphinxhyphen{}interpretation sind.\\
Es existieren verschiedene Methoden zur Extrapolation von Gaußpunktwerten zu den Knoten, die jeweils unterschiedliche Implikationen für die Ergebnisdarstellung und \sphinxhyphen{}genauigkeit haben:
\begin{itemize}
\item {} 
\sphinxAtStartPar
\sphinxstylestrong{Zentroidaler Wert (Centroidal Value):} Bei dieser einfachen Methode wird der Durchschnitt der Gaußpunkt\sphinxhyphen{}Ergebnisse über alle Gaußpunkte eines Elements berechnet und dieser gemittelte Wert allen Knoten des jeweiligen Elements zugewiesen. Dies führt zu einer sehr starken Glättung der Ergebnisse innerhalb eines Elements.

\item {} 
\sphinxAtStartPar
\sphinxstylestrong{Nodalwerte, extrapoliert von Gaußpunkten (Nodal Values Extrapolated from Gauss Points):} Diese Methode verwendet lineare Formfunktionen, um die Gaußpunktwerte zu den Knoten zu extrapolieren. Die Extrapolation entspricht einem Least\sphinxhyphen{}Squares\sphinxhyphen{}Fit. Sie wird im Allgemeinen für die Mehrheit der linearen Materialmodelle bevorzugt. Eine wichtige Einschränkung ist jedoch, dass diese Option für nichtlineare Materialien unter Umständen nicht gültig ist, da die Extrapolation von Gaußpunktspannungen zu den Knoten zu nodal Spannungen führen könnte, die die Materialgrenzen überschreiten, was physikalisch inkorrekt wäre.

\item {} 
\sphinxAtStartPar
\sphinxstylestrong{Gaußpunktwerte direkt an Knoten platziert (Gauss Point Values Placed at Nodes):} Hierbei wird jedes Gaußpunkt\sphinxhyphen{}Ergebnis direkt seinem nächstgelegenen Knoten zugewiesen, ohne weitere Extrapolation. Diese Option ist in der Regel am zuverlässigsten für nichtlineare Materialien, da die direkt zugewiesenen Knotenwerte niemals die Materialgrenzen überschreiten werden.

\end{itemize}

\sphinxAtStartPar
Die grundlegende Natur sekundärer Größen als „abgeleitete“ Größen hat weitreichende Auswirkungen auf ihre Genauigkeit. Während primäre Größen direkt aus der Lösung der Gleichgewichtsgleichungen resultieren , werden sekundäre Größen durch Rücksubstitution aus den primären Größen berechnet. Da die FEM eine Näherungstechnik ist, propagieren Fehler in den primären Größen und können sich in den abgeleiteten sekundären Größen sogar verstärken, was zu einer höheren Fehleranfälligkeit führt. Die Wahl der Extrapolationsmethode ist somit nicht willkürlich, sondern hängt maßgeblich vom Materialmodell ab. Eine lineare Extrapolation kann für nichtlineare Materialien physikalisch unmögliche Ergebnisse liefern, was eine konservativere, direkte Zuweisung erforderlich macht. Dies verdeutlicht die Notwendigkeit, das zugrunde liegende Materialverhalten und dessen Wechselwirkung mit den numerischen Methoden genau zu verstehen, da dies die Gültigkeit und den physikalischen Realismus der postprozessierten Ergebnisse direkt beeinflusst.


\begin{savenotes}\sphinxattablestart
\sphinxthistablewithglobalstyle
\centering
\begin{tabulary}{\linewidth}[t]{TTTT}
\sphinxtoprule
\sphinxstyletheadfamily 
\sphinxAtStartPar
Methode
&\sphinxstyletheadfamily 
\sphinxAtStartPar
Prinzip
&\sphinxstyletheadfamily 
\sphinxAtStartPar
Anwendungsbereich / Vorteile
&\sphinxstyletheadfamily 
\sphinxAtStartPar
Nachteile / Überlegungen
\\
\sphinxmidrule
\sphinxtableatstartofbodyhook
\sphinxAtStartPar
\sphinxstylestrong{Zentroidaler Wert}
&
\sphinxAtStartPar
Durchschnitt aller Gaußpunktwerte eines Elements wird allen Knoten zugewiesen.
&
\sphinxAtStartPar
Einfache, schnelle Glättung.
&
\sphinxAtStartPar
Stark geglättet, verliert lokale Details.
\\
\sphinxhline
\sphinxAtStartPar
\sphinxstylestrong{Nodalwerte, extrapoliert von Gaußpunkten}
&
\sphinxAtStartPar
Lineare Formfunktionen extrapolieren Gaußpunktwerte zu Knoten.
&
\sphinxAtStartPar
Bevorzugt für lineare Materialmodelle.
&
\sphinxAtStartPar
Kann bei nichtlinearen Materialien Materialgrenzen überschreiten.
\\
\sphinxhline
\sphinxAtStartPar
\sphinxstylestrong{Gaußpunktwerte direkt an Knoten platziert}
&
\sphinxAtStartPar
Jedes Gaußpunkt\sphinxhyphen{}Ergebnis wird direkt dem nächstgelegenen Knoten zugewiesen.
&
\sphinxAtStartPar
Am zuverlässigsten für nichtlineare Materialien, da Materialgrenzen nicht überschritten werden.
&
\sphinxAtStartPar
Führt zu Diskontinuitäten an Elementgrenzen, da keine Glättung erfolgt.
\\
\sphinxbottomrule
\end{tabulary}
\sphinxtableafterendhook\par
\sphinxattableend\end{savenotes}

\sphinxstepscope


\part{Diskretisierung}

\sphinxstepscope


\chapter{Diskretisierung}
\label{\detokenize{chapters/chapter4/Diskretisierung:diskretisierung}}\label{\detokenize{chapters/chapter4/Diskretisierung::doc}}
\sphinxstepscope


\section{Fehlerquellen}
\label{\detokenize{chapters/chapter4/Fehlerquellen:fehlerquellen}}\label{\detokenize{chapters/chapter4/Fehlerquellen::doc}}
\sphinxstepscope


\part{Quellenverzeichnis}

\sphinxstepscope


\chapter{Quellenverzeichnis}
\label{\detokenize{quellen:quellenverzeichnis}}\label{\detokenize{quellen::doc}}
\begin{sphinxthebibliography}{GHSchrod}
\bibitem[Bat06]{quellen:id19}
\sphinxAtStartPar
Klaus\sphinxhyphen{}Jürgen Bathe. \sphinxstyleemphasis{Finite element procedures}. Klaus\sphinxhyphen{}Jurgen Bathe, 2006.
\bibitem[Bra13]{quellen:id14}
\sphinxAtStartPar
Dietrich Braess. \sphinxstyleemphasis{Finite elemente: Theorie, schnelle löser und anwendungen in der elastizitätstheorie}. Springer\sphinxhyphen{}Verlag, 2013.
\bibitem[Cha13]{quellen:id10}
\sphinxAtStartPar
Eduardo WV Chaves. \sphinxstyleemphasis{Notes on continuum mechanics}. Springer Science \& Business Media, 2013.
\bibitem[EM13]{quellen:id15}
\sphinxAtStartPar
Klaus Ehrlenspiel and Harald Meerkamm. Integrierte produktentwicklung: denkabläufe, methodeneinsatz. \sphinxstyleemphasis{Hanser, Zusammenarbeit}, 2013.
\bibitem[Fea14]{quellen:id11}
\sphinxAtStartPar
Roy Featherstone. \sphinxstyleemphasis{Rigid body dynamics algorithms}. Springer, 2014.
\bibitem[GHSchroderW07]{quellen:id13}
\sphinxAtStartPar
Dietmar Gross, Werner Hauger, Jörg Schröder, and Wolfgang A Wall. \sphinxstyleemphasis{Technische Mechanik: Band 2: Elastostatik}. Springer, 2007.
\bibitem[GABlugelM09]{quellen:id5}
\sphinxAtStartPar
Johannes Grotendorst, Norbert Attig, Stefan Blügel, and Dominik Marx. Multiscale simulation methods in molecular sciences. \sphinxstyleemphasis{Lecture Notes, NIC Series}, 42:145, 2009. URL: \sphinxurl{https://juser.fz-juelich.de/record/3737/files/nic-series-volume42.pdf}.
\bibitem[Hau13]{quellen:id7}
\sphinxAtStartPar
Peter Haupt. \sphinxstyleemphasis{Continuum mechanics and theory of materials}. Springer Science \& Business Media, 2013.
\bibitem[Hug03]{quellen:id18}
\sphinxAtStartPar
Thomas JR Hughes. \sphinxstyleemphasis{The finite element method: linear static and dynamic finite element analysis}. Courier Corporation, 2003.
\bibitem[KW91]{quellen:id4}
\sphinxAtStartPar
Klaus Knothe and Heribert Wessels. \sphinxstyleemphasis{Finite elemente}. Volume 2. Springer, 1991.
\bibitem[Nac22]{quellen:id17}
\sphinxAtStartPar
Prof. Dr.\sphinxhyphen{}Ing. Udo Nackenhorst. Numerische mechanik. 2022. Vorlesungsskript, Sommersemester.
\bibitem[NGorkeMK16]{quellen:id8}
\sphinxAtStartPar
Thomas Nagel, Uwe\sphinxhyphen{}Jens Görke, Kevin M Moerman, and Olaf Kolditz. On advantages of the kelvin mapping in finite element implementations of deformation processes. \sphinxstyleemphasis{Environmental Earth Sciences}, 75:1–11, 2016.
\bibitem[Sha97]{quellen:id6}
\sphinxAtStartPar
Ahmed A Shabana. Flexible multibody dynamics: review of past and recent developments. \sphinxstyleemphasis{Multibody system dynamics}, 1:189–222, 1997. URL: \sphinxurl{https://citeseerx.ist.psu.edu/document?repid=rep1\&type=pdf\&doi=83ab06769ba4d9ea6a300799675760db2bf69ac9}.
\bibitem[VWBZ09]{quellen:id16}
\sphinxAtStartPar
Sandor Vajna, Christian Weber, Helmut Bley, and Klaus Zeman. \sphinxstyleemphasis{CAx für Ingenieure: eine praxisbezogene Einführung}. Springer\sphinxhyphen{}Verlag, 2009.
\bibitem[Wri08]{quellen:id9}
\sphinxAtStartPar
Peter Wriggers. \sphinxstyleemphasis{Nonlinear finite element methods}. Springer Science \& Business Media, 2008.
\bibitem[WNB+06]{quellen:id12}
\sphinxAtStartPar
Peter Wriggers, Udo Nackenhorst, Sascha Beuermann, Holger Spiess, and Stefan Löhnert. \sphinxstyleemphasis{Technische Mechanik Kompakt}. Springer, 2006.
\bibitem[ZTZ05]{quellen:id3}
\sphinxAtStartPar
Olgierd Cecil Zienkiewicz, Robert Leroy Taylor, and Jian Z Zhu. \sphinxstyleemphasis{The finite element method: its basis and fundamentals}. Elsevier, 2005.
\end{sphinxthebibliography}







\renewcommand{\indexname}{Stichwortverzeichnis}
\printindex
\end{document}